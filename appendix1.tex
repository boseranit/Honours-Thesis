
\chapter{Reference Guide}
\label{appendix1}

\section{Reference Guide}
\cite{cernTQFT} is a 63 page introduction to TQFT, including the Mathai-Quillen
formalism and Donaldson Witten theory using twisted $N=2$ supersymmetry.
Lacking details

\cite{wittenTQFT} is the seminal article on twisted supersymmetric gauge theory.
Including how donaldson invariants arise

\cite{birminghamTFT} is a 366pg textbook on supersymmetry, BRST, euler
character, sigma models, donaldson theory, Schwarz type TFTs, topological
gravity and renormalization

\cite{cordes95} is 247pg of lecture notes, and focuses on equivariant cohomology
and TQFT from pg 75. Moves through topics very quickly.

\cite{axiomTQFTintro} Motivates the functorial axiomatisation of TQFT using the
path integral approach

\cite{marino} 40 pages, goes through Donaldson invariants, supersymmetry and donaldson
witten theory, by stating a number of result, but without detail and lacking
proofs. 

\cite{moore} 80 pg, similar to marino. Focuses on path integral approach to
cohomological TFT, and Mathai quillen

\cite{TQFTbook} More detailed version of marino's lectures on TFT. 

\cite{MQformula} is the original paper by Mathai and Quillen which introduced a
formula for the euler number.

\cite{atiyahlagrangians} a short article showing how Witten's Lagrangian should
be understood in terms on the Gauss-Bonnet formula. Starts with Mathai-Quillen
formula for Thom class

\cite{MQintro} is a 35 pg introductory accound of the previous paper, beginning
with Mathai-Quillen formalism for finite dimen vector bundles




