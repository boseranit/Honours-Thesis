\chapter*{Preface}\label{preface}

\addcontentsline{toc}{chapter}{Preface}
Weyl (1918) introduced the concepts of gauge transformation and gauge
invariance, while seeking to unify electromagnetism with general relativity. After the
development of quantum mechanics, an idea was to make the global phase symmetry
into a local gauge symmetry by replacing the momentum operator with a covariant 
derivative in the Schrodinger equation, given by 
$\widehat{p}=\frac{\hbar}{i}(\nabla+iA)$. The result of this $\U(1)$ gauge symmetry
turns out to be profound, since it introduces EM interactions with a charged
quantum mechanical particle. This was the first widely accepted gauge theory,
and popularised by Pauli (\citeyear{pauli_em}).\cite{pauli_em}
Gauge theory was mostly limited to EM and GR until the paper of Yang and Mills 
(1954). They introduced $\SU(2)$ gauge theories for
isotopic spin axes, to understand the strong interaction. This idea later found
application to the model of the electroweak interaction by Weinberg and Salam
(1967), and also the Higgs mechanism (1966) for spontaneous symmetry breaking
allowing gauge fields to acquire mass. 

This also motivated the search for a strong force gauge theory, which is now
known as quantum chromodynamics. The Standard Model unifies the description of
electromagnetism, weak interactions and strong interactions in the language of
quantized gauge theory. 

The origin of TQFT can be traced to the work of Schwarz and Witten.
Schwarz showed that the Ray-Singer torsion, a particular topological invarint,
could be represented as the partition function of a certian quantum field theory
(1978). Unrelated to this observation, Witten gave a framework for understanding
Morse theory in terms of supersymmetric quantum mechanics (1982), which was 
generalised to an infinite dimensional version by Floer. 

In \citeyear{don83}, Donaldson built on his doctoral advisor Atiyah's work on 
Yang-Mills instantons to introduce his famous polynomial invariants, and prove 
Donaldson's theorem.\cite{don83}
Freedman used this work to exhibit existence of exotic differentiable
structures on $\mathbb{R}^{4}$. 
Witten (\citeyear{wittenTQFT}) provided a way
of obtaining the polynomials by formulating a supersymmetric TQFT, which became
known as Donaldson-Witten theory.\cite{wittenTQFT}
Another approach to obtain topological invariants, related to Donaldson's
construction, comes from Seiberg-Witten theory (1994). They formulated a set of
equations that contains the same information as the Yang-Mills equations but
are technically much easier to work with. 

\section{Introduction}

A quantum field theory is called a TQFT when the correlation functions are
independent of the metric. 
TQFT has been an active area of research ever since the seminal work by Witten
\cite{wittenTQFT}. TQFT contains no excitations that may propagate in the
spacetime, so it does not describe any waves we know in the real world. The
characteristic quantity describing a configuration: the action, remains
invariant under any continuous changes of the topology. 

There are two ways to formally guarantee that the correlation functions remain
invariant under variations of the metric. % labatista, lozano
\begin{itemize}
	\item Schwarz type TFT: the action and the operators are defined without 
		using the metric of the manifold. The most notable example is
		Chern-Simmons gauge theory. Another import set of examples are the BF
		theories. 
	\item Witten type TFT: there is explicit metric
	dependence, but the theory has a scalar symmetry $\delta$ acting on
	the fields such that the correlation
	functions do not depend on the background metric. 
\end{itemize}

This thesis will focus on a particular Witten type TQFT: Donaldson-Witten
theory. Witten-type TQFTs can be formulated in a variety of frameworks. 
\begin{itemize}
	\item 
 The most geometric one corresponds to the Mathai-Quillen formalism. In this
formalism a TQFT is constructed out of a moduli problem. Topological invariants
are then defined as integrals of a certain Euler class over the resulting moduli
space.\cite{cernTQFT}
	\item 
 A different framework is the one based on the twisting of N = 2 supersymmetry.
In this case, information on the physical theory can be used in the
TQFT. (Seiberg-Witten invariants have shown up in this framework)
\end{itemize}



Assumed knowledge:

Manifolds, Lie groups, Differential forms, integration on manifolds, de Rham cohomology
(\citetitle{intro_tu} by Loring Tu \cite{intro_tu}, or Introduction to Smooth Manifolds by
John Lee) 

Connections, curvature and characteristic classes \cite{loringtu}

\begin{comment}
	For supersymmetry chapter: 
	\citetitle{hall} \citet{hall}
\end{comment}
