\chapter{Donaldson-Witten theory in MQ formalism}
\label{chapter4}
% See AJ intro
In finite dimensions, the Mathai-Quillen formula gives an explicit differential
form representative of the Euler class. 
It was shown by Atiyah and
Jeffrey \cite{atiyahlagrangians} that not only can the zero dimensional
Donaldson invariant can be identified with the Euler number of a vector bundle
over $\mathcal{A} /\mathcal{G}$, but the Donaldson invariants in general can be
written as an integral of a Mathai-Quillen type form over the gauge equivalence
classes of irreducible connections, similar to (reference Gauss-Bonnet formula
with Mathai-Quillen).
Moreover, they have been able to reproduce Witten's
action functional from twisted SUSY YM theory term by term from purely geometric
considerations.

Conversely,
the Mathai-Quillen formalism makes it possible to construct a cohomological field
theory starting from a moduli problem.


\section{Atiyah-Jeffrey formula}
This section derives another version of the Mathai-Quillen
Thom form, which will be then be formally applied to the Donaldson-Witten context. 
We consider the following setup: $G$ is a compact connected Lie group
of dimension $d$ acting on a principal $G$-bundle  $P\to M$ of dimension  $2m+d$.
By a similar construction as in Theorem \ref{thm:lie_inner_product}, we can
construct an inner product on $P$ which is invariant under the action of  $G$,
by averaging any Riemannian metric on $P$ with respect to the Haar measure on
$G$. At each point $p\in P$, this Riemannian metric defines an orthogonal
complement to the vertical tangent space. Since these subspaces are invariant
under the action of $G$, this determines a connection  $\omega$ on  $P$.
Moving forward, we will solely use this connection.
Also let $V$ be a real vector space of dimension  $2m$  with a
representation $\rho : G \to \SO(V)$, so we can form the associated
vector bundle $E:= P\times_G V$. 

% AJ p124, Naber p123
Our starting point is the universal Mathai-Quillen formula
(equation \ref{eq:universal_thom_form}), an element in the
Cartan model $S(\mathfrak{g}^*)\otimes \Omega(V)$, restated here
\begin{equation} \label{eq:universal_thom_form2}
	U = (2\pi)^{-m}e^{-\abs{x}^2}\int^B
	\exp\paren{\frac{1}{2}\chi^{\intercal}\rho_*\chi - idx^\intercal \chi}
	\odif{\chi}
\end{equation}
where $\chi = (e_1,\ldots,e_{2m})$ is a basis for $V$.

\vspace{1ex}\noindent
\textbf{Manipulation 1: Replace $\Omega$ with  $d\omega$} \\
Recall that we can obtain a Thom form on $E$ via the Cartan map, which is given
by $\alpha\to \operatorname{Hor}(p_2^*\alpha(\Omega)) \in \Omega(P\times V)_{bas}$, 
where $p_2:P\times V\to V$ is the projection onto $V$.  
The curvature is related to the connection by 
$\Omega = d\omega + \frac{1}{2}[\omega\wedge \omega]$, but $\omega=0$ on
horizontal vectors (by definition of horizontal subspace). Since the Cartan map
then only evaluates on the horizontal projection of tangent vectors, we can 
replace  $\Omega$ by  $d\omega$ in the Cartan map. 

\vspace{1ex}\noindent
\textbf{Manipulation 2: Replace $d\omega$ with  $R^{-1}dC^*$} \\
Recall that $C_p : \mathfrak{g} \to V_pP$ defined by  $C_p(X) = \odv{}{t}_{t=0}
(p\cdot \exp(tX))$ is the canonical identification of the Lie algebra with the
vectical tangent space at $p\in P$. Both spaces are inner product vector spaces,
and hence $C_p$ has an adjoint  $C_p^* : T_pP \to \g$. For $X,Y\in \g$, it
satisfies 
\[
	\gen{C_p(X), C_p(Y)} = \gen[*]{C_p^*C_p(X),Y}_{\g} = \gen{R_p X, Y}_{\g}
\] 
where $R := C_p^*\circ C_p : \g \to \g$. It is clear that $R_p$ is self-adjoint
and an isomorphism since $C_p$ is an isomorphism. 
Also note that $C_p^*$ vanishes on horizontal vectors, as 
$\gen[*]{X,C_p^*(v)} = \gen{C_p(X),v}$ vanishes on
all horizontal $v\in T_p$ due to $C_p(X)\in V_pP$. 

If $\omega\in \Omega(P,\g)$ is 
the connection 1-form, then $\omega_p(X) = C_p^{-1}(X)$. Thus, we have the
pointwise matrix equation $C^* = R\omega$, or $\omega = R^{-1}C^*$. From this,
we compute its differential 
\[
d\omega = R^{-1} dC^* + dR^{-1} \wedge C^*
\] 
The last term vanishes on a pair of horizontal vectors, so again, we can replace
$d\omega$ with $R^{-1}dC^*$, since the Cartan map will only evaluate on the
horizontal vectors.


\vspace{1ex}\noindent
\textbf{Manipulation 3: Double Fourier transform to avoid inverting $R$} \\
The next objective is to remove the explicit inverse $R^{-1}$ by using the
Fourier inversion formula. Recall that the Fourier transform is an automorphism
of the Schwartz space on any vector space $W$. If  $dw\in\bigwedge^n W$ is a
volume element with corresponding dual volume element $dy\in \bigwedge^nW^*$,
then the Fourier inversion formula states that 
\[
	f(w) =
	(2\pi)^{-n}\int_{W^*}\int_{W}e^{i\gen{w,y}}e^{-i\gen{z,y}}f(z)\odif{z}\odif{y}
\] 
Note that for a real vector space the integral is the same if we multiply the
exponent by -1.
If we identify $W$ with  $W^*$ via some inner product, this becomes a double
integral over $W$. For a self-adjoint matrix $R$ with positive determinant, we
can compute  $f(R^{-1}w)$ by making the change of variables $w \to R^{-1}w$ and
$y\to Ry$, in which case $\gen{R^{-1}w,Ry}=\gen{w,y}$ and $d(Ry)=\det R dy$. 
The inversion formula becomes
\[
f(R^{-1}w) = (2\pi)^{-n}\iint_W e^{i\gen{w,y}}e^{-i\gen{z,Ry}}f(z)\det R\odif{z}\odif{y}
\] 
We now consider the universal Mathai-Quillen element as a $\Omega(V)$-valued 
function $f:\g \to \Omega(V)$ on the vector space $\g$, 
\[
f(X) = (2\pi)^{-m}e^{-\abs{x}^2}\int^B
	\exp\paren{\frac{1}{2}\chi^{\intercal}\rho_*(X)\chi - idx^\intercal \chi}
	\odif{\chi}
\] 
which we wish to
evaluate at $R^{-1}dC^* \in \Omega^2(P,\g)$. By the inversion formula above,
\begin{align} \label{eq:fourier_thom}
	f(R^{-1}dC^*) = (2\pi)^{-d-m}e^{-\abs{x}^2}\iint_{\g}\int^B 
	&\exp\Big(\frac{1}{2}\chi^{\intercal}\rho_*(X)\chi - idx^\intercal \chi\\
	&+i\gen{dC^*,\lambda}-i\gen{\phi,R\lambda}\Big) \det R
	\odif{\chi} \odif{\phi}\odif{\lambda} \nonumber
\end{align}
where $\lambda, \phi \in \g$ are Lie algebra variables. 
\begin{remark} % naber p126
	Note that the original function $f(X)$ is a polynomial in $X\in\g$, and
	therefore not in the Schwartz space. This can be made more precise by
	inserting a rapidly decaying test function $e^{-\epsilon\gen{X,X}}$ and
	taking the limit as $\epsilon\to 0$.
\end{remark}

\vspace{1ex}\noindent
\textbf{Manipulation 4: Horizontal projection via integration along $G$-orbits} \\
% naber p127
The final step is to take the horizontal projection of the element in
$\Omega^{2m}(P\times V)$. Our strategy is to take the wedge product with a
certain 1-form $W\in\Omega^d(P\times V)$ which contains all vertical components. 
Thus the result will
only contain terms which did not have a vertical part but now with a factor of 
$W$. We would then need to integrate out the vertical part of these terms so
that the result is the horizontal part of the original element in
$\Omega(P\times V)$.

Denote by $\omega \in \Omega^1(P\times V,\g)$ the pullback connection on
$P\times V$, and let $\eta_1,\ldots,\eta_d$ be an orthonormal basis for $\g$. We claim that
\[
W := \int^B \exp(\omega) d\eta 
= \int^B \exp(\omega_i\eta_i) d\eta = \omega_1 \wedge \cdots \omega_d
\] 
is the desired vertical volume form. The integral along the fiber is defined in
the standard way.??? see wikipedia

Using a local trivialisation, we can 
by pulling back to a form on $G$ via the diffeomorphism
$G\xrightarrow{\simeq} \pi^{-1}(x)$ induced by a local trivialisation. Then
integrating using the Haar measure from before, 
normalised so that the volume of $G$ is 1. 
The integral of $W$ is 1 ???

\section{Interpretation of Donaldson-Witten theory}
We can now apply the Atiyah-Jeffrey Thom form to the infinite dimensional
setting of Donaldson-Witten theory. 
The spaces we now consider are 
\begin{itemize}
	\item principal bundle $P$: space of irreducible connections  $\mathcal{A}$ 
		over a compact oriented 4-manifold $M$
	\item structure group $G$: group of bundle automorphisms  $\mathcal{G}$ 
	\item vector space $V$: self dual 2-forms $\Omega^{2,+}(\ad P)$ ??
	\item section $s:P\to V$: self-dual part of curvature  $s=-F^+$
\end{itemize}
Donaldson only treats the case $G=\SU(2)$ because of singularities in the moduli
space, however we do not worry about this (??). Since our objective is to apply the
formula to the infinite dimensional vector space $\Omega^{2,+}(\ad P)$, complete
rigor (for this particular application) is out of the question anyway. 

Although $e(E)$ and  $\int_X e(E)$ do not make sense for
infinite dimensional  $E$ and  $X$, the Mathai-Quillen form  $e_{s,\Delta}$ can
be used to formally define regularized Euler numbers $\chi_s(E) := \int_X
e_{s,\Delta}(E)$. Although not independent of $s$, these numbers  $\chi_s(E)$
are of topological interest for certain choices of  $s$.  


\section{TQFT as a generalisation of Mathai-Quillen}
% section 4.2 MQintro


\vspace{5mm}
\hrule 
\vspace{5mm}

\textbf{Bibliographical notes}
{\small
\begin{itemize}
	\item The lecture notes by \citet{cordes95} explores these topics from the
	from the perspective of physics, and how it relates to twisted
	$\mathcal{N}=2$ topological field theory.
	\item 
\end{itemize}
}


