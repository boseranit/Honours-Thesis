
\chapter{Donaldson-Witten theory}
\label{chapter4}

The space of gauge connections $\mathcal{A}$ on a principal bundle is the
universe bundle for the group of gauge transformations $\mathcal{G}$. (ch 15
Cordes)

\section{TQFT as a generalisation of Mathai-Quillen}
% section 4.2 MQintro
 This approach is
due to \cite{atiyahlagrangians}, who showed that TQFT can be regarded as an
infinite dimensional generalisation of the Mathai-Quillen construction. 
Although $e(E)$ and  $\int_X e(E)$ do not make sense for
infinite dimensional  $E$ and  $X$, the Mathai-Quillen form  $e_{s,\Delta}$ can
be used to formally define regularized Euler numbers $\chi_s(E) := \int_X
e_{s,\Delta}(E)$. Although not independent of $s$, these numbers  $\chi_s(E)$
are of topological interest for certain choices of  $s$.  


\section{Donaldson-Witten theory}
This is a retelling of Witten's 1988 TQFT paper. 
% https://physics.stackexchange.com/questions/224297/what-to-a-physicist-are-instantons-and-the-donaldson-invariants

The physical setting in which the Donaldson invariants will appear is a
Yang-Mills theory living on the four-dimensional manifold $M$ coupled to certain
fields such that the total action has a supersymmetry. The action of this theory
is 
\begin{align*}
	S &= \int \Tr \Big(\frac{1}{4}F_{\mu\nu}F^{\mu\nu} +\frac{1}{4}F_{\mu\nu}(\star
F)^{\mu\nu} +\frac{1}{2}\phi D_\mu D^\mu\lambda - i\eta D_\mu\psi^\mu +
iD_\mu\psi_\nu\chi^{\mu\nu} \\ 
	  &\qquad - \frac{i}{2}\lambda[\psi_\mu,\psi^\mu]-\frac{i}{2}\phi[\eta,\eta]-\frac{1}{8}[\phi,\lambda]^2\Big)\sqrt{g} \odif[order=4]{x}
\end{align*}
where $D_\mu = \nabla_\mu + [A_\mu,+]$ is the gauge covariant derivative and
$\nabla$ is the Riemannian covariant derivative, and all fields are
$\mathfrak{g}$-valued. There is a  $\mathbb{Z}_2$-grading on the space of
fields. The bosonic fields are $\phi,\lambda$ and the fermionic fields are
$\eta,\psi,\chi$ and  $\chi$ is additional constrained to be self-dual. The
action is invariant under the symmetry 
\begin{align*}
	\delta_\epsilon A = \mathrm{i}\epsilon\psi \quad \delta_\epsilon\phi = 0
	\quad \delta_\epsilon\lambda = 2\mathrm{i}\epsilon\eta \\
	\delta_\epsilon\psi = -\epsilon D\phi \quad \delta_\epsilon\eta =
	\frac{1}{2}\epsilon[\phi,\lambda] \quad \delta_\epsilon \chi = \epsilon(F +
	\ast F) 
\end{align*}
with $\epsilon$ a fermionic infinitesimal parameter. As with all
transformations, we think of this one as having a generator: its supercharge
$Q$, which gives all transformations as  $\delta \alpha=-i\epsilon Q(\alpha)$
where $\alpha$ is any field. in a Hamiltonian formulation,  $Q(\alpha)$ would be
the Poisson bracket  $\{Q,\alpha\}$, but on a general manifold we don't have
that option. By explicit computation, we find that 
\[
	\delta_\epsilon\delta_\zeta X - \delta_\zeta\delta_\epsilon X =
	-2\mathrm{i}\epsilon\zeta\phi
\] 
for every field $X$ except  $A$, where it is  $-D\phi$. This holds only on-shell
for  $\chi$, but off-shell for all others. Therefore, the commutator of two such
transformations is a gauge transformation, and hence has no physical impact. The
conserved current associated to this symmetry is 
 \[
	 J = \mathrm{Tr}_\mathfrak{g}\left((F_{\mu\nu} + (\ast F)_{\mu\nu})\psi^\nu
		 - \eta D_\mu \phi - D^\nu\phi \chi_{\mu\nu} -
	 \frac{1}{2}\psi_\mu[\lambda,\phi]\right)\mathrm{d}x^\mu
\] 
where conservation means that $\star J$ is closed, so for any homology 3-cycle
 $\gamma$, the integral  $Q(\gamma) = \int_\gamma \star J$ depends only on the
 homology class of  $\gamma$. Furthermore, one may show that the energy momentum
 tensor  $T_{\mu\nu}= 2 \frac{\delta S}{\delta g^{\mu\nu}}$ of this theory is an
 infinitesimal transform $T_{\mu\nu}=\{Q,\lambda_{\mu\nu}\}$ for $\lambda$ given
 in Witten's eq.(2.34). 

\section{Donaldson invariants as path integrals}
In the following, the path integral measure $\mathcal{D}X$ includes all fields,
and also intends to have gauge equivalence calsses quotiented out. The generic
object we consider is the (unnormalizedD expectation value of any observable
$\mathcal{O}$ which is any nice functional in the fields:
\[
Z(\mathcal{O}) = \int \mathcal{O} \exp(-S /e^2) \mathcal{D}X
\] 
If the supersymmetry transformation is non-anomalous, we have
$Z(\{Q,\mathcal{O}\})= 0$ for every observable. We now claim that $Z=Z(1)$ is a
smooth invariant, and in particular will turn out to be a Donaldson invariant.
For  $Z$ to be a smooth invariant, it must be invariant under changes in the
metric. The change of metric is by definition  $\delta
S=\frac{1}{2}\int_M\sqrt{g} T_{\mu\nu}\delta g^{\mu\nu}$ and this leads to
\[
	\delta Z(1) = -\frac{1}{e^2}Z(\{Q,\int_M \sqrt{g}\delta
	g^{\mu\nu}\lambda_{\mu\nu}\}) = 0
\] 
so $Z(1)$ is invariant under changes of the metric. Similarly, it is invariant
under changes of the gauge coupling constant  $e$,   as long as it stays
non-zero. But in the limit of small coupling, the path integral is strongly
dominated by the classical minima of the free theories, and the calssical minima
of the free gauge theory are the anti-self-dual instantons. The self-dual ones
are not minima beause we have added $F\wedge F$ to the Lagrangian. So we may
evaluate  $Z$ by looking at the instanton contributions. 



