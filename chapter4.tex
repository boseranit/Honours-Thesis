\chapter{Donaldson-Witten theory in MQ formalism}
\label{chapter4}

In finite dimensions, the Mathai-Quillen formula gives an explicit differential
form representative of the Euler class. However it was noticed by Atiyah and
Jeffrey \cite{atiyahlagrangians} that the action principle of the theory can be
recovered by constructing the Euler class of $\mathcal{E}$ using an extension of
the Mathai-Quillen representative to the infinite-dimensional case. Conversely,
the Mathai-Quillen formalism makes possible to construct a cohomological field
theory starting from a moduli problem.

keywords: Donaldson-Witten theory, Donaldson invariants, see AJ intro

\section{Atiyah-Jeffrey formula}
This section provides a more detailed exposition of Atiyah and Jeffrey's paper
\cite{atiyahlagrangians}, which derives another version of the Mathai-Quillen
Thom form. We consider the following setup: $G$ is a compact connected Lie group
acting freely on an oriented Riemannian manifold  $P$ of dimension  $2m+d$,
where  $\dim G = d$. Let  $V$ be a real vector space of dimension  $2m$  with a
representation $\rho : G \to \SO(V)$. Since  $G$ has a free action on  $P$, 
 $P\to P /G = X$ is a principal $G$-bundle, and we can form the associated
vector bundle $E:= P\times_G V$. 



\section{TQFT as a generalisation of Mathai-Quillen}
% section 4.2 MQintro
 This approach is
due to \citet{atiyahlagrangians}, who showed that TQFT can be regarded as an
infinite dimensional generalisation of the Mathai-Quillen construction. 
Although $e(E)$ and  $\int_X e(E)$ do not make sense for
infinite dimensional  $E$ and  $X$, the Mathai-Quillen form  $e_{s,\Delta}$ can
be used to formally define regularized Euler numbers $\chi_s(E) := \int_X
e_{s,\Delta}(E)$. Although not independent of $s$, these numbers  $\chi_s(E)$
are of topological interest for certain choices of  $s$.  


\vspace{5mm}
\hrule 
\vspace{5mm}

\textbf{Bibliographical notes}
{\small
\begin{itemize}
	\item The lecture notes by \citet{cordes95} explores these topics from the
	from the perspective of physics, and how it relates to twisted
	$\mathcal{N}=2$ topological field theory.
	\item 
\end{itemize}
}


