\chapter*{Introduction}

\addcontentsline{toc}{chapter}{Introduction}

\section*{Brief history of gauge theory}
In order to provide the context for the main topic of this thesis, we begin with
a brief history of gauge theory. There is a long-standing tradition of rich
areas of mathematics originating from physics, of which gauge theory and
Donaldson theory is no exception.

In 1918, Weyl introduced the concepts of gauge transformation and gauge
invariance, while seeking to unify electromagnetism with general relativity. After the
development of quantum mechanics, an idea was to make the global phase symmetry
into a local gauge symmetry by replacing the momentum operator with a covariant 
derivative in the Schrodinger equation, given by 
$\widehat{p}=\frac{\hbar}{i}(\nabla+iA)$. The result of this $\U(1)$ gauge symmetry
turns out to be profound, since it introduces eletromagnetic interactions with a charged
quantum mechanical particle. This was the first widely accepted gauge theory,
and popularised by Pauli (\citeyear{pauli_em}).\cite{pauli_em}

Gauge theory was mostly limited to electromagnetism and general relativity until 
the paper of Yang and Mills (1954). They extended the concept of gauge theory 
to non-abelian groups 
to understand the strong interaction. This idea later found
application to the model of the electroweak interaction which incorporated the 
Higgs mechanism by Weinberg and Salam (1967).
Yang-Mills theory also lead to the development of a strong force gauge theory, 
which is now known as quantum chromodynamics. The Standard Model unifies the description of
electromagnetism, weak interactions and strong interactions in the language of
quantized gauge theory. 
\begin{comment}
Since symmetries of spacetime and fields are
central to classification and predicting properties of particles, the search for
a non-trivial extension of the Poincare group lead to supersymmetry being 
independently discovered in 1971 by multiple physicists. It became increasingly
popular for its potential to provide an elegant solution to unsolved problems in
particle physics, such as the hierarchy problem, grand unification and dark
matter. 
\end{comment}

\section*{Donaldson-Witten theory}
The development of the study of solutions to the Yang-Mills equations 
has proven to be fruitful for differential topology.
In \citeyear{don83}, Simon Donaldson built on his doctoral advisor Atiyah's work on 
Yang-Mills instantons to introduce his famous polynomial invariants, and prove 
Donaldson's theorem.\cite{don83}
In particular, he constructed invariants of
smooth four dimensional manifolds from moduli spaces of anti-self-dual
connections on principal $\SU(2)$-bundles.
These invariants are sensitive to differentiable structures, whereas typical
invariants in topology are stable under homeomorphisms.
Furthermore, Freedman used his work to exhibit the existence of manifolds that
are homeomorphic but not diffeomorphic to Euclidean $\mathbb{R}^{4}$. 

However, Donaldson's work did not
indicate any relation to physical ideas which Yang-Mills theory is based
on. As a result, topologists with little or no knowledge of physics were able to 
learn and make important contributions to Donaldson theory. However, if physics
had lead to such a profound impact, perhaps there was more to be gained from a
physical formulation of the theory. Indeed this was the case, as we will see
next. 

The origin of TQFT can be traced to the work of Albert Schwarz and Edward Witten.
Schwarz (1978) showed that the Ray-Singer torsion, a particular topological invarint,
could be represented as the partition function of a certain quantum field theory. 
Unrelated to this observation, Witten gave a framework in supersymmetric (SUSY)
quantum mechanics that lead to a generalisation of Morse theory (1982).
It turns out that Donaldson invariants on a cylinder $Y^3\times \mathbb{R}$ is
formally a Morse theory for a certain action functional. Thus,
\citet{wittenTQFT} (\citeyear{wittenTQFT}) applied his idea
to Donaldson invariants on a $Y^3\times \mathbb{R}$ to obtain a $N=2$ 
SUSY quantum theory. But since this Morse theory is on a set of 
gauge fields, we get a QFT. In the same paper, he came up with a
relativistically covariant twisted version of QFT by generalising the existing 
fields and adding new fields, in which the Donaldson polynomials appeared as 
correlation functions of certain observables. This became known as Donaldson-Witten theory.

% Naber intro 
This construction is a remarkable achievement because it showed a profound
connection between SUSY TQFT and Donaldson theory that one might
expect. Topologists were eager to understand the insights that gave rise to the 
radically new view of smooth four-manifold invariants. However, this area of theoretical physics
was accessible to relatively few mathematicians. 
Fortunately, Atiyah and Jeffrey showed how to arrive at the very same action by 
purely geometric means. Starting with the universal Mathai-Quillen formula, they
made a series of manipulations which resulted in the same action of
Donaldson-Witten theory, but which could now be interpreted as an Euler form on
an infinite-dimensional vector bundle.

Donaldson-Witten theory caught the attention of many, but this
approach did not lead many mathematicians to gain new insight about Donaldson
invariants, for a few reasons. Aside from the time and effort required to learn
the relevant physics, the mathematics required to make the path integrals
rigourous was not (and is still not) available. But in 1994, Witten's
SUSY TQFT revealed its practical significance in a dramatic way. 
\begin{comment}
Nathan Seiberg had discovered new techniques to show that
certain SUSY theories constrains the form of the effective Lagrangian to such an
extent that quantum corrections vanish in higher orders in perturbation theory.
\end{comment}
Seiberg and Witten discovered new dualities in SUSY
theories, which illustrates features such as quark confinement. These ideas led to
the unification of string theories into $M$-theory, and other developments in
SUSY theories.  
They showed that the dual theory to Donaldson-Witten theory also has SUSY and
the same twist obtains a new TQFT. The corresponding correlation functions,
called the Seiberg-Witten invariants, revolutionised the study of 4-manifolds
once again. 

The new invariants are technically far easier to compute, and also leads to new
simpler proofs in Donaldson theory. 
Moreover, the duality predicts particular formulas that relate Donaldson
invariants to Seiberg-Witten invariants, although a rigorous proof has still not
yet been found. 

\begin{comment}
A quantum field theory is called a TQFT when the correlation functions are
independent of the metric. The
characteristic quantity describing a configuration: the action, remains
invariant under any continuous changes of the topology. 
TQFT has been an active area of research ever since the seminal work by Witten
\cite{wittenTQFT}. TQFT contains no excitations that may propagate in the
spacetime, so it does not describe any waves we know in the real world. 

There are two ways to formally guarantee that the correlation functions remain
invariant under variations of the metric. % labatista, lozano
\begin{itemize}
	\item Schwarz type TFT: the action and the operators are defined without 
		using the metric of the manifold. The most notable example is
		Chern-Simmons gauge theory. Another import set of examples are the BF
		theories. 
	\item Witten type TFT: there is explicit metric
	dependence, but the theory has a scalar symmetry $\delta$ acting on
	the fields such that the correlation
	functions do not depend on the background metric. 
\end{itemize}

Donaldson-Witten theory is a particular TQFT. 
Witten-type TQFTs can be formulated in a variety of frameworks. 
\begin{itemize}
	\item 
 The most geometric one corresponds to the Mathai-Quillen formalism. In this
formalism a TQFT is constructed out of a moduli problem. Topological invariants
are then defined as integrals of a certain Euler class over the resulting moduli
space.\cite{cernTQFT}
	\item 
 A different framework is the one based on the twisting of N = 2 supersymmetry.
In this case, information on the physical theory can be used in the
TQFT. (Seiberg-Witten invariants have shown up in this framework)
\end{itemize}

Topological quantum fields theories can be constructed based on the
Mathai-Quillen formalism. Although the main breakthroughs in these theories have
been provided in their formulation as twisted $\mathcal{N}=2$ supersymmetric
theories, this provides a useful geometric framework.\cite{TQFTbook}
\end{comment}

\section*{Structure of the thesis}
If there is anything to take away from the history presented, it is that
topologists can benefit from a deeper understanding of Donaldson-Witten theory. 
The goal for this thesis is to provide a thorough explanation of Atiyah and
Jeffrey's interpretation of Witten's correlation functions and Lagrangian in
terms of mathematically familiar concepts. Although the main breakthrough has
been made in in the SUSY TQFT formulation, this provides a useful framework for
mathematicians to think about the relation of Donaldson invariants with physics. 
The structure of the thesis is as follows.

In Chapter \ref{chapter_equivariant}, we motivate 
equivariant cohomology and proceed to describe the Borel construction. This
construction is topological in nature, and we formulate an equivalent algebraic
model called the Weil model, analogous to the de Rham isomorphism. We also
describe the simpler Cartan model, and its properties.

In Chapter \ref{chapter_mq}, we introduce the Thom isomorphism for vector
bundles, and then construct the Mathai-Quillen formula for the Thom form. 
Subsequently, we consider the generalisation of this formula to the Cartan model, 
called the universal Thom form.

In Chapter \ref{chapter_local}, we demonstrate the appliation of the
Mathai-Quillen formula, and more generally the Thom form to localisation of
integrals on manifolds. We prove a theorem stated in \cite{bott_tu} by utilising
some results in intersection theory.

In Chapter \ref{chapter_donaldson}, we sketch the construction of Donaldson
invariants. This requires first describing the Yang-Mills equations, then
explaining the structure of the anti-self-dual moduli space. The theory required
to do this is highly technical, so we do not go into the details of the analytic
arguments, but include proofs of some results needed in the next chapter. 
Finally the Donaldson map is defined. 

In Chapter \ref{chapter_aj}, we finally give an exposition of Atiyah and
Jeffrey's paper.\cite{atiyahlagrangians} We first need to construct an analogue
of the Thom form for a principal bundle. This is used in one of the series of
manipulations of the Mathai-Quillen formula for the universal Thom form. 
The result is then applied to Donaldson-Witten theory to show that the action 
functional is reproduced, which allows us to interpret it as an infinite
dimensional analogue of an Euler class. 

\section*{Assumed knowledge}
The reader is required to be familiar with the theory in introductory texts
on smooth manifolds, such as 
\citetitle{intro_tu} by Loring Tu \cite{intro_tu}, or Introduction to Smooth Manifolds by
John Lee \cite{lee_smooth}. Topics in particular include submanifolds, Lie groups, 
differential forms, integration on manifolds, de Rham cohomology, Riemannian
metrics and the exponential map.

Furthermore we assume the reader is familiar with the theory of connections,
curvature and Chern-Weil theory on vector bundles and principal bundles. 
An excellent textbook for this is \citetitle{loringtu} by Loring Tu
\cite{loringtu}.

The reader is recommended to have a basic background in algebraic
topology, which includes knowledge of simplicial homology and cohomology groups,
and the cup product. 

\begin{comment}
	For supersymmetry chapter: 
	\citetitle{hall} \citet{hall}
\end{comment}

\vspace{5mm}
\hrule 
\vspace{5mm}

\textbf{Bibliographical notes}
{\small
\begin{itemize}
	\item A overview of the main ideas in the two formulations of 
		Donaldson-Witten theory is given in \citet{TQFTbook}, focusing more on
		TQFT.

	\item The article \citetitle{history} \cite{history} presents a 
	more thorough review of the history of the development of Donaldson-Witten
	theory and Seiberg-Witten theory, with brief explanations of the ideas. 
\end{itemize}
}


