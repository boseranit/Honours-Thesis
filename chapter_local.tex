\chapter{Localisation}
\section{Poincar\'e-Hopf theorem}
One of the primitive applications of the Mathai-Quillen formula is to prove the
Poincar\'e-Hopf theorem. 
Let $v\in\mathcal{X}(M)$ be a vector field on $M$.
At a zero $p\in M$ of  $v$, the Lie bracket of vector fields defines an endomorphism 
$\mathcal{L}_p(v) : T_pM \to T_pM$ by  $X\mapsto [X,v]$. In a coordinate system in
which  $p=0$ and  $v= v^i \partial_i$,  $\mathcal{L}_p(v)$ is given by 
 \[
	 \mathcal{L}_p(v)\partial_i = \sum_{j=1}^{n} [\partial_i, v^j(p)\partial^j] 
	 = \sum_{j=1}^{n} (\partial_i v_j(p)) \partial_j
\] 
A zero $p\in M$ is called \underline{non-degenerate} if  $\mathcal{L}_p(v)$ is
invertible. In this case, denote $\nu(p,v) = \sgn \det(\mathcal{L}_p(v)) 
\in \{\pm 1\}$. If all of the zeros of $v$ are non-degenerate, we call  $v$
non-degenerate. 

\begin{thm}[Poincare-Hopf] % Thm 1.56 BGV
	If $v$ is a non-degenerate vector field on an oriented compact manifold $M$ 
	of dimension $n$, then 
	\[
		 \int_M \chi(TM) = \sum_{\set{p|v(p)=0}} \nu(p,v)	
	\] 
\end{thm}
\begin{proof}
	(adapted from \cite[Theorem 1.56]{bgv})
	Choose a chart $\phi_p : U_p \to \mathbb{R}^n$ in a neighbourhood $U_p$ of
	each zero $p\in M$ of $v$, which gives coordinates on the tangent bundle 
	$\psi_p : TU_p \to \mathbb{R}^n$. The vector field $v$ defines a
	smooth map $\psi_p \circ v : U_p\to \mathbb{R}^n$ 
	with $v(p) = 0$ and invertible
	derivative $\mathcal{L}_p(v)$. 
	% TODO why is this the derivative?

	By the inverse function theorem, $v$ is a local diffeomorphism
	between open neighbourhoods $V_p \subset U_p$ and $B \subset \mathbb{R}^n$. 
	The orientation on $V_p$ induced by the oriented chart $\phi_p$ and by 
	the local diffeomorphism $\psi_p \circ v$ differ by the sign $\nu(p,v)$. 

	We may assume  $V_p$ are disjoint for each $p$.  Choose a Riemannian metric
	on $M$ which agrees with the metric on each $V_p$ induced by the diffeomorphism
	into  $\mathbb{R}^n$. Since $M$ is compact,  $\exists \epsilon > 0$ such
	that  $\norm{v} \geq \epsilon$ on the compact set  $M\setminus \bigcup_p V_p$,
	because if $\norm{v}$ can get arbitrarily close to 0, we would find a convergent
	subsequence in $M\setminus \bigcup_p V_p$ whose limit is a zero of $v$.  

	Let $U\in \Omega^n(TM)$ be the Mathai-Quillen Thom form of  $TM$ with respect to the
	Riemannian metric and associated Levi-Civita connection. 
	Let $f : \mathbb{R}_+ \to [0,1]$ be a smooth bump function such that $f(s)=1$
	if  $s < \epsilon^2 /4$ and $f(s)=0$ if  $s>\epsilon^2$. Then for all $t>0$
	\begin{equation} \label{eq:euler_zeros}
			\int_M \chi(TM) = \int_M v_t^* U 
		= \int_M (1-f(\norm{v}^2))v_t^*U  + \int_M f(\norm{v}^2)v_t^*U
	\end{equation}
	where $v_t := tv$. From equation (\ref{eq:MQ_exp}), we see that
	$v_t^* U$ is of the form
	\[
		v_t^*U = e^{-t^2\norm{v}^2 /2} \sum_{k=0}^{n} t^k \alpha_k
	\] 
	where $\alpha_k \in \Omega(M)$. Therefore, the first integral in equation
	(\ref{eq:euler_zeros}) is rapidly decreasing in $t$ since $\norm{v}^2 >
	\epsilon^2 /4$, and approaches zero as $t\to\infty$.

	On each $V_p$, the metric and connection are trivial since they are induced 
	by $v$. If we write $v= \sum_{i=1}^{n} x^i \partial_i$ in the coordinates of
	the chart $\psi_p$,
	\begin{align*}
		v_t^*U |_{V_p} 
		&= (2\pi)^{-n/2} e^{-t^2\norm{x}^2 /2} \int^B e^{-itdv} \\
		&= (2\pi)^{-n/2} \nu(p,v) e^{-t^2\norm{x}^2 /2} t^n dx^1 \wedge
		\ldots\wedge dx^n \tag{using Lemma \ref{lem:gaussian_integral}}
	\end{align*}
	The sign $\nu(p,v)$ comes from the Berezin integral using the orientation of
	the manifold, while $x^1,\ldots,x^n$ are coordinates of the vector field. 
	Therefore, 
	\begin{align*}
		\lim_{t \to \infty} \int_{V_p} f(\norm{v}^2)v_t^*U 
		&= (2\pi)^{-n/2} \nu(p,v) 
		\lim_{t \to \infty} \int_{\mathbb{R}^n}
		f(\norm{x}^2) e^{-t^2\norm{x}^2 /2} t^n dx^1 \wedge \ldots\wedge dx^n\\
		&= (2\pi)^{-n/2} \nu(p,v) 
		\lim_{t \to \infty} \int_{\mathbb{R}^n} 
		f(\norm[*]{t^{-1}y}^2) e^{-\norm{y}^2 /2} dy^1 \wedge \ldots\wedge dy^n\\
		&= \nu(p,v) 
	\end{align*}
	where we have made the change of variables $x=t^{-1}y$. Summing over each of
	the neighbourhoods $V_p$ of zeros, we get the result.
\end{proof}
The Mathai-Quillen formula also gives us intuition on why the Euler number
should only depend on the zeroes of a section. 
The $e^{-\abs{\epsilon s(x)}^2/2}$ term in the formula tells us that as
$\epsilon \to 0$, the integral of the Euler form rapidly decays to zero at any
point outside $s^{-1}(0)$. 

There is a generalisation of this result, where we assume that the zeros
of $v$ are isolated instead of non-degenerate.\cite[Theorem 1.58]{bgv} 
It is desirable to generalise this further to relate the Euler number of
arbitrary vector bundles to zeros of a section, but the Poincar\'e-Hopf theorem
relies on the Lie bracket of vector fields to define the orientation of each
zero. In order to describe such a generalisation we first need to introduce some
intersection theory.

\section{Poincar\'e dual as a Thom class}
% p51 Bott Tu
Let $M$ be an oriented manifold and $i: S \xhookrightarrow{} M$ be a compact,
oriented submanifold with $\dim S = k, \dim M = n$. 
Poincar\'e duality (Theorem \ref{thm:poincare_duality}) can be used to associate 
$S$ to a unique cohomology class $[\eta_S]\in H^{n-k}_c(M)$ called its
\underline{Poincare dual} as follows. Define the linear functional
\[
H^k(M) \to \mathbb{R}, \qquad 
\omega \mapsto \int_S i^*\omega
\] 
Here $\int_S i^*\omega$ is defined because $S$ is compact. 
It follows by Poincare duality that integration over $S$ corresponds to a
unique cohomology class $[\eta_S]\in H^{n-k}_c(M)$, with the property that 
for any $\omega\in H^k(M)$
\begin{equation} \label{eq:poincare_dual_property}
	\int_S i^*\omega = \int_M \omega \wedge \eta_S
\end{equation}
\begin{remark} % TODO and closed (as a subspace) needed?
	The same argument can be applied to define the Poincar\'e dual of an
	oriented submanifold $S \subset M$ that is not
	necessarily compact. This is associated to a linear functional 
	$H^k_c(M) \to \mathbb{R}$ by integrating over $S$. This is well defined
	because for any	compact $K\subset M$, $K\cap S$ is compact in $S$.
	This corresponds to a unique cohomology class $[\eta_S] \in H^{n-k}(M)$,
	satisfying equation (\ref{eq:poincare_dual_property}) for $\omega \in
	H^k_c(M)$.
\end{remark}
Notice that the Poincar\'e dual has a remarkably similar property to the 
Thom class (c.f. equation (\ref{eq:thom_form_property})). 
To relate the two ideas, this requires constructing a vector bundle over the
submanifold $S$, which leads to the concept of a tubular neighbourhood.
\begin{defn}
	The \underline{normal bundle} of $S$ in  $M$ is the quotient vector bundle
	$N_S\to S$ defined by the exact sequence 
	\[
	\begin{tikzcd}[column sep = 1.6em]
		0 \arrow[r] & TS \arrow[r] & TM|_S \arrow[r] 
						& N_S \arrow[r] & 0
	\end{tikzcd}
	\]
	Let $S \subset M$ be a submanifold. 
	A \underline{tubular neighbourhood} of  $S$ in  $M$ is an open
	neighbourhood of  $S$ in  $M$ diffeomorphic to the normal bundle of $S$ in
	 $M$, such that  $S$ is diffeomorphic to the zero section.
\end{defn}
For a Riemannian manifold, one can identify $N_S$ with the orthogonal
complement. If we further assume $S$ and $M$ are oriented, then there is an
induced orientation on the normal bundle via 
\begin{equation} \label{eq:normal_orientation}
	\operatorname{or}(TM|_S) = \operatorname{or}(TS)\wedge \operatorname{or}(N_S)
\end{equation}
where $\operatorname{or}(N_S)$ denotes the orientation form on the vector bundle. 

% https://luis.impa.br/aulas/anvar/Spivak_Vol1_3ed.pdf Spival p346
% https://www.math.tecnico.ulisboa.pt/~acannas/Books/lsg.pdf p37
% \cite[Theorem 6.5]{anacannas}
% http://alpha.math.uga.edu/~usher/8210-notes2.pdf Thm 2.11
\begin{thm}[Tubular neighbourhood theorem {\cite[Thm 5.25]{riemannian_manifolds}}] 
	Every submanifold $S$ of a Riemannian manifold $M$ has a 
	tubular neighbourhood  $T\subset M$. If $S$ is compact, the tubular
	neighbourhood can be chosen to have constant radius.
\end{thm}
\begin{comment}
\begin{proof}[Proof (sketch).]
	Let $V\subset TM$ be the domain of the exponential map, and 
	The idea is that for each $x\in S$, the exponential map is a diffeomorphism
	on a neighbourhood of $x$
	\[
		V_\delta(x) = \{(x',v')\in N_S \mid d_g(x,x')<\delta, \abs{v'}_g<\delta\}
	\] 
	for some $\delta$. It can be shown that 
	 \[
		 \Delta(x) = \sup \{\delta \leq 1 \mid \exp \text{ is a diffeomorphism
		 from} V_\delta \text{ to its image}\}
	\] 
	is a continuous function $\Delta : S \to \mathbb{R}$. Then 
	\[
		U = \{(x,v)\in N_S \mid \abs{v}_g < \frac{1}{2}\Delta(x)\}
	\] 
 	is a neighbourhood of $N_S$ containing the zero section $S$. It can be shown
	that  $\exp$ is injective on  $U$. Therefore  $\exp$ is a diffeomorphism
	from  $U$ to its image  $T$, which shows that  $T$ is a tubular
	neighbourhood.
\end{proof}
\end{comment}
\begin{figure}[htb]
	\hfill
	\begin{minipage}[c]{0.56\textwidth}
		\includegraphics[trim={4cm 3mm 4cm 3.7mm},clip,width=\textwidth]{figs/tubular_neighbourhood.pdf}
	\end{minipage} 
	\begin{minipage}[c]{0.3\textwidth}
        \caption{A tubular neighbourhood}
        \label{fig:tubular_neighbourhood}
	\end{minipage} 
\end{figure}
The proof of the theorem shows that the tubular neighbourhood is the
diffeomorphic image under the exponential map of a subset of the form
\begin{equation} \label{eq:tubular_radius}
	V^\delta_g = \set[*]{(x,v)\in N_S \mid \abs{v}_g < \delta(x)}
	\;\mapsto\; 
	T^\delta_g = \set{x \in M \mid \operatorname{dist}_g(x,S) < \delta(x)}
\end{equation}
for some positive smooth function $\delta : S \to \mathbb{R}$, called the
radius, which can be chosen as small as desired. Note that $V^\delta_g$ is 
diffeomorphic to  $N_S$. In this construction, the fiber over $x\in S$ is
\begin{equation} \label{eq:tubular_fiber}
		T^\delta_g(x) = \{z \in M \mid \operatorname{dist}_g(z,S) 
		= d_g(z,x) < \delta(x)\}
\end{equation}

We can now explain the relation between the Poincar\'e dual and Thom class.
Let $S$ be an oriented submanifold of an oriented manifold $M$. 
% which is closed as a subspace. % Not using closed homology, see BottTu p51 
If $j:T \hookrightarrow M$ is the inclusion of a tubular 
neighborhood of $S$, we have a Thom class $U \in H^{n-k}_{cv}(T)$ by identifying 
$T$ with the normal bundle of $S$. 
We can define a map $j_*:H_{cv}^{*}(T) \to H^*(M)$ which extends by zero,
because forms that are compactly supported along the fiber go to zero near 
the boundary of $T$. 
% The following theorem illuminates the relation between the Poincar\'e dual 
% and Thom class: % repetitive

\begin{thm}[Poincar\'e dual as a Thom class] \label{thm:poincare_thom} % p67 Bott Tu
	The Poincar\'e dual $\eta_S$ of $S$ is equal to the Thom class 
	$U\in H^{n-k}_{cv}(T)$ of the
	tubular neighbourhood of $S$. That is, 
	\[
		\int_M \omega\wedge j_*U = \int_S i^*\omega \qquad
		\text{for all } \omega\in H^k(M)
	\] 
	where $U$ is defined by identifying 
	$T$ with the normal bundle.
\end{thm}
\begin{proof}
	Let $\omega \in H^k(M)$. Consider $\omega$ as a form on  $T$, since
	we are not concerned with its values outside this region, so we may view
	the inclusion $i : S \to T$ as the zero section. Note that $\pi : T \to S$
	induces a deformation retraction of $T$ onto $S$, so $\pi$ and  $i$
	are inverse isomorphisms in cohomology. This means $\omega$ differs from
	$\pi^*i^*\omega$ by an exact form:  $\omega = \pi^*i^*\omega + d\tau$.
	\begin{align*}
		\int_M \omega\wedge j_*U 
		= \int_T \omega \wedge U 
		&= \int_T (\pi^*i^*\omega + d\tau) \wedge U \\
		&= \int_T (\pi^*i^*\omega) \wedge U \tag{by Stokes' theorem}\\
		&= \int_S i^*\omega  \wedge \pi_*U 
		\tag{by Proposition \ref{prop:projection_formula}}\\
		&= \int_S i^*\omega \tag{since $\pi_*U = 1$} 
	\end{align*}
\end{proof}
\vspace{-1.5ex}
In particular, this means that the Poincar\'e dual of the image of the zero 
section $M_0\subset E$ of a vector bundle $E\to M$ is the equal to the Thom 
class of  $E$. This is because the exact sequence 
\[
	\begin{tikzcd}[column sep = 1.6em]
		0 \arrow[r] & TM_0 \arrow[r] & TE|_{M_0} \arrow[r] 
						& E \arrow[r] & 0
	\end{tikzcd}
	\]
shows that the normal bundle of the zero section is  $E$ itself.  

\begin{remark}[Localisation principle]
	Another observation is that the support of the Poincar\'e dual of $S$ can be
	shrunk into any arbitrarily small tubular neighbourhood of $S$, 
	by simply pulling back the Thom class of the
	normal bundle into the tubular neighbourhood.
\end{remark}

To find the Poincar\'e dual of the zero set of an arbitrary section, or more
generally the inverse image of a submanifold, we first need a result 
in intersection theory.
It will be useful at this stage to review transversality in Appendix
\ref{appendix4} before continuing. 

\section{Poincar\'e dual of submanifolds}
\begin{defn} % def 2.5 nicolescu
	Let $S$ and  $L$ be oriented submanifolds of the
	oriented manifold  $M$ such that  $\dim S + \dim L = \dim M$. Suppose 
	 $S$ and  $L$ intersect transversely. Then for each  $p\in L\cap S$, define
	 $\epsilon(S,L,p) \in \{\pm 1\}$ via
	 \[
		 \operatorname{or}(T_pS) \wedge \operatorname{or}(T_pL)
		 = \epsilon(S,L,p) \operatorname{or}(T_p M)
	 \] 
	 If $S\cap L$ is finite, we can define the
	 \underline{intersection number} of  $L$ and  $S$ to be the integer
	  \[
		  [S]\cdot [L] = \sum_{p\in S\cap L} \epsilon(S,L,p)
	 \] 
\end{defn}
\begin{remark}
	One way to guarantee that $S\cap L$ is finite is when $S\cap L$ is compact. 
	Theorem \ref{thm:inverse_submanifold} applied to
the inclusion $i: S \to M$ and  $L$ tells us that  $S\cap L$ is a zero
dimensional submanifold. If $S\cap L$ is not finite, there is a sequence  
$(x_n)_{n\in \mathbb{N}} \subset S\cap L$ such that $x_n \to x \in S\cap L$. So 
there is no neighbourhood of $x$ in $M$ whose restriction to $S\cap L$ is a
point, contradicting that $S\cap L$ is a zero dimensional submanifold. 
\end{remark}

\begin{thm} \label{thm:intersection_poincare} % nicolescu theorem 2.6 p7
	Suppose $S$ and  $L$ are oriented submanifolds of the 
	oriented manifold  $M$, and $S$ or $L$ is compact.
	Assume  $S \pitchfork L$ and  $\dim L + \dim S = \dim M$.
	Then 
	\[ 
		[S] \cdot [L] = \int_M \eta_S \wedge \eta_L = \int_L \eta_S
	\] 
\end{thm}
\begin{proof}
	Set $s=\dim S$, $l=\dim L$ and $n=s+l$. For any $p\in S\cap L$, there are
	submersions $f : U_p \to \mathbb{R}^l$ and $g:U_p\to \mathbb{R}^s$ from a
	neighbourhood of $p$ such that  $f^{-1}(0) = S \cap U_p$ and $g^{-1}(0) =
	L\cap U$. Then the joint function $\psi=(g,f) : U_p \to \mathbb{R}^n$ satisfies 
	\[
		S\cap U_p = \{x_{s+1}=\cdots=x_{s+l}=0\}, \quad
		L\cap U_p = \{x_{1}=\cdots=x_{s}=0\}, 
	\] 
	Since $\dim M = l + s$ and $S\pitchfork L$, the tangent space is a direct sum 
	$T_pM = T_pS \oplus T_pL$. Since $f$ and  $g$ are also submersions,
	it follows that $D\psi|_p : T_p S \oplus T_p L \to \mathbb{R}^{s+l}$ is 
	bijective. Hence, by the
	inverse function theorem, we can assume $\psi$ is a diffeomorphism to its
	image. 

	\begin{comment}
	By permuting the components if necessary, assume that orientation of $S\cap
	U_p$ is described in coordinates as $dx_1\wedge\cdots\wedge dx_s$, while the 
	orientation of $L\cap S$ is $dx_{s+1}\wedge \cdots\wedge dx_{s+l}$.
	\end{comment}
 	\begin{figure}[htb]
		\hfill
 		\begin{minipage}[c]{0.6\textwidth}
			\includegraphics[trim={35mm 0 35mm 0},clip,width=\textwidth]{figs/tubular_intersection.pdf}
 		\end{minipage} 
 		\begin{minipage}[c]{0.38\textwidth}
 	        \caption{The fibers of the tubular neighbourhood of $S$ at points of
			transverse intersection coincides with  $L$}
 	        \label{fig:tubular_intersection}
 		\end{minipage} 
 	\end{figure}	
	Our key idea is that we can choose neighbourhoods $V_p \subset U_p$
	and a Riemannian metric $g$ on  $M$ such that on $V_p$ the metric is the
	Euclidean norm $(dx_1)^2 + \ldots+(dx_n)^2$. This means that we can choose a
	tubular neighbourhood $T_g \subset M$ of $S$  such that the fibers coincide
	with the submanifold $L$, i.e. (see Figure \ref{fig:tubular_intersection})
	\[
	T_g(p) = (L \cap V_p) \cap T_g
	\] 
	where we must choose $T_g$ small enough so that the fiber of $p$ is 
	contained in $V_p$. Note we can assume $V_p$ are disjoint, so that $L\cap
	T_g = \bigcup_{p\in S\cap L} T_g(p)$. Let $\eta_S \in \Omega^l(M)$ be a
	representative of the Poincar\'e dual of $S$ with support on  $T_g$. 
	
	The fiber $T_g(p)$ is equipped with the co-orientation of  $S$ in $M$
	(equation (\ref{eq:normal_orientation})). 
	Denote $L_p := T_g(p)$ to be the same fiber but equipped with the
	orientation induced by  $L$. Then we have the relation 
	\[
		\operatorname{or}(L_p) = \epsilon(S,L,p) \operatorname{or}(T_g(p))
	\] 
	Now observe that 
	\[
	\int_M \eta_S \wedge \eta_L 
	= \int_L \eta_S
	= \sum_{p\in S\cap L} \int_{L_p} \eta_S
	= \sum_{p\in S\cap L} \epsilon(S,L,p)\int_{T_g(p)} \eta_S
	= \sum_{p\in S\cap L} \epsilon(S,L,p)
	\] 
	where $\int_{T_g(p)}\eta_S = 1$ by Theorem \ref{thm:poincare_thom}.
\end{proof}

\begin{comment} % Poincare dual for smooth cycles
We can generalise the notion of Poincar\'e dual of submanifolds to smooth
cycles.  The pair  $(N,f)$ determines a linear
functional 
\[
H^n(M) \to \mathbb{R}, \qquad \omega \mapsto \int_N f^*\omega
\] 
The integral is well defined because $N$ is compact. We call this linear
functional a smooth cycle, and denote it $f_*[N]$. As before, it follows by
Poincar\'e duality that integration over $N$ corresponds to a unique
$[\eta_{f_*[N]}] \in H^{m-n}(M)$ such that 
\[
	\int_N f^*\omega = \int_M \omega \wedge \eta_{f_*[N]} \quad\; 
	\fall \omega \in H^n(M)
\] 
\end{comment}

We can generalise Theorem \ref{thm:poincare_thom} 
by finding the Poincar\'e dual of the zero set of an arbitrary section, or more
generally the inverse image of a submanifold.

Suppose $M$ and $N$ are oriented manifolds of dimensions $m$
and $n$, $f:N\to M$ is a smooth map, and
$S \subset M$ is a submanifold of dimension $k$ that is transverse to $f$. 
By Theorem \ref{thm:inverse_submanifold}, $f^{-1}S\subset N$ is a submanifold 
of dimension $n-m+k$, assuming of course that $n+k \geq m$. 
The following lemma gives an orientation on the normal
bundle $N_{f^{-1}(S)}$, which in turn gives a natural orientation on $f^{-1}(S)$
via equation (\ref{eq:normal_orientation}). 

\begin{lem} \label{lem:normal_isomorphism}
	There is a bundle isomorphism $Df : N_{f^{-1}(S)} \to f^*N_{S}$.
\end{lem}
\begin{proof}
	First observe that $f \pitchfork S$ if and only if  $Df|_x : T_x N \to
	T_{f(x)}M$ maps surjectively to the quotient vector space $T_{f(x)}M /T_{f(x)} S =
	(N_S)_{f(x)}$ for all $x\in f^{-1}(S)$. Note that $Df|_x$ maps
	$T_xf^{-1}(S)$ into $T_{f(x)}S$ (this is true in general), and hence
	the kernel of the map $Df|_x : T_x N \to (N_S)_{f(x)}$ is $T_xf^{-1}(S)$. 
	The kernel cannot be any larger due to dimension constraints from
	surjectivity. Therefore, by the first isomorphism theorem 
	$Df|_x : (N_{f^{-1}(S)})_x \to (N_S)_{f(x)}$ is a linear
	isomorphism. Consequently, 
	$Df : N_{f^{-1}(S)} \to f^*N_{S}$ is a bundle isomorphism. 
\end{proof}


\begin{thm} \label{thm:localisation}% nicolescu thm 3.4 % Bott Tu pg 69 
	Let $f : N \to M$ and  $S \subset M$ as above such that $f^{-1}(S)$ is
	compact. Then $f^*\eta_S = \eta_{f^{-1}(S)}$,  i.e. 
	\[
	\int_{N} \omega \wedge f^*\eta_S =\int_{f^{-1}(S)} i^*\omega  
	\qquad
	\text{for all }\omega\in H^{n-m+k}(N)
	\]
	where $i: f^{-1}(S) \to N$ is the inclusion map.
\end{thm}
\begin{proof} 
	Fix Riemannian metrics $g$ on $N$ and $h$ on $M$.
	Let $T^\epsilon_g \subset N$ and $T^\delta_h \subset M$ be tubular
	neighbourhoods of  $f^{-1}(S)$ in $N$ and  $S$ in $M$, of the form in
	equation (\ref{eq:tubular_radius}). Since $f^{-1}(S)$ is compact, assume  
	$\epsilon$ is constant. Denote by $T^\epsilon_g(x)$ the fiber over 
	$x\in f^{-1}(S)$,
	defined in equation (\ref{eq:tubular_fiber}). 
	We claim that there exists $\epsilon > 0$ such that the restriction of  $f$
	to  $T^\epsilon_g(x)$ is diffeomorphism to its image for all  
	$x\in f^{-1}(S)$. % lemma 3.5
	\begin{subproof}
		By Lemma \ref{lem:normal_isomorphism}, 
		$Df|_x : (N_{f^{-1}(S)})_x \to (N_S)_{f(x)}$ is an isomorphism.
		By identifying the fiber $(N_{f^{-1}(S)})_x$ as the tangent space to 
		the fiber of the tubular neighbourhood, we can write  
		$Df|_x : T_xT_g^\epsilon(x) \to T_{f(x)}T_h^\delta(f(x))$. 
		This can be interpreted as the differential of the restriction to
		$T^\epsilon_g(x)$. 
		So by the inverse function theorem, for each $x\in f^{-1}(S)$,
		there exists a neighbourhood $U_x \subset N$ of $x$ and  $\epsilon(x)>0$
		such that the restriction to  $T^{\epsilon(x)}_g(x)$ is a diffeomorphism
		to its image. Since $f^{-1}(S)$ is compact, there is a finite subcover 
		$U_{x_1},\ldots,U_{x_r}$ and define 
		$
			\epsilon = \min\{\epsilon(x_1),\ldots,\epsilon(x_r)\}
		$. 
	\end{subproof}
	\begin{figure}[htb]
		\centering
		\includegraphics[width=\textwidth]{figs/normal_fiber_map.pdf}
		\caption{The diffeomorphism $f:T^\epsilon_g(x) \to L_x^\epsilon$ mapping
		the normal fiber}
		\label{fig:normal_fiber_map}
	\end{figure}
	For such an $\epsilon$,  denote the image of the embedding  $L_x^\epsilon :=
	f(T^\epsilon_g(x))$. 

	\begin{comment} % TODO needed?
	Next, we claim that there exists 
	$\rho > 0$ such that for any  $x\in f^{-1}(S)$ the intersection 
	$L_x^\epsilon \cap T_h^\rho$ is closed in $T^\rho_h$.
	Note that $T^\rho_h$ is not necessarily a tubular neighbourhood of $S$.
	\begin{subproof}
		Denote the boundary of the tubular neighbourhood $\Sigma_\epsilon = \{z\in
		N \mid \operatorname{dist}_g(z,f^{-1}(S)) = \epsilon\}$. 
		Then $f(\Sigma_\epsilon)$ is a compact image of a compact set, and is
		disjoint from $S$.  Thus, 
		$\operatorname{dist}_h(f(\Sigma_\epsilon),S) > 0$.

		We will show $\rho = \frac{1}{2}\operatorname{dist}_h(f(\Sigma_\epsilon),S)$
		is the desired value. Suppose $(y_n)_{n\in \mathbb{N}} \subset 
		L_x^\epsilon \cap T_h^\rho$ is a sequence that converges to $y\in T^\rho_h$. 
		We need to show $y\in L_x^\epsilon$. We can choose points 
		 $p_n \in T^\epsilon_g(x)$ such that $f(p_n) = x_n$. Since  $f^{-1}(S)$ is
		 compact, there is a subsequence which converges to a point $p$ such
		 that $d_g(p,x) \leq \epsilon$. Since $y$ is the unique limit, we have 
		 $f(p) = y$. And since $\operatorname{dist}_h(y_n,S) < \rho$,   
		 \[
			 \operatorname{dist}_h(y,S) \leq \rho < \dist_h(f(\Sigma_\epsilon),S)
		 \] 
		Then $f(p) \notin f(\Sigma_\epsilon)$ implies $p\notin \Sigma_\epsilon$,
		i.e. $d_g(p,x) < \epsilon$ is a strict inequality. Thus  $p\in
		T^\epsilon_g(x)$ so $y\in L_x^\epsilon$.
	\end{subproof}
	Replace $\delta$ with a smooth function bounded above by $\min(\delta(x),\rho)$ 
	so that the above holds, and $T_h^\delta$ is a tubular neighbourhood.
	\end{comment}

	Choose a form $\eta_S$ representing the Poincar\'e dual with support in
	$T^\delta_h$. The integral of $f^*\eta_S$ along the fiber  $T^\epsilon_g(x)$
	is 
	 \[
	\int_{T^\epsilon_g(x)} f^*\eta_S 
	= \int_{L_x^\epsilon} \eta_S = [S] \cdot [L_x^\epsilon] = 1
	\] 
	The first equality applies via the diffeomorphism
	$f:T^\epsilon_g(x) \to L_x^\delta$. 
	The second equality follows from Theorem \ref{thm:intersection_poincare},
	since $L_x^\epsilon$ is compact.
	
	Therefore, $f^*\eta_S$ is the pullback of Thom class of the normal bundle over
	$f^{-1}(S)$, so it is equal to the Poincar\'e dual of $f^{-1}(S)$ by Theorem 
	\ref{thm:poincare_thom}.
\end{proof}
\begin{comment} % old proof using Bott Tu
\begin{proof}
	Bott Tu version:
	Also $f^{-1}T \subset N$ is a neighbourhood of 
	dimension $n$ (why?). 
	We can choose $T$ to be a sufficiently small tubular
	neighbourhood of  $S$ such that  
	$f^{-1}T$ is contained within a tubular
	neighbourhood of $f^{-1}S$ in $N$, and hence diffeomorphic to 



	The following diagram commutes (why?)
	% https://q.uiver.app/?q=WzAsNixbMCwwLCJIXiooUykiXSxbMSwwLCJIXnsqK2t9KFQpIl0sWzIsMCwiSF4qKE0pIl0sWzIsMSwiSF4qKE0nKSJdLFsxLDEsIkheeyora30oZl57LTF9VCkiXSxbMCwxLCJIXiooZl57LTF9UykiXSxbMCw1LCJmXioiXSxbMSw0LCJmXioiXSxbMiwzLCJmXioiXSxbMCwxLCJcXHdlZGdlIFxcUGhpKFQpIl0sWzEsMiwial8qIl0sWzQsMywial8qIl0sWzUsNCwiXFx3ZWRnZSBcXFBoaShmXnstMX1UKSJdXQ==
	\[\begin{tikzcd}[row sep=2.85em, column sep=4.45em]
			{H^*(S)} & {H^{*+k}_{cv}(T)} & {H^*(M)} \\
				{H^*(f^{-1}S)} & {H^{*+k}_{cv}(f^{-1}T)} & {H^*(M')}
					\arrow["{f^*}", from=1-1, to=2-1]
						\arrow["{f^*}", from=1-2, to=2-2]
							\arrow["{f^*}", from=1-3, to=2-3]
								\arrow["{\wedge \Phi(T)}", from=1-1, to=1-2]
									\arrow["{j_*}", from=1-2, to=1-3]
										\arrow["{j_*}", from=2-2, to=2-3]
											\arrow["{\wedge \Phi(f^{-1}T)}",
											from=2-1, to=2-2]
	\end{tikzcd}\]
	Starting with $1\in H^0(S)$, the maps on the top row gives 
	$\eta_S \in H^{n-k}(M) \mapsto f^*\eta_S \in H^{n-k}(M')$ 
	On the other hand, the maps on the bottom row gives
	$\eta_{f^{-1}S} \mapsto \eta_{f^{-1}S} \in H^{n-k}$
\end{proof}
\end{comment}

\begin{cor} \label{cor:vb_localisation} % prop 12.8 Bott Tu
	Let $E\xrightarrow{\pi} M$ be an oriented vector bundle of rank $k$ over 
	an oriented manifold of dimension $n$, 
	and $s: M\to E$ be a section which is transverse to the zero section. If 
	$n\geq k$ and $s^{-1}(0)$ is compact, then $s^*\Phi(E)=\eta_{s^{-1}(0)}$, i.e. 
	\[
	\int_M \omega \wedge s^* \Phi(E) =\int_{s^{-1}(0)} i^*\omega  
	\qquad
	\text{for all }\omega\in H^{n-k}(M)
	\] 
\end{cor}
\begin{proof}
	Denote $M_0 \subset E$ as the image of the zero section.
	By direct application of the previous theorem, we have $s^*\eta_{M_0} =
	\eta_{s^{-1}(0)}$. But we know from Theorem \ref{thm:poincare_thom} that 
	$\eta_{M_0} = \Phi(E)$, and the result follows.
\end{proof}

Note that $s^*\Phi(E) = \chi(E)$ is the Euler class of the vector bundle for any
choice of section, as we have proved. This is remarkable, because the left side
is independent of the section, but can be computed as the integral over the zero
set of any transverse section.

The result above can be viewed as the generalisation of the Poincar\'e-Hopf
theorem that we alluded to, when we take the vector bundle to be the
tangent space and $\omega=1$, in which case $s^{-1}(0)$ is zero dimensional. 
The integral computes the intersection number between the images of the zero
section and $s$, which gives us a new perspective on the Euler number. 

% discussion from cordes 11.10.3, physical p15
There is an extension of the previous result for a section $s$ which is not
transverse to the zero section. Set $s= \dim s^{-1}(0)$. Choose a connection on
$E$, which defines a linear map $\nabla_x s : T_xM \to E_x$ for each  $x\in M$.
Recall that when $s \pitchfork M_0$, 
there is a surjection $Ds : T_x M \to T_{s(x)} E / T_{s(x)} M_0$ from which 
we deduce that $n = k + s$. 
But now suppose instead that  $s > n - k$, so
the section is not transverse to $M_0$, but such that the fibers of 
$\coker (\nabla s)$ defined by 
\[
0 \to \Im(\nabla_x s) \to E_x \to \coker (\nabla_x s) \to 0
\] 
have constant rank so that $\coker \nabla s$ is a vector bundle over  $s^{-1}(0)$.
The rank of $\coker (\nabla s)$ is $r := k-(n-s)$.
Notice that $\nabla_x s = Ds|_x$ for all  $x\in s^{-1}(0)$. 
Given orientations of  $M$ and  $E$, $\coker(\nabla s)$ is canonically oriented.


\begin{thm}[Generalised localisation theorem]
	Let $E \xrightarrow{\pi} M$ be an oriented vector bundle over an oriented
	manifold, and  $s : M \to E$ be a section with the properties 
	above such that $s^{-1}(0)$ is compact. Then for all $\omega\in H^{s-r}(M)$
	\[
		\int_M \omega \wedge \chi(E\to M)
		= \int_{s^{-1}(0)} i^*\omega \wedge \chi(\coker(\nabla s)\to s^{-1}(0))
\]
\end{thm}
\begin{proof}[\textbf{\textit{Proof}} (sketch)] % witten N matrix model p229
	In the following, $E$ is understood to be a vector bundle over  $M$, while
	 $\coker(\nabla s)$ is a vector bundle over  $s^{-1}(0)$. 
	To uncover the relation between $\chi(E)$ and  
	$\chi(\coker(\nabla s))$, pick
	a generic section $u : s^{-1}(0) \to \coker(\nabla s)$. Then by Corollary
	\ref{cor:vb_localisation}, the zero set $u^{-1}(0) \subset s^{-1}(0)$ is 
	Poincar\'e dual to  $\chi(\coker(\nabla s))$, i.e. 
	\[
	\int_{s^{-1}(0)} \omega \wedge \chi(\coker(\nabla s)) 
	= \int_{u^{-1}(0)} \omega, \qquad
	\text{for all } \omega\in H^{s-r}(s^{-1}(0))
	\] 
	Since $u^{-1}(0) \subset s^{-1}(0) \subset M$, $u^{-1}(0)$ also determines a
	cohomology class of $M$. We claim that the Poincar\'e dual of $u^{-1}(0)$ in
	$M$ is $\chi(E)$, i.e. 
	\[
	\int_{M} \omega \wedge \chi(E) 
	= \int_{u^{-1}(0)} \omega, \qquad
	\text{for all } \omega\in H^{s-r}(M)
	\] 
	from which the theorem follows directly.
	To prove this, lift and extend $u$ to $\overline{u} : M \to E$
	such that $\overline{u}$ vanishes outside some tubular neighbourhood of
	$s^{-1}(0)$. The lift can be done by identifying $\coker(\nabla s) 
	= (\Im \nabla s)^{\perp}$ via a metric on $E$. 

	The main idea is that the section $s+ \epsilon \overline{u}$ is now
	transverse to $M_0 \subset E$ for all $\epsilon > 0$ at points $x\in
	u^{-1}(0)$. This is because 
	$\nabla_x s : T_x M / T_x s^{-1}(0) \to \Im(\nabla_x s)$ is surjective, 
	while  $\nabla_x \overline{u} : T_x s^{-1}(0) \to \Im(\nabla_x s)^\perp$ 
	is surjective. Hence, $\nabla_x(s+\epsilon \overline{u}) : T_xM \to E_x$ is
	surjective. Moreover, we extend $u$ to the neighbourhood of  $s^{-1}(0)$ such
	that $s + \epsilon\overline{u}$ is transverse to the zero section for all
	$\epsilon > 0$. 
	\begin{figure}[htb]
		\begin{minipage}[c]{0.46\textwidth}
			\includegraphics[trim={43mm 5mm 41mm 2.8mm},clip,width=\textwidth]{figs/witten_coker.pdf}
		\end{minipage} 
		\begin{minipage}[c]{0.52\textwidth}
			% https://q.uiver.app/#q=WzAsNSxbMiwwLCJFIl0sWzIsMSwiTSJdLFsxLDEsInNeey0xfSgwKSJdLFswLDEsInVeey0xfSgwKSJdLFsxLDAsIlxcb3BlcmF0b3JuYW1le2Nva2VyfShcXG5hYmxhIHMpIl0sWzMsMiwiIiwwLHsic3R5bGUiOnsidGFpbCI6eyJuYW1lIjoiaG9vayIsInNpZGUiOiJ0b3AifX19XSxbMiwxLCIiLDAseyJzdHlsZSI6eyJ0YWlsIjp7Im5hbWUiOiJob29rIiwic2lkZSI6InRvcCJ9fX1dLFswLDEsIiIsMix7Im9mZnNldCI6MX1dLFsyLDQsInUiLDIseyJvZmZzZXQiOjF9XSxbNCwyLCIiLDAseyJvZmZzZXQiOjF9XSxbMSwwLCJzIiwyLHsib2Zmc2V0IjoxfV1d
				\[\begin{tikzcd}
						& {\operatorname{coker}(\nabla s)} & E \\
							{u^{-1}(0)} & {s^{-1}(0)} & M
								\arrow[hook, from=2-1, to=2-2]
									\arrow[hook, from=2-2, to=2-3]
										\arrow[shift right=1, from=1-3, to=2-3]
											\arrow["u"', shift right=1,
											from=2-2, to=1-2]
												\arrow[shift right=1, from=1-2,
												to=2-2]
													\arrow["s"', shift right=1,
													from=2-3, to=1-3]
						\end{tikzcd}\]
			
		\end{minipage} 
	    \hfill
	\end{figure}
	Although $s+ \epsilon \overline{u}$ may have zeros outside of
	$u^{-1}(0)$ where the two functions cancel out, these values must
	become non-zero as $\epsilon\to 0$. This is because outside  $s^{-1}(0)$,
	$\epsilon\overline{u} \to 0$ while $s$ is non-zero, and on 
	$s^{-1}(0) \setminus u^{-1}(0)$,  $s=0$ while  $\epsilon \overline{u}$ is
	non-zero. Hence as $\epsilon \to 0$, the zero set approaches a
	neighbourhood of  $u^{-1}(0)$. 

	Therefore, we have 
	\[
	\int_{M} \omega \wedge \chi(E) 
	= \int_{(s+\epsilon \overline{u})^{-1}(0)} \omega
	\xrightarrow{\epsilon \to 0} 
	\int_{u^{-1}(0)} \omega, \qquad
	\text{for all } \omega\in H^{s-r}(M)
	\]
	The limit is defined because $u^{-1}(0)$ is compact. 
	Also note $(s+\epsilon \overline{u})^{-1}(0)$ has dimension $n-k$, while 
	$u^{-1}(0)$ has dimension $s-r$, which are equal because
	$r = k-(n-s)$. 
	\begin{comment}
	Want to say that we can find a neighbourhood of $s^{-1}(0)$ such that the
	zero set is only $u^{-1}(0)$ and there is no path of zeros into $u^{-1}(0)$.
	
	$\nabla_x s(v)$ and $\nabla_x \overline{u}(v)$ are linearly independent in 
	$E_x$ for all  $v\in T_x M$. 
	The problem is that $\overline{u}$ and $s$ may have derivative zero in some
	direction, and all the subsequent derivatives may match. 

		
	An important observation is that it is not possible for there to be a path
	$\gamma(t) : (0,1) \to M \setminus u^{-1}(0)$ such that $\gamma$ approaches
	a point in $u^{-1}(0)$ and $(s+\epsilon \overline{u})(\gamma(t)) = 0$. 
	Otherwise, this would mean $Ds$

	For any $x\in u^{-1}(0)$, the derivatives of $\overline{u}$ in a certain
	direction $v \in T_x M$ are always contained in $(\Im(\nabla_x s))^\perp$

	\end{comment}
\end{proof}

Note that if $s$ is a generic section then  $\coker (\nabla s) = \{0\}$, which
reduces to Corollary \ref{cor:vb_localisation}. This generalisation is due to
\citet[Sec 3.3]{witten_coker}, which is the only paper I found where any 
justification of this is given, and the proof is my best attempt to make it
precise.  
 

\vspace{5mm}
\hrule 
\vspace{5mm}

\textbf{Bibliographical notes}
{\small
\begin{itemize}
	\item The intersection theory in this chapter is largely owed to
	\citet{nicolaescu_intersection}, whose writing is very detailed and
	precise. The notes are intended to provide a complete proof of Theorem
	\ref{thm:localisation}, which is claimed at the end of page 69 in 
	\citet{bott_tu}.
\end{itemize}
}
