\chapter{Supersymmetry} 
\label{appendix3}
\section{Spin representations}
Given a Lie group of symmetries $G$ on fields, physicists are often interested in the 
representations (reps) of its universal covering group. The reason is 
Wigner's theorem shows that symmetry transformations on a projective Hilbert space are
represented by a unitary or antiunitary operator on the Hilbert space. The
former is of more physical interest, so the Hilbert
space of the QFT must carry a projective unitary representation of $G$. 
This analysis is made easier by
Bargmann's theorem, which says that if $G$ is a connected 
Lie group and its Lie algebra cohomology group
$H^2(\mathfrak{g},\mathbb{R})$ is trivial, then every projective unitary
representation can be lifted to a unitary representation of the universal cover
of $G$.\cite[Theorem 4.8]{cft} This applies to any semisimple Lie
group, and in particular for $\SO^+(1,3)$, the restricted
Lorentz group. 

The spin group $\Spin(n)$ or $\Spin(p,q)$ is defined such that it is the double cover of
$\SO(n)$ or $\SO^+(p,q)$ respectively. 
The latter denotes the connected component of the identity of $\SO(p,q)$. 
That is, there is a group homomorphism $\Spin(p,q) \to \SO^+(p,q)$ whose kernel
has two elements $\{-1,1\}$. 
As a result, they also share the same Lie algebra.

For $n\geq 3$,
$\Spin(n)$ is simply connected, and thus is the unique universal covering 
of $\SO(n)$. However, in general $\Spin(p,q)$ is not simply connected. 
But for Minkowski space, we have the isomorphism $\Spin(3,1) \simeq
\SL(2,\mathbb{C})$, which is simply connected. 
Therefore, every representation of its Lie algebra can be integrated to a group
representation, so we only need to consider representations of
$\mathfrak{sl}(2,\mathbb{C})$.\cite[Theorem 5.6]{hall} 
This is the relativistic analogue of $\SU(2)$ being the double cover of  $\SO(3)$,
and explains why representations of $\SL(2,\mathbb{C})$ is central to the study
of relativistic spin, and will turn out to be also for supersymmetry. 

\begin{defn}
	\underline{Spin representations} are the simplest representations of 
	$\Spin(p,q)$ that do not come from representations of  $\SO(p,q)$. 
	More precisely, it is a representation $\rho : \Spin(p,q) \to \GL(V)$ such 
	that $-1$ is not in the kernel of $\rho$.

\underline{Spinors} are elements of a spin representation $V$.
\end{defn}

\begin{prop}
	\begin{enumerate}[(1)]
	    \item 
	$\mathfrak{sl}(2,\mathbb{C}) \simeq \mathfrak{so}(1,3)
	\simeq\mathfrak{su}(2)_{\mathbb{C}}$
		\item \phantom{}\vspace{-1.0cm}
	\begin{align*}
		\mathfrak{sl}(2,\mathbb{C})_{\mathbb{C}}
		&\simeq \mathfrak{sl}(2,\mathbb{C})\oplus i\mathfrak{sl}(2,\mathbb{C})\\
		&\simeq \mathfrak{sl}(2,\mathbb{C})\oplus \mathfrak{sl}(2,\mathbb{C}) \\
		&\simeq \mathfrak{su}(2)_{\mathbb{C}} \oplus \mathfrak{su}(2)_{\mathbb{C}}
	\end{align*}
	\end{enumerate}
\end{prop}
\begin{proof}
	\begin{enumerate}[(1)]
	    \item 
	The first isomorphism comes from the double covering $\SL(2,\mathbb{C}) \to
	\SO(1,3)$ being a local diffeomorphism.

	Any $X\in \mathfrak{sl}(2,\mathbb{C})$ can be written $X = X_1 + iX_2$ where
	$X_1= (X-X^*) /2$ and $X_2= (X+X^*) /(2i)$, where $X^*$ is the conjugate
	transpose. Note that both $X_1$ and $X_2$ are traceless and skew-Hermitian
	so are in $\mathfrak{su}(2)$.
		\item 
	We view $\mathfrak{sl}(2,\mathbb{C})$ as a real Lie algebra, so its
	complexification is a complex Lie algebra with two copies. 
	The last line follows from part (1).
	\end{enumerate}
\end{proof}
 
Note that a complex rep refers to a rep on a complex
vector space, but the Lie algebra homomorphism can still be either
$\mathbb{R}$-linear or $\mathbb{C}$-linear. 

\begin{prop}
	Let $\mathfrak{g}$ be a real Lie algebra. Then
	$\mathbb{R}$-linear reps of $\mathfrak{g}$ are in one-to-one
	correspondence with $\mathbb{C}$-linear reps of  $\mathfrak{g}_{\mathbb{C}}$
	on a complex vector space. 
\end{prop}
\begin{proof}
	Given an $\mathbb{R}$-linear rep $\pi : \mathfrak{g} \to \mathfrak{gl}(V)$,
	where  $V$ is a complex vector space, this extends to a unique
	$\mathbb{C}$-linear rep of $\mathfrak{g}_{\mathbb{C}}$ given by 
	$\pi(X+iY) = \pi(X) + i\pi(Y)$. It is easy to check that this map is  a
	$\mathbb{C}$-linear Lie algebra homomorphism. Uniqueness comes from
	$\mathbb{C}$-linearity.

	Let $\pi : \mathfrak{g}_{\mathbb{C}} \to\mathfrak{gl}(V)$ be a 
	$\mathbb{C}$-linear rep , and the inclusion $i : \mathfrak{g} \to
	\mathfrak{g}\otimes_{\mathbb{R}}\mathbb{C}$ defined by $i(X) = X\otimes 1$. We can restrict
	the rep to an $\mathbb{R}$-linear rep on $\mathfrak{g}$ by defining
	$\widetilde{\pi}(X) = \pi(i(X))$, which is a rep because it is the composition of
	two Lie algebra homomorphisms. 

	Finally, after checking that the two maps above are inverses, we conclude
	that there is a one-to-one correspondence. 
\end{proof}

Therefore, the proposition shows that $\mathbb{R}$-linear reps of
$\mathfrak{sl}(2,\mathbb{C})$ are obtained from $\mathbb{C}$-linear reps of
$\mathfrak{sl}(2,\mathbb{C})\oplus \mathfrak{sl}(2,\mathbb{C})$.
We know for each integer $m\geq 0$, there is a unique irreducible complex
rep of  $\mathfrak{sl}(2,\mathbb{C})$ of dimension $m+1$.\cite{hall} Hence,
$\mathbb{R}$-linear irreducible reps  of $\mathfrak{sl}(2,\mathbb{C})$ (and
$\SL(2,\mathbb{C})$)
are indexed by two integers $(j_1,j_2)$ where $j_1-1$ and $j_2-2$ are the
dimensions. Sometimes this pair is divided by 2, and each half integer
represents the spin. 

% A.2 labastida
The group $\SL(2,\mathbb{C})$ has the standard representation on a two dimensional
complex vector space $S$. Similarly, there is the complex conjugate
representation $\overline{S}$, dual representation $\widetilde{S}$, and dual
complex conjugate representation $\widetilde{\overline{S}}$. 
The action of $M\in \SL(2,\mathbb{C})$ in these representations are
multiplication by its complex conjugate $\overline{M}$, inverse transpose
$(M^\intercal)^{-1}$ and inverse adjoint $(M^{\dagger})^{-1}$ respectively.
We denote spinors of these representations by 
\[
\psi_\alpha \in S, \qquad \psi_{\dot{\alpha}}\in \overline{S}, \qquad
\psi^\alpha \in \widetilde{S}, \qquad \psi^{\dot{\alpha}}\in \widetilde{\overline{S}}
\] 
and operators are denoted with dotted or undotted components consistent with the
above, e.g. $M_{\alpha}{}^\beta \in \mathfrak{gl}(S),
M_{\dot{\alpha}}{}^{\dot{\beta}} \in \mathfrak{gl}(\overline{S})$. 
Contractions with the $\epsilon_{\alpha\beta}$ and 
$\sigma_{\dot{\alpha}\dot{\beta}}$ tensor raise and lower spinor indices.

The standard rep  $S$ corresponds to the index $(1,0)$ while $\overline{S}$ to $(0,1)$. 


\section{Supersymmetry algebra}
% West ch2
In the 1960s, with the growing awareness of the significance of internal
symmetries such as $\SU(2)$ in electroweak theory and larger groups,
physicists attempted to find a Lie group which would combine the spacetime
Poincare group with an internal symmetry group. After much effort, Coleman and
Mandula proved under general assumptions that any Lie group which contains the
Poincare group $P$ and an internal symmetry group $G$ which is consistent with
QFT, must be a direct product of $P$ and  $G$, i.e. their generators commute.

The supersymmetry algebra avoids the restrictions of the Coleman-Mandula theorem
by relaxing the notion of a Lie algebra to include algebraic systems whose
defining relations involve anticommutators as well as commutators, called
a Lie superalgebra. In 

\begin{defn}
	A \underline{Lie superalgebra} is a nonassociative 
	$\mathbb{Z}_2$-graded algebra over a commutative ring (usually $\mathbb{R}$
	or $\mathbb{C}$) $\mathfrak{g} = \mathfrak{g}_0 \oplus \mathfrak{g}_1$ with
	an operation $[\cdot,\cdot]:\mathfrak{g}\times \mathfrak{g} \to
	\mathfrak{g}$ satisfying 
	\begin{itemize}
		\item Super skew-symmetry: $[x,y] = -(-1)^{\abs{x}\abs{y}}[y,x]$
		\item Super Jacobi identity: 
		\[
			(-1)^{\abs{x}\abs{z}}[x,[y,z]]+(-1)^{\abs{y}\abs{x}}[y,[z,x]]
			+(-1)^{\abs{z}\abs{y}}[z,[x,y]]=0
		\]
	\end{itemize}
\end{defn}
Note that in particular, the 

\begin{defn}
	supermanifold
\end{defn}

Functions on superspace = differential forms

integration on supermanifolds using (Haar?) measure
