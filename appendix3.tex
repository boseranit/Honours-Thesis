\chapter{Supersymmetry} 
\label{appendix3}

The spin group $\Spin(n)$ or $\Spin(p,q)$ is defined such that it is the double cover of
$\SO(n)$ or $\SO^+(p,q)$ respectively. 
The latter denotes the connected component of the identity of $\SO(p,q)$. 
That is, there is a smooth group homomorphism $\Spin(p,q) \to \SO^+(p,q)$ whose kernel
has two elements $\{-1,1\}$. 
As a result, they also share the same Lie algebra.

For $n\geq 3$,
$\Spin(n)$ is simply connected, and thus is the unique universal covering 
of $\SO(n)$. However, in general $\Spin(p,q)$ is not simply connected. 
But for Minkowski space, we have the isomorphism $\Spin(3,1) \simeq
\SL(2,\mathbb{C})$, which is simply connected. 
Therefore, every representation of its Lie algebra can be integrated to a group
representation, so we only need to consider representations of
$\mathfrak{sl}(2,\mathbb{C})$.\cite[Theorem 5.6]{hall} 
This is the relativistic analogue of $\SU(2)$ being the double cover of  $\SO(3)$,
and explains why representations of $\SL(2,\mathbb{C})$ is central to the study
of relativistic spin, and will turn out to be also for supersymmetry. 


\section{Motivation}
Given a Lie group $G$ of symmetries on fields, physicists are often interested in the 
representations (reps) of its universal covering group. One reason is that
Wigner's theorem shows that symmetry transformations on a projective Hilbert space 
come from a unitary or antiunitary operator on the Hilbert space. The
former is of more physical interest, so the Hilbert
space of the QFT must carry a projective unitary representation of $G$. 
This analysis is made easier by
Bargmann's theorem, which says that if $G$ has a connected universal covering
group and its Lie algebra cohomology group
$H^2(\mathfrak{g},\mathbb{R})$ is trivial, then every projective unitary
representation can be lifted to a unitary representation of the universal cover
of $G$.\cite[Theorem 4.8]{cft} 
This applies to any semisimple Lie
group, and in particular for the Poincar\'e group, which we expect to act as a
symmetry group on the Hilbert space.  

% https://math.stackexchange.com/questions/49345/unitary-representations-of-non-compact-lie-groups
% https://physics.stackexchange.com/questions/358390/on-finite-dimensional-unitary-representations-of-non-compact-lie-groups
% wiki Reps of lorentz group - finite dim reps
These representations must be infinite dimensional because a non-compact 
connected simple Lie group has no non-trivial
finite dimensional unitary reps. % TODO reference 
Note that $\SO^+(1,3)$ is non-compact, connected and simple; hence so is its universal
covering group because it shares the same Lie algebra and the statement
also applies to the universal cover of the Poincar\'e group. 
The usual approach to finding irreducible unitary reps of its covering
group is Wigner's little group method, which leads to Wigner's classification of
particles.\cite{wigner_classification} 

However, we are also interested in finite dimensional reps of the universal
cover of $\SO^+(1,3)$. The reason comes from 
the fact that we expect a finite discrete number of spin states, as the
Stern-Gerlach experiment suggests. We can allow for this by having the fields 
transform as a finite dimensional representation of $\SL(2,\mathbb{C})$.  
The relationship between this finite dim rep on fields and the infinite dim rep
on the Hilbert space is expressed in one of the Wightman axioms, known as the
transformation law of the field. 
\begin{comment}
	The axioms states that the fields are operator-valued, acting upon some
	Hilbert space $H$. 
	This axiom states that there is a projective representation
	$U:\mathbb{R}^{1,3}\rtimes\SO(1,3) \to \operatorname{PU}(H)$, and a
	spin representation $\sigma : \SL(2,\mathbb{C}) \to V$ such that 
	\[
	U(L,a)^{\dagger} \phi(x) U(L,a) = \sigma(L)\phi(L^{-1}(x-a))
	\] 
	where on the r.h.s, $\sigma(L)$ is a finite dimensional matrix acting upon
	the vector $(\phi^1, \ldots,\phi^{\dim V})$, and on the l.h.s
	the operator acts on each component of $\phi$. 
	So the finite dim rep gives the "classical field", while the infinite dim
	unitary rep tells us about the particle by Wigner's classification.
\end{comment}

\section{Representations of \texorpdfstring{$\SL(2,\mathbb{C})$}{SL(2,C)}}
\begin{defn}
	\underline{Spin representations} are the simplest representations of 
	$\Spin(p,q)$ that do not come from representations of  $\SO(p,q)$. 
	More precisely, it is a representation $\rho : \Spin(p,q) \to \GL(V)$ such 
	that $-1$ is not in the kernel of $\rho$.

	\underline{Spinors} are elements of a spin representation $V$.
\end{defn}



\begin{prop}
	\begin{enumerate}[(1)]
	    \item 
	$\mathfrak{sl}(2,\mathbb{C}) \simeq \mathfrak{so}(1,3)
	\simeq\mathfrak{su}(2)_{\mathbb{C}}$
		\item \phantom{}\vspace{-1.0cm}
	\begin{align*}
		\mathfrak{sl}(2,\mathbb{C})_{\mathbb{C}}
		&\simeq \mathfrak{sl}(2,\mathbb{C})\oplus i\mathfrak{sl}(2,\mathbb{C})\\
		&\simeq \mathfrak{sl}(2,\mathbb{C})\oplus \mathfrak{sl}(2,\mathbb{C}) \\
		&\simeq \mathfrak{su}(2)_{\mathbb{C}} \oplus \mathfrak{su}(2)_{\mathbb{C}}
	\end{align*}
	\end{enumerate}
\end{prop}
\begin{proof}
	\begin{enumerate}[(1)]
	    \item 
	The first isomorphism comes from the double covering $\SL(2,\mathbb{C}) \to
	\SO(1,3)$ being a local homeomorphism.

	Any $X\in \mathfrak{sl}(2,\mathbb{C})$ can be written $X = X_1 + iX_2$ where
	$X_1= (X-X^*) /2$ and $X_2= (X+X^*) /(2i)$, where $X^*$ is the conjugate
	transpose. Note that both $X_1$ and $X_2$ are traceless and skew-Hermitian
	so are in $\mathfrak{su}(2)$.
		\item 
	We view $\mathfrak{sl}(2,\mathbb{C})$ as a real Lie algebra, so its
	complexification is a complex Lie algebra with two copies. 
	The last line follows from part (1).
	\end{enumerate}
\end{proof}
 
Note that a complex rep refers to a rep on a complex
vector space, but the Lie algebra homomorphism can still be either
$\mathbb{R}$-linear or $\mathbb{C}$-linear. 

\begin{prop}
	Let $\mathfrak{g}$ be a real Lie algebra. Then
	$\mathbb{R}$-linear reps of $\mathfrak{g}$ are in one-to-one
	correspondence with $\mathbb{C}$-linear reps of  $\mathfrak{g}_{\mathbb{C}}$
	on a complex vector space. 
\end{prop}
\begin{proof}
	Given an $\mathbb{R}$-linear rep $\pi : \mathfrak{g} \to \mathfrak{gl}(V)$,
	where  $V$ is a complex vector space, this extends to a unique
	$\mathbb{C}$-linear rep of $\mathfrak{g}_{\mathbb{C}}$ given by 
	$\pi(X+iY) = \pi(X) + i\pi(Y)$. It is easy to check that this map is  a
	$\mathbb{C}$-linear Lie algebra homomorphism. Uniqueness comes from
	$\mathbb{C}$-linearity.

	Let $\pi : \mathfrak{g}_{\mathbb{C}} \to\mathfrak{gl}(V)$ be a 
	$\mathbb{C}$-linear rep , and the inclusion $i : \mathfrak{g} \to
	\mathfrak{g}\otimes_{\mathbb{R}}\mathbb{C}$ defined by $i(X) = X\otimes 1$. We can restrict
	the rep to an $\mathbb{R}$-linear rep on $\mathfrak{g}$ by defining
	$\widetilde{\pi}(X) = \pi(i(X))$, which is a rep because it is the composition of
	two Lie algebra homomorphisms. 

	Finally, after checking that the two maps above are inverses, we conclude
	that there is a one-to-one correspondence. 
\end{proof}

Therefore, the proposition shows that $\mathbb{R}$-linear reps of
$\mathfrak{sl}(2,\mathbb{C})$ are obtained from $\mathbb{C}$-linear reps of
$\mathfrak{sl}(2,\mathbb{C})\oplus \mathfrak{sl}(2,\mathbb{C})$.
We know for each integer $m\geq 0$, there is a unique irreducible complex
rep of $\mathfrak{sl}(2,\mathbb{C})$ of dimension $m+1$.\cite{hall} Hence,
$\mathbb{R}$-linear irreducible reps  of $\mathfrak{sl}(2,\mathbb{C})$ (and
$\SL(2,\mathbb{C})$)
are indexed by two half integers $(j_1,j_2)$ where $2j_1+1$ and $2j_2+2$ are the
dimensions.  

% A.2 labastida
The group $\SL(2,\mathbb{C})$ has the standard representation on a two dimensional
complex vector space $S$. Similarly, there is the complex conjugate
representation $\overline{S}$, dual representation $\widetilde{S}$, and dual
complex conjugate representation $\widetilde{\overline{S}}$. 
The action of $M\in \SL(2,\mathbb{C})$ in these representations are
multiplication by its complex conjugate $\overline{M}$, inverse transpose
$(M^\intercal)^{-1}$ and inverse adjoint $(M^{\dagger})^{-1}$ respectively.
We denote spinors of these representations by 
\[
\psi_\alpha \in S, \qquad \psi_{\dot{\alpha}}\in \overline{S}, \qquad
\psi^\alpha \in \widetilde{S}, \qquad \psi^{\dot{\alpha}}\in \widetilde{\overline{S}}
\] 
where $\alpha=1,2$.
Operators are denoted with dotted or undotted components consistent with the
above, e.g. $M_{\alpha}{}^\beta \in \mathfrak{gl}(S),
M_{\dot{\alpha}}{}^{\dot{\beta}} \in \mathfrak{gl}(\overline{S})$. 
Contractions with the $\epsilon_{\alpha\beta}$ and 
$\sigma_{\dot{\alpha}\dot{\beta}}$ tensor raise and lower spinor indices.
% TODO is the last line necessary?

The standard rep  $S$ corresponds to the index $(\frac{1}{2},0)$ while 
$\overline{S}$ to $(0,\frac{1}{2})$. Another example is the Dirac spinor, which
corresponds to the $(\frac{1}{2},0)\oplus (0,\frac{1}{2})$ rep. 


\section{Supersymmetry algebra}
% West ch2
In the 1960s, with the growing awareness of the significance of internal
symmetries such as $\SU(2)$ in electroweak theory and larger groups,
physicists attempted to find a Lie group which would combine the spacetime
Poincare group with an internal symmetry group. After much effort, Coleman and
Mandula proved under general assumptions that any Lie group containing the
Poincare group $P$ and an internal symmetry group $G$ which is consistent with
QFT, must be a direct product of $P$ and  $G$, i.e. their generators commute.

The supersymmetry algebra avoids the restrictions of the Coleman-Mandula theorem
by relaxing the notion of a Lie algebra to include generators whose
defining relations involve anticommutators as well as commutators, called
a Lie superalgebra. 

\begin{defn}
	A \underline{Lie superalgebra} is a 
	$\mathbb{Z}_2$-graded algebra $\mathfrak{g} = \mathfrak{g}_0 \oplus \mathfrak{g}_1$ 
	over a commutative ring (usually $\mathbb{R}$ or $\mathbb{C}$)  with 
	Lie bracket $[\cdot,\cdot]:\mathfrak{g}\times \mathfrak{g} \to \mathfrak{g}$ satisfying 
	\begin{itemize}
		\item Super skew-symmetry: $[x,y] = -(-1)^{\abs{x}\abs{y}}[y,x]$
		\item Super Jacobi identity: 
		\[
			(-1)^{\abs{x}\abs{z}}[x,[y,z]]+(-1)^{\abs{y}\abs{x}}[y,[z,x]]
			+(-1)^{\abs{z}\abs{y}}[z,[x,y]]=0
		\]
	\end{itemize}
\end{defn}
Note that the Lie bracket is the multiplication operation in the graded algebra,
so it must respect the grading, and is bilinear. This product acts as an
anticommutator on two degree 1 elements, and as a commutator if either of them
have degree 0. 

Recall that the Poincar\'e group $\mathbb{R}^{1,3} \rtimes \O(1,3)$ is the 
semidirect product of translations and the Lorentz group with group
multiplication $(\alpha,f) \cdot (\beta, g) = (\alpha + f\cdot \beta, f\cdot
g)$. Its universal cover is the double cover  $\mathbb{R}^{1,3}\rtimes
\SL(2,\mathbb{C})$. In component form, the Poincar\'e algebra has the 10
dimensional basis $\{P_\mu,M_{\mu\nu}\}$ where $\mu<\nu$ and  $\mu,\nu\in
\{0,1,2,3\}$. Here  $P_\mu$ are the four generators of translations, and
$M_{\mu\nu}$ are the six generators of boosts and rotations, with commutation relations
\begin{align*}
	[P_\mu, P_\nu] &= 0 \\
	-i[M_{\mu\nu}, P_{\rho}] &= \eta_{\mu\rho}P_{\nu} - \eta_{\nu\rho}P_{\mu} \\
	-i[M_{\mu\nu}, M_{\rho\sigma}] &= \eta_{\mu\rho}M_{\nu\sigma} -
	\eta_{\mu\sigma}M_{\nu\rho} - \eta_{\nu\rho}M_{\mu\sigma} +
	\eta_{\nu\sigma}M_{\mu\rho} 
\end{align*}
where $\eta$ is the  $(+,-,-,-)$ Minkowski metric. 

\begin{comment} % from wiki Poincare group
	From the previous section, we know the Lorentz group admits two inequivalent 
	two-dimensional complex spin representations 2 and $\overline{2}$ whose tensor 
	product decomposes as $2\otimes\overline{2} \simeq 3 \oplus 1$, to give the
	adjoint representation. 
	Normally, we treat such a decomposition as relating to specific particles,
	e.g. the pion is a quark and anti-quark pair. However, we can also identify
	$3\oplus 1$ with Minkowski spacetime. 

	This leads to the question: if Minkowski spacetime belongs to the adjoint
	representation, can Poincar\'e symmetry be extended to the fundamental
	representation? 

	The physical appeal of this idea is that the fundamental representations
	correspond to fermions. 
	So far, however, the implied supersymmetry here,
	of a symmetry between spatial and fermionic directions, has not been seen
	experimentally in nature. 
\end{comment}

without central charges

% See section 4.1 labastida 
A supersymmetry algebra is a Lie superalgebra such that the even part is a Lie
algebra containing the Poincar\'e algebra, and the odd part is built from
spinors, on which there is an anticommutation relation. 
In the simplest supersymmetry algebra, the even part is the Poincar\'e algebra and
the odd part is generated by $Q_{\alpha} \in S$, and
$\overline{Q}_{\dot{\alpha}} \in \overline{S}$, with the Lie bracket relations
\[
	\{Q_\alpha,\overline{Q}_{\dot{\beta}}\} 
	= 2\sigma^{\mu}_{\alpha \dot{\beta}} P_{\mu} 
	\qquad
	\{Q_\alpha,Q_{\beta}\} = 0
	\qquad
	\{\overline{Q}_{\dot{\alpha}}, \overline{Q}_{\dot{\beta}}\} = 0 
\] 
\begin{equation*}
	\begin{split}
		&[M^{\mu\nu}, Q_{\alpha}] 
		= \frac{1}{2} (\sigma^{\mu\nu})_{\alpha}{}^{\beta}Q_\beta \\
		&[P_{\mu}, Q_{\alpha}] 
		= 0 
	\end{split}
	\qquad
	\begin{split}
		&[M^{\mu\nu}, \overline{Q}^{\dot{\alpha}}] = 
		-(\sigma^{\mu\nu})^{\alpha}{}_{\dot{\beta}}\overline{Q}^{\dot{\beta}}\\
		&[P_\mu, \overline{Q}_{\dot{\alpha}}] = 0
	\end{split}
\end{equation*}
where $(\sigma^{\mu\nu})_{\alpha}{}^{\beta}$ is a homomorphism between the
representations $S\oplus\overline{S}$ to the vector representation of
$\SO(1,3)$ using the Littlewood-Richardson rule.

- Why not specify [Q bar, P], or [Q bar, M], are they implied from [Q,P] \\
- where do the commutation relations come from? \\
- how is each vector component $Q_{\alpha}\in S$ a generator? 

\begin{defn}
	superspace
\end{defn}

Functions on superspace = differential forms

integration on supermanifolds using (Haar?) measure


\begin{thm}[Chebsch-Gordan rule]
	Let $V_m$ and $V_n$ be the irreducible representations of
	$\mathfrak{sl}(2,\mathbb{C})$ of highest weights $m$ and $n$ respectively.
	Assume  $m\geq n$, then
	\[
	V_m \otimes V_n \simeq V_{m+n} \oplus V_{m+n-2} \oplus \cdots \oplus 
	V_{m-n+2} \oplus V_{m-n}
	\] 
	is an isomorphism of representations of $\mathfrak{sl}(2,\mathbb{C})$
\end{thm}
\begin{comment}
	The idea of the proof is to define formal characters of lie algebra reps.
	prove two reps are isomorphic iff they have the same character. 
	Then show how the character behaves under direct sum and tensor product.
	Then show that the two sides have the same character.
\end{comment}
