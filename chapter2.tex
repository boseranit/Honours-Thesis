\chapter{Equivariant Cohomology}
\label{chapter2}

\section{Preliminaries}
Equivariant differential topology extends the results of differential topology
to manifolds/topological spaces with a group action, called a $G$-space. A
$G$-space is a topological space $X$ with a continuous action $X\times G \to X$ 
such that $x\cdot e = x$ for $e$ the identity in  $G$ and $(x\cdot g) \cdot h =
x\cdot (gh)$ for $g,h\in G$. 

We will often be dealing with both left and right group actions, but these are
equivalent. If we have a right $G$ action on $X$, this can be turned into a left
action by defining  $g \cdot x = x \cdot g^{-1}$, so that $h\cdot (g\cdot x) =
(x\cdot g^{-1})\cdot h^{-1} = x\cdot (g^{-1}h^{-1}) = (hg)\cdot x$. The same
argument works in the other direction.


There are two main formulations of equivariant
cohomology: the Borel construction which uses classifying spaces, and the Cartan model.
These have been proven to be equivalent in the case of compact Lie groups 
by Cartan (Theorem \ref{thm:equivariant_de_Rham} and \ref{thm:weil_cartan_iso}). 

First let us recall some basic definitions in algebraic topology. All maps
considered are continuous.
\begin{defn}
	Let $(X,x_0)$ and $(Y,y_0)$ be based topological spaces. Two maps
	$f,g:(X,x_0)\to(Y,y_0)$ are \underline{homotopic} if there is a continuous
	map $F:X\times [0,1]\to Y$ such that
	$F(x,0)=f(x),F(x,1)=g(x),F(x_0,t)=y_0$. Then we denote $f\sim g$.

	A \underline{homotopy equivalence} is a continuous map $f:(X,x_0)\to(Y,y_0)$
	that has a homotopy inverse, i.e. a continuous map $g:(Y,y_0)\to(X,x_0)$
	such that $f\circ g \sim 1_Y$ and $g\circ f \sim 1_X$.
	Then we say $X$ and  $Y$ have the same homotopy type.

	A topological space $(X,x_0)$ is \underline{contractible} if it has the
	homotopy type of a point. 

	A map $f:X\to Y$ is a \underline{weak homotopy equivalence} if
	it induces an isomorphism of homotopy groups
	$f_*:\pi_q(X)\to\pi_q(Y)$ for all  $q\geq 0$. 
	A space $X$ is \underline{weakly contractible} if $\pi_q(X)=0$ for all  $q\geq 0$. 
\end{defn}

%TODO reference
\begin{thm}[Whitehead's theorem	{\cite[Thm 4.5]{hatcher}}] 
	If a continuous map $f:X\to Y$ of CW complexes 
	is a weak homotopy equivalence, then  $f$ is a homotopy equivalence. 
\end{thm}
In particular, this
means that a weakly contractible CW complex is contractible, using the inclusion
map $x_0 \to X$. 
% every smooth manifold is homotopy equivalent to a CW complex pg 18 Tu

\begin{thm}[{\cite[Prop 4.21]{hatcher}}] \label{thm:weak_to_cohomology}
	A weak homotopy equivalence $f:X\to Y$ induces isomorphisms
	$f^*:H^n(Y;R)\to H^n(X;R)$ in cohomology for all  $n$.
\end{thm}
A useful tool for computing homotopy groups is the homotopy exact sequence of a
fiber bundle. 
\begin{thm}[{\cite[Thm 4.41]{hatcher}}] \label{thm:fiber_les}
	Suppose $(E,x_0) \xrightarrow{\pi} (B,b_0)$ is a fiber bundle with fiber
	$F=\pi^{-1}(b_0)$ and path-connected base space $B$. Let  $x_0$ also be the
	basepoint of $F$, and  $i:(F,x_0)\to (E,x_0)$ the inclusion map. Then there
	exists a long exact sequence
	\[
		\ldots\to\pi_n(F,x_0) \xrightarrow{i_*}\pi_n(E,x_0)
		\xrightarrow{\pi_*}\pi_n(B,b_0)
		\to \pi_{n-1}(F,x_0)\to\ldots\to\pi_0(E,x_0)\to 0
	\] 
	All maps are group homomorphisms except the last three maps which are set
	maps.
\end{thm}

\begin{prop} \label{prop:total_base_iso}
	If $E \xrightarrow{\pi} B$ is a fiber bundle with weakly contractible fiber $F$,
	and path-connected base $B$, $\pi$ is a weak homotopy equivalence.
	Furthermore if $B$ and $F$ are CW complexes, 
	then $\pi$ is a homotopy equivalence.  
\end{prop}
\begin{proof}
	By Theorem \ref{thm:fiber_les}, the sequence of induced maps
	\[
	\ldots \to \pi_n(F) \to \pi_n(E) \to\pi_n(B) \to
	\pi_{n-1}(F)\to \ldots
	\] 
	is exact. So the induced maps $\pi_* : \pi_n(E) \to
	\pi_n(B)$ on homotopy groups are isomorphisms. Hence $\pi$ is a weak
	homotopy equivalence. 

	Now suppose in addition that $B$ and $F$ are CW complexes. 
	Since each cell of $B$ is contractible, the fiber bundle
	restricted to a cell is trivial and is homeomorphic to  cell$\times F$. 
	Since  $F$ also has a CW structure, the product is also a CW complex, and
	gluing these pieces together shows that $E$ is a CW complex. Hence, by
	Whitehead's theorem,  $\pi$ is a homotopy equivalence. 
\end{proof}

Let $G$ be a topological group.
If  $P$ is a right  $G$-space and  $M$ is a left  $G$-space, the 
\underline{Cartan mixing space} of  $P$ and  $M$ is the quotient 
$P\times_G M:= (P\times M) / \sim$ by the 
equivalence relation $(p,m)\sim (pg,g^{-1}m) \textrm{ for all }g\in G$.
Equivalently, this is the orbit space $(P\times M) /G$ under the diagonal action
$g(p,m) = (pg,g^{-1}m)$.

If in addition $P\xrightarrow{\pi} B$ is a principal $G$-bundle, define the projection
$\tau_1:P\times_GM\to B$ by $\tau_1([p,m])=\pi(p)$. This is well defined because
$\pi$ preserves the fiber.

\begin{prop} \label{prop:cartan_mixing}% prop 4.5 Tu
	If $P\xrightarrow{\pi} B$ is a principal  $G$-bundle and  $M$ is a left  $G$-space,
	then  $\tau_1 : P\times_G M\to B$ is a fiber bundle with fiber  $M$.
\end{prop}
\begin{proof}
	Suppose $\pi^{-1}(U)\simeq U\times G$. It suffices to show
	$\tau_1^{-1}(U)\simeq U\times M$. 
	\begin{align*}
		\tau_1^{-1}(U)
		&= \{[p,m]\in P\times_GM \mid \pi(p)\in U\} \\
		&= \pi^{-1}(U)\times _G M \\
		&\simeq (U\times G) \times_G M 
	\end{align*}
	To show this is homeomorphic to $U\times M$, we define 
	$\varphi:(U\times G) \times_G M \to U\times M$ by 
	$[(x,g),m] \mapsto (x,gm)$. It has inverse $(x,m)\mapsto [(x,1),m]$.
\end{proof}

\section{Borel construction}
Given a $G$-space  $M$, the  aim of equivariant cohomology is to study the
cohomology of the quotient space by the group action. But if the $G$-action is not free,
the space $M/G$ doesn't capture information from non-trivial stabilisers. For
instance, if  $S^1$ acts on $S^2$ by rotation about the vertical axis, the
quotient is a segment which has trivial cohomology. Furthermore, if the action is not
free and we have a $G$-equivariant homotopy equivalence  $f:M\to N$, then  
the induced map $M /G \to N /G$ is not necessarily a homotopy
equivalence. As an example, consider $M=\mathbb{R}$ with $\mathbb{Z}$-action
given by translation, and a point $N=*$ with trivial $\mathbb{Z}$-action. Then
$\mathbb{R} /\mathbb{Z} \simeq S^1$ but $N / \mathbb{Z}$ is a point.

To address these limitations, the idea behind the Borel constuction is to force 
the action to be free by replacing  $M$ with  $E\times M$ where $E$ is a  
$G$-space with a free action, and then studying the quotient  $(E\times M) /G$. 
But we need an appropriate choice of $E$ such that the cohomology does not 
depend on it.

\begin{defn}
	Given a topological group $G$, let  $EG \to BG$ be a principal  $G$-bundle
	with weakly contractible total space  $EG$. Define the  \underline{homotopy
	quotient} of a $G$-space $M$ by $M_G:=EG\times_G M$.
	 
	The \underline{equivariant cohomology} of $M$ is defined to be the singular
	cohomology of the homotopy quotient: $H_G^*(M;R) := H^*(M_G;R)$.
\end{defn}
Of course, for this definition to make sense we need to show that it is
independent of the choice of weakly contractible $EG$ and find out when $EG$ 
exists, which we do in this section. 

The following result shows that when the action is free, equivariant cohomology
is the cohomology of the orbit space.
\begin{prop}
	If $M$ has a free  $G$-action, then $H^*(M_G)\simeq H^*(M /G)$. 
\end{prop}
\begin{proof}
	Let $EG \to BG$ be a principal  $G$-bundle with weakly contractible total
	space. Since the $G$-action is free,  $M \to M /G$ is a principal bundle.
	By Proposition \ref{prop:cartan_mixing}, 
	$(EG\times M) /G \to M /G$ is a fiber bundle with
	fibre homeomorphic to $EG$. Then by Proposition \ref{prop:total_base_iso},
	the result follows.
\end{proof}

\begin{lem} % lemma 4.9 Tu
	If $E$ is a weakly contractible $G$-space, and $P\to P/G$ is a principal
	$G$-bundle, there is a weak homotopy equivalence $(E\times P) /G \to P /G$.  
\end{lem}
\begin{proof}
	Consider $E\times_G P = (E\times P) /G$ as the orbit space under 
	the diagonal action $(e,p)g = (g^{-1}e,pg)$. 
	By Proposition \ref{prop:cartan_mixing}, $(E\times P) /G \to P /G$ is a
	fiber bundle with fiber $E$. Then the result follows by Proposition
	\ref{prop:total_base_iso}.
\end{proof}

The next proposition shows that the definition of equivariant cohomology is
independent of the choice of  $E$. 
\begin{prop} % Tu Thm 4.10 doesn't work, not weak homotopy equivalence
	Suppose $M$ is a left  $G$-space. If $E\to B$ and  $E'\to B'$ are two principal
	$G$-bundles with weakly contractible total spaces, then $H^*(E\times_G M)
	\simeq H^*(E'\times_G M)$.
\end{prop}
\begin{proof}
	Since the $G$-action on  $E'\times M$ is also free,  $E'\times M \to
	(E'\times M) /G$ is a principal bundle. Then by the lemma above, 
	there is a weak homotopy equivalence 
	$(E\times E' \times M) /G \to (E'\times M) /G$. By 
	Theorem \ref{thm:weak_to_cohomology}, it
	induces isomorphisms in cohomology  $H^n((E\times E'\times M )/G) \simeq
	H^n((E'\times M) /G)$ for all  $n\geq 0$.

	By symmetry of $E$ and  $E'$, we also conclude  $H^n((E'\times E\times M )/G) \simeq
	H^n((E\times M) /G)$ for all  $n\geq 0$. The canonical homeomorphism
	$(E'\times E \times M) /G \to (E\times E'\times M) /G$ induces an
	isomorphism on cohomology, and thus $H^n((E\times M) /G) \simeq
	H^n((E'\times M) /G)$.
\end{proof}

The remaining question in our definition of equivariant cohomology of a
$G$-space $M$ is the existence of a weakly contractible principal $G$-bundle $EG$. 
It turns out that for CW complexes,  
a weakly contractible $G$-bundle is equivalent to a universal $G$-bundle, as
defined below.
\begin{defn}
	A principal $G$-bundle  $\pi:EG\to BG$ is a \underline{universal $G$-bundle} 
	if the following two conditions hold:
	\begin{enumerate}[(i)]
	    \item for any principal $G$-bundle  $P$ over a CW complex $X$, there
			exists a continuous map  $h:X\to BG$ such that  $P \simeq h^*EG$ 
		\item If $h_0,h_1:X\to BG$ and $h_0^*EG \simeq h_1^*EG$ over a CW
			complex $X$, then  $h_0$ and $h_1$ are homotopic
	\end{enumerate}
	The base space $BG$ of a universal  $G$-bundle is called a
	\underline{classifying space} for  $G$.
\end{defn}
\begin{thm}[Steenrod 1951] % thm 5.2 Tu % TODO: reference
	Let $E\to B$ be a principal  $G$-bundle. If  $E$ is weakly contractible,
	then  $E\to B$ is a universal bundle. Conversely, if  $E\to B$ is a
	universal bundle and  $B$ is a CW complex, then  $E$ is weakly contractible.
\end{thm}
\begin{thm}[Milnor's construction] % TODO reference
	A universal $G$-bundle exists for any topological group  $G$.
\end{thm}

\begin{comment}
Let $X$ be a CW complex, and $[X,B]$ be the homotopy classes of maps $h:X\to B$, 
and  $\mathcal{P}_G(X)$ be the isomorphism classes of principal $G$-bundles over $X$. 
Then the definition of universal $G$-bundle states the map
$\varphi:[X,BG]\to\mathcal{P}_G(X)$ given by $h\mapsto h^*(EG)$ is surjective
(condition (i)) and injective (condition (ii)). 
% well defined by Theorem 5.3 Tu
\end{comment}

\begin{thm} %thm 5.6
	If a CW classifying space exists for a topological group $G$, it is unique 
	up to homotopy equivalence.
\end{thm}
\begin{proof}
	Suppose $E\to B$ and  $E'\to B'$ are universal bundles, where $B$ and  $B'$
	are CW complexes. Since  $E'$ is
	universal there is a map  $f:B\to B'$ such that  $E'\simeq h^*E$. Similarly
	there is a map  $h:B'\to B$ such that  $E\simeq h^*E'$. Therefore,  $E\simeq
	f^*h^*E=(h\circ f)^*E$. 

	But this means  $(h\circ f)^*E = \id_B^*E$, so by condition (ii) of a
	universal bundle, $h\circ f \simeq \id_B$. Similarly, $f\circ h\simeq
	\id_{B'}$. Therefore $B$ and  $B'$ are homotopy equivalent.
\end{proof}



\section{Cartan model}
\subsection{Differential graded algebras}
\begin{defn} \label{def:contraction} % BGV
	If $V$ is a vector space, the \underline{contraction} (or interior
	multiplication) operator $\iota(v) :
	\Lambda V^* \to \Lambda V^*$ for $v\in V$ is the unique operator such that
	\begin{enumerate}[(1)]
	    \item $\iota(v)\alpha = \alpha(v)$ if  $\alpha\in V^*$
		\item $\iota(v)(\alpha\wedge\beta) = (\iota(v)\alpha)\wedge\beta + 
			(-1)^{\abs{\alpha}}\alpha\wedge(\iota(v)\beta)$, if  $\alpha,\beta$
			are homogeneous elements of  $\Lambda V^*$.
	\end{enumerate} % mention exterior operator?
\end{defn}
Hence, we can define \underline{interior multiplication} with a vector
field as the map $\iota(X) : \Omega^*(M) \to \Omega^{*-1}(M)$.
The definition similarly extends to vector valued forms.
\begin{thm} % thm 10.4 tu equivariant, p18 BGV
	Let $X$ be a vector field on  $M$, and $\alpha\in \Omega(M)$. 
	\begin{enumerate}[(i)]
	    \item $\mathcal{L}(X)d = d\mathcal{L}(X)$
		\item $\mathcal{L}(X)(\iota(Y)\alpha) = \iota([X,Y])\alpha +
			\iota(Y)(\mathcal{L}(X)\alpha)$ 
		\item Cartan's homotopy formula: $\mathcal{L}(X) = d
			\cdot\iota(X)+\iota(X)\cdot d$
	\end{enumerate}
\end{thm}
\begin{proof}
	% TODO
\end{proof}

\begin{defn}
	Let $\mathfrak{g}$ be a Lie algebra. A \underline{$\mathfrak{g}$-differential graded
	algebra} is a commutative graded algebra $\Omega= \bigoplus_{k\geq
	0}\Omega^k$ with 
	\begin{itemize}
		\item an antiderivation $d:\Omega\to\Omega$ of degree 1 such that
	$d\circ d = 0$
		\item two actions of $\mathfrak{g}$: $\iota:\mathfrak{g}\times\Omega\to\Omega$
			and  $\mathcal{L}:\mathfrak{g}\times\Omega\to\Omega$, where for
			$X\in\mathfrak{g}$,  $\iota_X$ and  $\mathcal{L}_X$ are
			$\mathbb{R}$-linear in $X$,  $\iota_X$ acts on  $\Omega$ as an
			antiderivation of degree -1,  $\iota_x = 0$, and  $\mathcal{L}_X$
			acts as a derivation of degree 0.
	\end{itemize}
	Furthermore, the operators satisfy Cartan's homotopy formula:
	$\mathcal{L}_X= d\iota_X+\iota_Xd$.
\end{defn}
Note that commutativity for a graded algebra means that if $a\in
\Omega^k,b\in\Omega^l$ then $ba = (-1)^{kl}ab$.  

\begin{defn}
	A differential form $\alpha\in\Omega$ is 
	\underline{horizontal} if $\iota_X\alpha=0$ for all $X\in\mathfrak{g}$. 
	It is \underline{invariant} if $\mathcal{L}_X\alpha = 0$ for all $X\in
	\mathfrak{g}$. It is \underline{basic} if it is both horizontal and
	invariant.
\end{defn}
Note that in the case of a principal bundle, we think of $X\in \mathfrak{g}$ as 
a vertical vector field via $X_p = \odv{}{t}_{t=0} p \exp (t X)$. 
\begin{comment} %%% stuff about principal and associated bundles
\begin{defn}
	A differential form $\alpha\in\Omega(P,V)$ on a principal $G$-bundle  with 
	representation $(V,\rho)$ is \underline{$\rho$-equivariant}   
	if for every $g\in G$, $r_g^*\alpha = \rho(g^{-1})\alpha$. 
	\\
	A differential form $\alpha\in\Omega(P,V)$ on a principal $G$-bundle  with 
	representation $(V,\rho)$ is \underline{basic}  
	if it is horizontal and $\rho$-equivariant. This subspace is
	denoted $\Omega_{bas}(P,V)$.
\end{defn}
Let $P$ be a principle $G$-bundle, and let ($V,\rho$) be a representation of
$G$. Let $E= P\times_\rho V$ be the associated bundle.
\begin{thm} %thm 31.9 tu
	The map $\Omega_{bas}^k(P,V) \to \Omega^k(M,P\times_\rho V)$ given by 
	$\omega \mapsto \alpha_x = f_p \circ \omega_p$
	is a linear isomorphism, where $f_p: V\to E_x$ is the isomorphism  $v\mapsto
	[p,v]$, and $p\in\pi^{-1}(x)$ is any point.
\end{thm}

\begin{cor} % Prop 1.7 BGV
	There is a natural isomorphism between $\Gamma(M,P\times_\rho V)$ and
	$\rho$-equivariant maps in
	$C^{\infty}(P,V)$, given by sending $s\in C^{\infty}(P,V)^G$ to $s_M$
	defined by  $s_M(x) = [p,s(p)]$, where  $p\in\pi^{-1}(x)$ is any element.
\end{cor}


Recall that a \underline{vertical vector} on a fibre bundle $E$ with base  $M$
is a tangent vector  $X\in TE$ such that  $X(\pi^* f) = 0$ for any  $f\in
C^\infty(M)$. The space $V_pP$ of vertical tangent vectors to a point $p$ in a
principal $G$-bundle can be canonically identified with the Lie algebra $\mathfrak{g}$ 
of $G$ in the following way. 
If $X\in\mathfrak{g}$, the \underline{fundamental vector field} associated to
$X$, denoted $X_P \in \Gamma(VP)$, is
 \[
X_P = \odv{}{t}_{t=0} p \exp (t X)
\] 
This is a vertical vector field because $X_P(\pi^*f) = \odv{}{t}_{t=0} f(p \exp
(t X)) = 0$. 
\end{comment}


If $(A,d_A)$ and  $(B,d_B)$ are differential graded algebras, then their tensor product has
multiplication given by 
 \[
	 (a\otimes B)(a'\otimes b') = (-1)^{(\deg b)(\deg a')}aa'\otimes bb'
\] 
which respects the grading $(A\otimes B)^k = \bigoplus_{i+j= k} A^i\otimes B^j$.
The differential on $A\otimes B$ is defined by 
\[
d(a\otimes b) = (d_A a)\otimes b + (-1)^{\deg a}a\otimes d_Bb
\] 
The interior product is defined similarly as 
\[
\iota_X(a\otimes b) = (\iota_X a)\otimes b + (-1)^{\deg a}\otimes \iota_Xb
\] 
\begin{prop}[Operations on DGAs] % prop 18.7, 18.10 tu
	\begin{enumerate}[(i), leftmargin=\parindent]
	    \item 
			If $(A,d_A)$ and  $(B,d_B)$ are $\mathfrak{g}$-differential graded 
			algebras, then $(A\otimes B,d)$ is a $\mathfrak{g}$-differential graded 
			algebra
		\item 
	If $\Omega$ is a  $\mathfrak{g}$-differential graded algebra, the vector
	subspace of basic elements $\Omega_{bas}$ is a $\mathfrak{g}$-differential
	graded algebra.
	\end{enumerate}
\end{prop}
% TODO tensor product of dgas

\subsection{Weil model}
% TODO defined weil model follow ch19 Tu
\begin{thm}[Equivariant de Rham theorem] \label{thm:equivariant_de_Rham} % 19.4 Tu
	For a compact connected Lie group $G$ with Lie algebra $\mathfrak{g}$, and
	$G$-manifold  $M$, there is a graded-algebra isomorphism 
	 \[
		 H_G^*(M) \simeq H^*((W(\mathfrak{g})\otimes \Omega(M))_{bas}, \delta)
	\] 
\end{thm}
The complex $(W(\mathfrak{g})\otimes \Omega(M))_{bas}$ with the Weil
differential is called the \underline{Weil model}.

\begin{thm}[Weil-Cartan isomorphism] \label{thm:weil_cartan_iso} % Thm 21.1 Tu
	Let $G$ be a connected Lie group, and $M$ be a left $G$-manifold. There is a
	graded-algebra isomorphism 
	\[
		F : (W(\mathfrak{g})\otimes \Omega(M))_{hor} \to S(g^*\otimes \Omega(M))
	\] 
	\[
	a+ \sum \theta_I a_I \mapsto a
	\] 
	which induces a graded-algebra isomorphism on the basic algebras
	$F : F : (W(\mathfrak{g})\otimes \Omega(M))_{bas} \to (S(g^*\otimes
	\Omega(M)))^G$. 
\end{thm}
The complex $(S(g^*\otimes \Omega(M)))^G$ is called the \underline{Cartan
model}. Elements of the Cartan model are called \underline{equivariant forms}.

\section{BRST Model}
In the context of topological field theories, another model of equivariant
cohomlogy arises naturally, called the BRST model. As a vector space, it is
identical to the Weil model $W(\mathfrak{g})\otimes \Omega(M)$, but with
differential $d_B = d_W + \theta^i\otimes \mathcal{L}_i - u^i \otimes \iota_i$.

% TODO how does it arise? proof of conjugagion theorem, why conjugate?
It was shown by Kalkman that the BRST and Weil models are related by the algebra
automorphism of conjugation by $\exp (\theta^i \iota_i)$. 


\section{Equivariant characteristic classes}
\begin{defn} % from ch1 BGV
	Let $\pi: E\to M$ be a fibre bundle and let $G$ be a Lie group. We say $E$
	is a \underline{$G$-equivariant bundle} if $E$ and  $M$ are 
	$G$-manifolds  with $g \cdot \pi = \pi \cdot g$ for all  $g\in G$.\\
	If  $E$ is a vector bundle, we further require that the action  $g: E_x \to
	E_{gx}$ is linear.
\end{defn}

\begin{prop} % prop 9.4 Tu
	Let $f:M\to N$ be a $G$-equivariant map of  $G$-spaces, and let
	$f_G:M_G\to N_G$ be defined by $[e,x]\mapsto [e,f(x)]$. 
	\begin{enumerate}[(i)]
	    \item If $f$ is injective, then $f_G$ is injective
		\item If $f$ is surjective, then $f_G$ is surjective
		\item If $M\xrightarrow{f} N$ is a fiber bundle with fiber $F$, then
			$M_G\xrightarrow{f_G}N_G$ is a fiber bundle with fiber  $F$. 	
	\end{enumerate}
\end{prop}

Let $G$ be a topological group, and  $\pi : E\to M$ be a  $G$-equivariant vector
bundle. By the proposition above, this induces a vector bundle  $\pi_G : E_G\to
M_G$ on homotopy quotients of the same rank. If $E\to M$ is oriented, then so is
$E_G \to M_G$. 

The \underline{equivariant Euler class} of an oriented equivariant vector bundle
$\pi : E \to M$ is defined to be the Euler class of $\pi_G : E_G \to M_G$, i.e.
it is an element of $H_G^*(M)$. 

\vspace{5mm}
\hrule 
\vspace{5mm}

\textbf{Bibliographical notes}
{\small
\begin{itemize}
	\item An excellent book on equivariant cohomology which this chapter is
	largely based on is \citetitle{equivariant_tu} by
	\citet{equivariant_tu}.
\end{itemize}
}

\hrule
