\chapter{Equivariant Cohomology}
\label{chapter2}

\section{Preliminaries}
Equivariant differential topology extends the results of differential topology
to manifolds/topological spaces with a group action, called a $G$-space. A
$G$-space is a topological space $X$ with a continuous action $X\times G \to X$ 
such that $x\cdot e = x$ for $e$ the identity in  $G$ and $(x\cdot g) \cdot h =
x\cdot (gh)$ for $g,h\in G$. 

We will often be dealing with both left and right group actions, but these are
equivalent. If we have a right $G$ action on $X$, this can be turned into a left
action by defining  $g \cdot x = x \cdot g^{-1}$, so that $h\cdot (g\cdot x) =
(x\cdot g^{-1})\cdot h^{-1} = x\cdot (g^{-1}h^{-1}) = (hg)\cdot x$. The same
argument works in the other direction.


There are two definitions of equivariant
cohomology: the Borel construction which uses classifying spaces, and the Cartan model.
These have been proven to be equivalent in the case of compact manifolds with
compact group actions by Cartan. % TODO reference theorem, venselaar

First let us recall some basic definitions in algebraic topology.
\begin{defn}
	Let $(X,x_0)$ and $(Y,y_0)$ are based topological spaces. Two maps
	$f,g:(X,x_0)\to(Y,y_0)$ are \underline{homotopic} if there is a continuous
	map $F:X\times [0,1]\to Y$ such that
	$F(x,0)=f(x),F(x,1)=g(x),F(x_0,t)=y_0$. Then we denote $f\sim g$.

	A \underline{homotopy equivalence} is a continuous map $f:(X,x_0)\to(Y,y_0)$
	that has a homotopy inverse, i.e. a continuous map $g:(Y,y_0)\to(X,x_0)$
	such that $f\circ g \sim \mathbb{1}_Y$ and $g\circ f \sim \mathbb{1}_X$.
	Then we say $X$ and  $Y$ have the same homotopy type.

	A topological space $(X,x_0)$ is \underline{contractible} if it has the
	homotopy type of a point. 
	A space $X$ is \underline{weakly contractible} if all its homotopy groups
	are trivial, i.e.  $\pi_q(X)=0$ for all  $q\geq 0$. 
\end{defn}
An important theorem in homotopy theory is Whitehead's theorem, which states
that if a continuous map $f:X\to Y$ of CW complexes induces an isomorphism on
all homotopy groups, then  $f$ is a homotopy equivalence. In particular, this
means that a weakly contractible CW complex is contractible, using the inclusion
map $x_0 \to X$.
% every smooth manifold is homotopy equivalent to a CW complex pg 18 Tu

A useful tool for computing homotopy groups is the homotopy exact sequence of a
fiber bundle. 
\begin{thm}[{\cite[Thm 4.41]{hatcher}}] \label{thm:fiber_les}
	Suppose $(E,x_0) \xrightarrow{\pi} (B,b_0)$ is a fiber bundle with fiber
	$F=\pi^{-1}(b_0)$ and path-connected base space $B$. Let  $x_0$ also be the
	basepoint of $F$, and  $i:(F,x_0)\to (E,x_0)$ the inclusion map. Then there
	exists a long exact sequence
	\[
		\ldots\to\pi_n(F,x_0) \xrightarrow{i_*}\pi_n(E,x_0)
		\xrightarrow{\pi_*}\pi_n(B,b_0)
		\to \pi_{n-1}(F,x_0)\to\ldots\to\pi_0(E,x_0)\to 0
	\] 
	All maps are group homomorphisms except the last three maps which are set
	maps.
\end{thm}

In this section, assume $G$ is a topological group. 
Cartan's mixing contruction, also
called Borel's construction, turns a principal  $G$-bundle and a  $G$-space  $M$
into a fiber bundle with fiber  $M$. 

If  $P$ is a right  $G$-space and  $M$ is a left  $G$-space, the 
\underline{Cartan mixing space} of  $P$ and  $M$ is the quotient 
$P\times_G M:= (P\times M) / \sim$ by the 
equivalence relation $(p,m)\sim (pg,g^{-1}m) \textrm{ for some }g\in G$.
Equivalently, this is the orbit space $(P\times M) /G$ under the diagonal action
$g(p,m) = (pg,g^{-1}m)$.

If in addition $P\xrightarrow{\pi} B$ is a principal $G$-bundle, define the projection
$\tau_1:P\times_GM\to B$ by $\tau_1([p,m])=\pi(p)$. This is well defined because
$\pi$ preserves the fiber.

\begin{prop} \label{prop:cartan_mixing}% prop 4.5 Tu
	If $P\xrightarrow{\pi} B$ is a principal  $G$-bundle and  $M$ is a left  $G$-space,
	then  $\tau_1 : P\times_G M\to B$ is a fiber bundle with fiber  $M$.
\end{prop}
\begin{proof}
	Suppose $\pi^{-1}(U)\simeq U\times G$. It suffices to show
	$\tau_1^{-1}(U)\simeq U\times M$. 
	\begin{align*}
		\tau_1^{-1}(U)
		&= \{[p,m]\in P\times_GM \mid \pi(p)\in U\} \\
		&= \pi^{-1}(U)\times _G M \\
		&\simeq (U\times G) \times_G M 
	\end{align*}
	To show this is homeomorphic to $U\times M$, we define 
	$\varphi:(U\times G) \times_G M \to U\times M$ by 
	$[(x,g),m] \mapsto (x,gm)$. It has inverse $(x,m)\mapsto [(x,1),m]$.
\end{proof}


\section{Borel construction}
\begin{defn}
	Given a topological group $G$, let  $EG \to BG$ be a principal  $G$-bundle
	with weakly contractible total space  $EG$. Define the  \underline{homotopy
	quotient} of a $G$-space $M$ by $M_G:=EG\times_G M$.
	 
	The \underline{equivariant cohomology} of $M$ is defined to be the singular
	cohomology of the homotopy quotient: $H_G^*(M;R) := H^*(M_G;R)$.
\end{defn}
Of course, for this definition to make sense we need to show that it is
independent of the choice of weakly contractible $EG$, which we do now. 

\begin{defn}
	A map $f:X\to Y$ is a \underline{weak homotopy equivalence}
	it induces an isomorphism of homotopy groups
	$f_*:\pi_q(X)\to\pi_q(Y)$ for all  $q\geq 0$. 
\end{defn}

\begin{lem} % lemma 4.9 Tu
	If $E$ is a weakly contractible $G$-space, and $P\to P/G$ is a principal
	$G$-bundle, there is a weak homotopy equivalence $(E\times P) /G \to P /G$.  
\end{lem}
\begin{proof}
	We consider $E\times_G P = (E\times P) /G$ as the orbit space under 
	the diagonal action $(e,p)g = (g^{-1}e,pg)$. 
	By Proposition \ref{prop:cartan_mixing}, $(E\times P) /G \to P /G$ is a
	fiber bundle with fiber $E$. By the long exact sequence of a fiber
	bundle (Theorem \ref{thm:fiber_les}), the sequence of induced maps
	\[
	\ldots \to \pi_q(E) \to \pi_q((E\times P) /G) \to\pi_q(P /G) \to
	\pi_{q-1}(E)\to \ldots
	\] 
	is exact. Since $E$ is weakly contractible, the middle map is an isomorphism
	 $\pi_q((E\times P) /G) \to \pi_q(P /G)$ for any  $q\geq 0$.
\end{proof}

\begin{cor}
	Suppose $M$ is a left  $G$-space. If $E\to B$ and  $E'\to B'$ are two principal
	$G$-bundles with weakly contractible total spaces, then $E\times_G M$ and
	$E'\times_G M$ are weakly homotopy equivalent.
\end{cor}
\begin{proof}
	Note that homotopy equivalence is not a symmetric relation, but by symmetry
	of $E$ and  $E'$ both directions hold. 
	Applying the previous lemma to $E$ and $E'\times M$, there is a weak
	homotopy equivalence $(E\times E'\times M)/G$ to $(E'\times M) /G$.

\end{proof}
\begin{thm}[{\cite[Proposition 4.21]{hatcher}}]
	A weak homotopy equivalence $f:X\to Y$ induces isomorphisms
	$f^*:H^n(Y;R)\to H^n(X;R)$ in cohomology for all  $n$.
\end{thm}

Hence, combining the last two results tells us that for any two choices of
weakly contractible principal $G$-bundles  $E$ and  $E'$,  $H^*(E\times_G M)
\simeq H^*(E'\times_G M)$ so the definition of equivariant cohomology is
independent of the choice of  $E$. 

To define the equivariant cohomology of a $G$-space  $M$, we need a weakly
contractible principal $G$- bundle $EG$. It turns out that for CW complexes,  
a weakly contractible $G$-bundle is equivalent to a universal $G$-bundle, as
defined below.
\begin{defn}
	A principal $G$-bundle  $\pi:EG\to BG$ is a \underline{universal $G$-bundle} 
	if the following two conditions hold:
	\begin{enumerate}[(i)]
	    \item for any principal $G$-bundle  $P$ over a CW complex $X$, there
			exists a continuous map  $h:X\to BG$ such that  $P \simeq h^*EG$ 
		\item If $h_0,h_1:X\to BG$ and $h_0^*EG \simeq h_1^*EG$ over a CW
			complex $X$, then  $h_0$ and $h_1$ are homotopic
	\end{enumerate}
	The base space $BG$ of a universal  $G$-bundle is called a
	\underline{classifying space} for  $G$.
\end{defn}
\begin{thm}[Steenrod 1951] % thm 5.2 Tu
	Let $E\to B$ be a principal  $G$-bundle. If  $E$ is weakly contractible,
	then  $E\to B$ is a universal bundle. Conversely, if  $E\to B$ is a
	universal bundle and  $B$ is a CW complex, then  $E$ is weakly contractible.
\end{thm}

\begin{comment}
Let $X$ be a CW complex, and $[X,B]$ be the homotopy classes of maps $h:X\to B$, 
and  $\mathcal{P}_G(X)$ be the isomorphism classes of principal $G$-bundles over $X$. 
Then the definition of universal $G$-bundle states the map
$\varphi:[X,BG]\to\mathcal{P}_G(X)$ given by $h\mapsto h^*(EG)$ is surjective
(condition (i)) and injective (condition (ii)). 
% TODO: well defined by Theorem 5.3 Tu
\end{comment}

\begin{thm} %thm 5.6
	If a CW classifying space exists for a topological group $G$, it is unique 
	up to homotopy equivalence.
\end{thm}
\begin{proof}
	Suppose $E\to B$ and  $E'\to B'$ are universal bundles, where $B$ and  $B'$
	are CW complexes. Since  $E'$ is
	universal there is a map  $f:B\to B'$ such that  $E'\simeq h^*E$. Similarly
	there is a map  $h:B'\to B$ such that  $E\simeq h^*E'$. Therefore,  $E\simeq
	f^*h^*E=(h\circ f)^*E$. 

	But this means  $(h\circ f)^*E = \id_B^*E$, so by condition (ii) of a
	universal bundle, $h\circ f \simeq \id_B$. Similarly, $f\circ h\simeq
	\id_{B'}$. Therefore $B$ and  $B'$ are homotopy equivalent.
\end{proof}

Moreover, it can be shown that a universal $G$-bundle exists for any topological
group  $G$. % Milnor's construction TODO: elaborate?


\section{Cartan model}
\begin{defn} \label{def:contraction} % BGV
	If $V$ is a vector space, the \underline{contraction} (or interior
	multiplication) operator $\iota(v) :
	\Lambda V^* \to \Lambda V^*$ for $v\in V$ is the unique operator such that
	\begin{enumerate}[(1)]
	    \item $\iota(v)\alpha = \alpha(v)$ if  $\alpha\in V^*$
		\item $\iota(v)(\alpha\wedge\beta) = (\iota(v)\alpha)\wedge\beta + 
			(-1)^{\abs{\alpha}}\alpha\wedge(\iota(v)\beta)$, if  $\alpha,\beta$
			are homogeneous elements of  $\Lambda V^*$.
	\end{enumerate} % mention exterior operator?
\end{defn}
Hence, we can define \underline{interior multiplication} with a vector
field as the map $\iota(X) : \Omega^*(M) \to \Omega^{*-1}(M)$.
The definition similarly extends to vector valued forms.
\begin{thm} % thm 10.4 tu equivariant, p18 BGV
	Let $X$ be a vector field on  $M$, and $\alpha\in \Omega(M)$. 
	\begin{enumerate}[(i)]
	    \item $\mathcal{L}(X)d = d\mathcal{L}(X)$
		\item $\mathcal{L}(X)(\iota(Y)\alpha) = \iota([X,Y])\alpha +
			\iota(Y)(\mathcal{L}(X)\alpha)$ 
		\item Cartan's homotopy formula: $\mathcal{L}(X) = d
			\cdot\iota(X)+\iota(X)\cdot d$
	\end{enumerate}
\end{thm}
\begin{proof}
	
\end{proof}

\begin{defn}
	Let $\mathfrak{g}$ be a Lie algebra. A \underline{$\mathfrak{g}$-differential graded
	algebra} is a commutative graded algebra $\Omega= \bigoplus_{k\geq
	0}\Omega^k$ with 
	\begin{itemize}
		\item an antiderivation $d:\Omega\to\Omega$ of degree 1 such that
	$d\circ d = 0$
		\item two actions of $\mathfrak{g}$: $\iota:\mathfrak{g}\times\Omega\to\Omega$
			and  $\mathcal{L}:\mathfrak{g}\times\Omega\to\Omega$, where for
			$X\in\mathfrak{g}$,  $\iota_X$ and  $\mathcal{L}_X$ are
			$\mathbb{R}$-linear in $X$,  $\iota_X$ acts on  $\Omega$ as an
			antiderivation of degree -1,  $\iota_x = 0$, and  $\mathcal{L}_X$
			acts as a derivation of degree 0.
	\end{itemize}
	Furthermore, the operators satisfy Cartan's homotopy formula:
	$\mathcal{L}_X= d\iota_X+\iota_Xd$.
\end{defn}
Note that commutativity for a graded algebra means that if $a\in
\Omega^k,b\in\Omega^l$ then $ba = (-1)^{kl}ab$.  

\begin{defn}
	A differential form $\alpha\in\Omega$ is 
	\underline{horizontal} if $\iota_X\alpha=0$ for all $X\in\mathfrak{g}$. 
	It is \underline{invariant} if $\mathcal{L}_X\alpha = 0$ for all $X\in
	\mathfrak{g}$. It is \underline{basic} if it is both horizontal and
	invariant.
\end{defn}
Note that in the case of a principal bundle, we think of $X\in \mathfrak{g}$ as 
a vertical vector field via $X_p = \odv{}{t}_{t=0} p \exp (t X)$. 
\begin{comment} %%% stuff about principal and associated bundles
\begin{defn}
	A differential form $\alpha\in\Omega(P,V)$ on a principal $G$-bundle  with 
	representation $(V,\rho)$ is \underline{$\rho$-equivariant}   
	if for every $g\in G$, $r_g^*\alpha = \rho(g^{-1})\alpha$. 
	\\
	A differential form $\alpha\in\Omega(P,V)$ on a principal $G$-bundle  with 
	representation $(V,\rho)$ is \underline{basic}  
	if it is horizontal and $\rho$-equivariant. This subspace is
	denoted $\Omega_{bas}(P,V)$.
\end{defn}
Let $P$ be a principle $G$-bundle, and let ($V,\rho$) be a representation of
$G$. Let $E= P\times_\rho V$ be the associated bundle.
\begin{thm} %thm 31.9 tu
	The map $\Omega_{bas}^k(P,V) \to \Omega^k(M,P\times_\rho V)$ given by 
	$\omega \mapsto \alpha_x = f_p \circ \omega_p$
	is a linear isomorphism, where $f_p: V\to E_x$ is the isomorphism  $v\mapsto
	[p,v]$, and $p\in\pi^{-1}(x)$ is any point.
\end{thm}

\begin{cor} % Prop 1.7 BGV
	There is a natural isomorphism between $\Gamma(M,P\times_\rho V)$ and
	$\rho$-equivariant maps in
	$C^{\infty}(P,V)$, given by sending $s\in C^{\infty}(P,V)^G$ to $s_M$
	defined by  $s_M(x) = [p,s(p)]$, where  $p\in\pi^{-1}(x)$ is any element.
\end{cor}


Recall that a \underline{vertical vector} on a fibre bundle $E$ with base  $M$
is a tangent vector  $X\in TE$ such that  $X(\pi^* f) = 0$ for any  $f\in
C^\infty(M)$. The space $V_pP$ of vertical tangent vectors to a point $p$ in a
principal $G$-bundle can be canonically identified with the Lie algebra $\mathfrak{g}$ 
of $G$ in the following way. 
If $X\in\mathfrak{g}$, the \underline{fundamental vector field} associated to
$X$, denoted $X_P \in \Gamma(VP)$, is
 \[
X_P = \odv{}{t}_{t=0} p \exp (t X)
\] 
This is a vertical vector field because $X_P(\pi^*f) = \odv{}{t}_{t=0} f(p \exp
(t X)) = 0$. 
\end{comment}


\begin{prop} % prop 18.10 tu
	If $\Omega$ is a  $\mathfrak{g}$-differential graded algebra, the vector
	space of basic elements $\Omega_{bas}$ is a $\mathfrak{g}$-differential
	graded algebra.
\end{prop}
% TODO tensor product of dgas
Tensor prouct of dgas is a dga

\begin{thm}[Equivariant de Rham theorem] % 19.4 Tu
	For a compact connected Lie group $G$ with Lie algebra $\mathfrak{g}$, and
	$G$-manifold  $M$, there is a graded-algebra isomorphism 
	 \[
		 H_G^*(M) \simeq H^*((W(\mathfrak{g})\otimes \Omega(M))_{bas}, \delta)
	\] 
\end{thm}
The complex $(W(\mathfrak{g})\otimes \Omega(M))_{bas}$ with the Weil
differential is called the \underline{Weil model}.

\begin{thm}[Weil-Cartan isomorphism] % Thm 21.1 Tu
	Let $G$ be a connected Lie group, and $M$ be a left $G$-manifold. There is a
	graded-algebra isomorphism 
	\[
		F : (W(\mathfrak{g})\otimes \Omega(M))_{hor} \to S(g^*\otimes \Omega(M))
	\] 
	\[
	a+ \sum \theta_I a_I \mapsto a
	\] 
	which induces a graded-algebra isomorphism on the basic algebras
	$F : F : (W(\mathfrak{g})\otimes \Omega(M))_{bas} \to (S(g^*\otimes
	\Omega(M)))^G$. 
\end{thm}
The complex $(S(g^*\otimes \Omega(M)))^G$ is called the \underline{Cartan
model}. Elements of the Cartan model are called \underline{equivariant forms}.

\section{BRST Model}
In the context of topological field theories, another model of equivariant
cohomlogy arises naturally, called the BRST model. As a vector space, it is
identical to the Weil model $W(\mathfrak{g})\otimes \Omega(M)$, but with
differential $d_B = d_W + \theta^i\otimes \mathcal{L}_i - u^i \otimes \iota_i$.

% TODO how does it arise? proof of conjugagion theorem, why conjugate?
It was shown by Kalkman that the BRST and Weil models are related by the algebra
automorphism of conjugation by $\exp (\theta^i \iota_i)$. 


\section{Equivariant characteristic classes}
\begin{defn} % from ch1 BGV
	Let $\pi: E\to M$ be a fibre bundle and let $G$ be a Lie group. We say $E$
	is a \underline{$G$-equivariant bundle} if $E$ and  $M$ are 
	$G$-manifolds  with $g \cdot \pi = \pi \cdot g$ for all  $g\in G$.\\
	If  $E$ is a vector bundle, we further require that the action  $g: E_x \to
	E_{gx}$ is linear.
\end{defn}

\begin{prop} % prop 9.4 Tu
	Let $f:M\to N$ be a $G$-equivariant map of  $G$-spaces, and let
	$f_G:M_G\to N_G$ be defined by $[e,x]\mapsto [e,f(x)]$. 
	\begin{enumerate}[(i)]
	    \item If $f$ is injective, then $f_G$ is injective
		\item If $f$ is surjective, then $f_G$ is surjective
		\item If $M\xrightarrow{f} N$ is a fiber bundle with fiber $F$, then
			$M_G\xrightarrow{f_G}N_G$ is a fiber bundle with fiber  $F$. 	
	\end{enumerate}
\end{prop}

Let $G$ be a topological group, and  $\pi : E\to M$ be a  $G$-equivariant vector
bundle. By the proposition above, this induces a vector bundle  $\pi_G : E_G\to
M_G$ on homotopy quotients of the same rank. If $E\to M$ is oriented, then so is
$E_G \to M_G$. 

The \underline{equivariant Euler class} of an oriented equivariant vector bundle
$\pi : E \to M$ is defined to be the Euler class of $\pi_G : E_G \to M_G$, i.e.
it is an element of $H_G^*(M)$. 


