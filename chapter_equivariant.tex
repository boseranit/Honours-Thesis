\chapter{Equivariant Cohomology}
\label{chapter_equivariant}

\section{Preliminaries}
Equivariant differential topology extends the results of differential topology
to manifolds/topological spaces with a group action, called a $G$-space. A
$G$-space is a topological space $X$ with a continuous action $X\times G \to X$ 
such that $x\cdot e = x$ for $e$ the identity in  $G$ and $(x\cdot g) \cdot h =
x\cdot (gh)$ for $g,h\in G$. 

We will often be dealing with both left and right group actions, but these are
equivalent. If we have a right $G$ action on $X$, this can be turned into a left
action by defining  $g \cdot x = x \cdot g^{-1}$, so that $h\cdot (g\cdot x) =
(x\cdot g^{-1})\cdot h^{-1} = x\cdot (g^{-1}h^{-1}) = (hg)\cdot x$. The same
argument works in the other direction.


There are two main formulations of equivariant
cohomology: the Borel construction which uses classifying spaces, and the Cartan model.
These have been proven to be equivalent in the case of compact Lie groups 
by Cartan (Theorem \ref{thm:equivariant_de_Rham} and \ref{thm:weil_cartan_iso}). 

First let us recall some basic definitions in algebraic topology. All maps
considered are continuous.
\begin{defn}
	Let $(X,x_0)$ and $(Y,y_0)$ be based topological spaces. Two maps
	$f,g:(X,x_0)\to(Y,y_0)$ are \underline{homotopic} if there is a continuous
	map $F:X\times [0,1]\to Y$ such that
	$F(x,0)=f(x),F(x,1)=g(x),F(x_0,t)=y_0$. Then we denote $f\sim g$.

	A \underline{homotopy equivalence} is a continuous map $f:(X,x_0)\to(Y,y_0)$
	that has a homotopy inverse, i.e. a continuous map $g:(Y,y_0)\to(X,x_0)$
	such that $f\circ g \sim 1_Y$ and $g\circ f \sim 1_X$.
	Then we say $X$ and  $Y$ have the same homotopy type.

	A topological space $(X,x_0)$ is \underline{contractible} if it has the
	homotopy type of a point. 

	A map $f:X\to Y$ is a \underline{weak homotopy equivalence} if
	it induces an isomorphism of homotopy groups
	$f_*:\pi_q(X)\to\pi_q(Y)$ for all  $q\geq 0$. 
	A space $X$ is \underline{weakly contractible} if $\pi_q(X)=0$ for all  $q\geq 0$. 
\end{defn}

%TODO reference
\begin{thm}[Whitehead's theorem	{\cite[Thm 4.5]{hatcher}}] 
	If a continuous map $f:X\to Y$ of CW complexes 
	is a weak homotopy equivalence, then  $f$ is a homotopy equivalence. 
\end{thm}
In particular, this
means that a weakly contractible CW complex is contractible, using the inclusion
map $x_0 \to X$. 
% every smooth manifold is homotopy equivalent to a CW complex pg 18 Tu

\begin{thm}[{\cite[Prop 4.21]{hatcher}}] \label{thm:weak_to_cohomology}
	A weak homotopy equivalence $f:X\to Y$ induces isomorphisms
	$f^*:H^n(Y;R)\to H^n(X;R)$ in cohomology for all  $n$.
\end{thm}
\begin{defn}
	Given manifolds $F,E$ and $B$, 
	a \underline{fiber bundle} with fiber $F$ is a surjection  $\pi: E\to B$
	that is locally homeomorphic to a product $U\times F$; i.e. every  $b\in B$
	has a neighborhood  $U\subset B$ such that there is a fiber-preserving
	homeomorphism $\phi_U : \pi^{-1}(U) \to U\times F$.
\end{defn}
A useful tool for computing homotopy groups is the homotopy exact sequence of a
fiber bundle. 
\begin{thm}[{\cite[Thm 4.41]{hatcher}}] \label{thm:fiber_les}
	Suppose $(E,x_0) \xrightarrow{\pi} (B,b_0)$ is a fiber bundle with fiber
	$F=\pi^{-1}(b_0)$ and path-connected base space $B$. Let  $x_0$ also be the
	basepoint of $F$, and  $i:(F,x_0)\to (E,x_0)$ the inclusion map. Then there
	exists a long exact sequence
	\[
		\ldots\to\pi_n(F,x_0) \xrightarrow{i_*}\pi_n(E,x_0)
		\overset{\pi_*}{\xrightarrow{}}\pi_n(B,b_0)
		\to \pi_{n-1}(F,x_0)\to\ldots\to\pi_0(E,x_0)\to 0
	\] 
	All maps are group homomorphisms except the last three maps which are set
	maps.
\end{thm}

\begin{prop} \label{prop:total_base_iso}
	If $E \xrightarrow{\pi} B$ is a fiber bundle with weakly contractible fiber $F$,
	and path-connected base $B$, then $\pi$ is a weak homotopy equivalence.
	Furthermore if $B$ and $F$ are CW complexes, 
	then $\pi$ is a homotopy equivalence.  
\end{prop}
\begin{proof}
	By Theorem \ref{thm:fiber_les}, the sequence of induced maps
	\[
	\ldots \to \pi_n(F) \to \pi_n(E) \to\pi_n(B) \to
	\pi_{n-1}(F)\to \ldots
	\] 
	is exact. So the induced maps $\pi_* : \pi_n(E) \to
	\pi_n(B)$ on homotopy groups are isomorphisms. Hence $\pi$ is a weak
	homotopy equivalence. 

	Now suppose in addition that $B$ and $F$ are CW complexes. 
	Since each cell of $B$ is contractible, the fiber bundle
	restricted to a cell is trivial and is homeomorphic to  cell$\times F$. 
	Since  $F$ also has a CW structure, the product is also a CW complex, and
	gluing these pieces together shows that $E$ is a CW complex. Hence, by
	Whitehead's theorem,  $\pi$ is a homotopy equivalence. 
\end{proof}

Let $G$ be a topological group.
If  $P$ is a right  $G$-space and  $M$ is a left  $G$-space, the 
\underline{Cartan mixing space} of  $P$ and  $M$ is the quotient 
$P\times_G M:= (P\times M) / \sim$ by the 
equivalence relation $(p,m)\sim (pg,g^{-1}m) \textrm{ for all }g\in G$.
Equivalently, this is the orbit space $(P\times M) /G$ under the diagonal action
$g(p,m) = (pg,g^{-1}m)$.

If in addition $P\xrightarrow{\pi} B$ is a principal $G$-bundle, define the projection
$\tau_1:P\times_GM\to B$ by $\tau_1([p,m])=\pi(p)$. This is well defined because
$\pi$ preserves the fiber.

\begin{prop} \label{prop:cartan_mixing}% prop 4.5 Tu
	If $P\xrightarrow{\pi} B$ is a principal  $G$-bundle and  $M$ is a left  $G$-space,
	then  $\tau_1 : P\times_G M\to B$ is a fiber bundle with fiber  $M$.
\end{prop}
\begin{proof}
	Suppose $\pi^{-1}(U)\simeq U\times G$. It suffices to show
	$\tau_1^{-1}(U)\simeq U\times M$. 
	\begin{align*}
		\tau_1^{-1}(U)
		&= \{[p,m]\in P\times_GM \mid \pi(p)\in U\} \\
		&= \pi^{-1}(U)\times _G M \\
		&\simeq (U\times G) \times_G M 
	\end{align*}
	To show this is homeomorphic to $U\times M$, we define 
	$\varphi:(U\times G) \times_G M \to U\times M$ by 
	$[(x,g),m] \mapsto (x,gm)$. It has inverse $(x,m)\mapsto [(x,1),m]$.
\end{proof}

\section{Borel construction}
Given a $G$-space  $M$, the  aim of equivariant cohomology is to study the
cohomology of the quotient space by the group action. But if the $G$-action is not free,
the space $M/G$ doesn't capture information from non-trivial stabilisers. For
instance, if  $S^1$ acts on $S^2$ by rotation about the vertical axis, the
quotient is a segment which has trivial cohomology. Furthermore, if the action is not
free and we have a $G$-equivariant homotopy equivalence  $f:M\to N$, then  
the induced map $M /G \to N /G$ is not necessarily a homotopy
equivalence. As an example, consider $M=\mathbb{R}$ with $\mathbb{Z}$-action
given by translation, and a point $N=*$ with trivial $\mathbb{Z}$-action. Then
$\mathbb{R} /\mathbb{Z} \simeq S^1$ but $N / \mathbb{Z}$ is a point.

To address these limitations, the idea behind the Borel constuction is to force 
the action to be free by replacing  $M$ with  $E\times M$ where $E$ is a  
$G$-space with a free action, and then studying the quotient  $(E\times M) /G$. 
But we need an appropriate choice of $E$ such that the cohomology does not 
depend on it.

\begin{defn}
	Given a topological group $G$, let  $EG \to BG$ be a principal  $G$-bundle
	with weakly contractible total space  $EG$. Define the  \underline{homotopy
	quotient} of a $G$-space $M$ by $M_G:=EG\times_G M$.
	 
	The \underline{equivariant cohomology} of $M$ is defined to be the singular
	cohomology of the homotopy quotient: $H_G^*(M;R) := H^*(M_G;R)$.
\end{defn}
Of course, for this definition to make sense we need to show that it is
independent of the choice of weakly contractible $EG$ and find out when $EG$ 
exists, which we do in this section. 

The following result shows that when the action is free, equivariant cohomology
is the cohomology of the orbit space.
\begin{prop} \label{prop:free_action_cohomology}
	If $M$ has a free  $G$-action, then $H^*(M_G)\simeq H^*(M /G)$. 
\end{prop}
\begin{proof}
	Let $EG \to BG$ be a principal  $G$-bundle with weakly contractible total
	space. Since the $G$-action is free,  $M \to M /G$ is a principal bundle.
	By Proposition \ref{prop:cartan_mixing}, 
	$(EG\times M) /G \to M /G$ is a fiber bundle with
	fibre homeomorphic to $EG$. Then by Proposition \ref{prop:total_base_iso},
	the result follows.
\end{proof}

\begin{lem} % lemma 4.9 Tu
	If $E$ is a weakly contractible $G$-space, and $P\to P/G$ is a principal
	$G$-bundle, there is a weak homotopy equivalence $(E\times P) /G \to P /G$.  
\end{lem}
\begin{proof}
	Consider $E\times_G P = (E\times P) /G$ as the orbit space under 
	the diagonal action $(e,p)g = (g^{-1}e,pg)$. 
	By Proposition \ref{prop:cartan_mixing}, $(E\times P) /G \to P /G$ is a
	fiber bundle with fiber $E$. Then the result follows by Proposition
	\ref{prop:total_base_iso}.
\end{proof}

The next proposition shows that the definition of equivariant cohomology is
independent of the choice of  $E$. 
\begin{prop} % Tu Thm 4.10 doesn't work, not weak homotopy equivalence
	Suppose $M$ is a left  $G$-space. If $E\to B$ and  $E'\to B'$ are two principal
	$G$-bundles with weakly contractible total spaces, then $H^*(E\times_G M)
	\simeq H^*(E'\times_G M)$.
\end{prop}
\begin{proof}
	Since the $G$-action on  $E'\times M$ is also free,  $E'\times M \to
	(E'\times M) /G$ is a principal bundle. Then by the lemma above, 
	there is a weak homotopy equivalence 
	$(E\times E' \times M) /G \to (E'\times M) /G$. By 
	Theorem \ref{thm:weak_to_cohomology}, it
	induces isomorphisms in cohomology  $H^n((E\times E'\times M )/G) \simeq
	H^n((E'\times M) /G)$ for all  $n\geq 0$.

	By symmetry of $E$ and  $E'$, we also conclude  $H^n((E'\times E\times M )/G) \simeq
	H^n((E\times M) /G)$ for all  $n\geq 0$. The canonical homeomorphism
	$(E'\times E \times M) /G \to (E\times E'\times M) /G$ induces an
	isomorphism on cohomology, and thus $H^n((E\times M) /G) \simeq
	H^n((E'\times M) /G)$.
\end{proof}

The remaining question in our definition of equivariant cohomology of a
$G$-space $M$ is the existence of a weakly contractible principal $G$-bundle $EG$. 
It turns out that for CW complexes,  
a weakly contractible $G$-bundle is equivalent to a universal $G$-bundle, as
defined below.
\begin{defn}
	A principal $G$-bundle  $\pi:EG\to BG$ is a \underline{universal $G$-bundle} 
	if the following two conditions hold:
	\begin{enumerate}[(i)]
	    \item for any principal $G$-bundle  $P$ over a CW complex $X$, there
			exists a continuous map  $h:X\to BG$ such that  $P \simeq h^*EG$ 
		\item If $h_0,h_1:X\to BG$ and $h_0^*EG \simeq h_1^*EG$ over a CW
			complex $X$, then  $h_0$ and $h_1$ are homotopic
	\end{enumerate}
	The base space $BG$ of a universal  $G$-bundle is called a
	\underline{classifying space} for  $G$.
\end{defn}
\begin{thm}[Steenrod 1951] % thm 5.2 Tu % TODO: reference
	Let $E\to B$ be a principal  $G$-bundle. If  $E$ is weakly contractible,
	then  $E\to B$ is a universal bundle. Conversely, if  $E\to B$ is a
	universal bundle and  $B$ is a CW complex, then  $E$ is weakly contractible.
\end{thm}
\begin{thm}[Milnor's construction] % TODO reference
	A universal $G$-bundle exists for any topological group  $G$.
\end{thm}

\begin{comment}
Let $X$ be a CW complex, and $[X,B]$ be the homotopy classes of maps $h:X\to B$, 
and  $\mathcal{P}_G(X)$ be the isomorphism classes of principal $G$-bundles over $X$. 
Then the definition of universal $G$-bundle states the map
$\varphi:[X,BG]\to\mathcal{P}_G(X)$ given by $h\mapsto h^*(EG)$ is surjective
(condition (i)) and injective (condition (ii)). 
% well defined by Theorem 5.3 Tu
\end{comment}

\begin{thm} %thm 5.6
	If a CW classifying space exists for a topological group $G$, it is unique 
	up to homotopy equivalence.
\end{thm}
\begin{proof}
	Suppose $E\to B$ and  $E'\to B'$ are universal bundles, where $B$ and  $B'$
	are CW complexes. Since  $E'$ is
	universal there is a map  $f:B\to B'$ such that  $E'\simeq h^*E$. Similarly
	there is a map  $h:B'\to B$ such that  $E\simeq h^*E'$. Therefore,  $E\simeq
	f^*h^*E=(h\circ f)^*E$. 

	But this means  $(h\circ f)^*E = \id_B^*E$, so by condition (ii) of a
	universal bundle, $h\circ f \simeq \id_B$. Similarly, $f\circ h\simeq
	\id_{B'}$. Therefore $B$ and  $B'$ are homotopy equivalent.
\end{proof}



\section{Weil model}
\subsection{Results on principal bundles}
\begin{defn} % tu intro to manifolds 4.6
	An \underline{antiderivation} of a graded algebra $A=\bigoplus_{k=0}^\infty
	A^k$ is a linear map $D : A\to A$ such that 
	 \[
	D(ab) = (Da)b + (-1)^{\deg a} a(Db)
	\] 
	If there is an integer $m$ such that  $D A^k \subset A^{k+m}$ for all $k$,
	then we call it an antidervation of degree  $m$.
\end{defn}
\begin{defn} 
	Let  $\omega\in \Omega(P)$ be a differential form on a principal $G$-bundle
	$P\xrightarrow{\pi} M$. 
	Then $\omega$ is a \underline{basic form} if  $\omega = \pi^*\eta$
	where  $\eta\in\Omega(M)$. 
	It is \underline{$G$-invariant} if $L^*_g \omega =\omega$ for all  $g\in G$,
	where $L_g : P\to P$ is left multiplication by  $g$.
	It is \underline{horizontal} if at each $p\in P$,
    $\omega$ vanishes whenever one of its arguments is vertical:
	$\iota_Y\omega_p = 0$ for all  $Y\in V_p$.
\end{defn}
\noindent
The definition of $G$-invariant also extends to forms on a  $G$-manifold. 
We can also identify each $X\in \mathfrak{g}$ with 
a vertical vector field via $X_p = \odv{}{t}_{t=0} p \exp (t X)$. 
\begin{thm}[{\cite[Theorem 12.2]{equivariant_tu}}] % Thm 12.2 equivariant_tu
	\label{thm:lie_invariant}
	For a connected Lie group $G$, a form $\omega\in \Omega(M)$ on a 
	 $G$-manifold is  $G$-invariant if and only if  $\mathcal{L}_A\omega=0$ for
	 all $A\in\mathfrak{g}$ in the Lie algebra.
\end{thm}
\begin{defn} \label{def:contraction} % BGV
	If $V$ is a vector space, the \underline{interior multiplication} (or 
	contraction) operator $\iota(v) :
	\Lambda V^* \to \Lambda V^*$ for $v\in V$ is the unique antidervation of -1 
	such that $\iota(v)\alpha = \alpha(v)$ if  $\alpha\in V^*$.
\end{defn}
\noindent
In particular, we have interior multiplication with a vector
field $\iota(X) : \Omega^*(M) \to \Omega^{*-1}(M)$.
The definition similarly extends to vector valued forms.

\begin{thm}[{\cite[Theorem 12.5]{equivariant_tu}}] % Thm 12.5 equivariant_tu
	A form  $\omega\in\Omega^k(P)$ on a principal $G$-bundle is 
	basic if and only if it is $G$-invariant and horizontal.
\end{thm}

\begin{thm}[Cartan's homotopy formula {\cite[Theorem 20.10]{intro_tu}}] 
	\label{thm:cartan_homotopy} % thm 10.4 tu equivariant, p18 BGV
	Let $X\in \mathcal{M}$ be a vector field on a manifold $M$, and 
	$\alpha\in \Omega(M)$. Then
	\[
	\mathcal{L}(X) = d\iota(X)+\iota(X) d
	\] 
	\begin{comment}
	\begin{enumerate}[(i)]
	    \item $\mathcal{L}(X)d = d\mathcal{L}(X)$
		\item $\mathcal{L}(X)(\iota(Y)\alpha) = \iota([X,Y])\alpha +
			\iota(Y)(\mathcal{L}(X)\alpha)$ 
		\item Cartan's homotopy formula: $\mathcal{L}(X) = d
			\cdot\iota(X)+\iota(X)\cdot d$
	\end{enumerate}
	\end{comment}
\end{thm}

\begin{thm}[{\cite[Theorem 30.4]{loringtu}}] \label{thm:pb_curvature_properties}
	The curvature $F$ of a connection  $\omega$ on a principal  bundle $P\to M$
	satisfies the following properties:
	\begin{enumerate}[(i)]
	    \item (Horizontality) For $p\in P$ and  $X_p,Y_p\in T_pP$,  $F_p(X_p,Y_p) =
			(d\omega)_p(hX_p,hY_p)$ where $hX_p$ is the horizontal projection.
		Consequently, $F$ is horizontal, i.e. $\iota_X F = 0$ for any vertical
		 vector  $X\in T_pP$
		\item ($G$-equivariance) $R_g^*F = \Ad_{g^{-1}}\circ F$
	\end{enumerate}
\end{thm}

\subsection{Differential graded algebras}
We would like to introduce the concept of equivariant differential forms on a
$G$-manifold  $M$, which should play the role of differential forms on  $M_G$.
\begin{defn}
	Let $\mathfrak{g}$ be a Lie algebra. A \underline{$\mathfrak{g}$-differential graded
	algebra} is a graded-commutative algebra $\Omega= \bigoplus_{k\geq
	0}\Omega^k$ with 
	\begin{itemize}
		\item an antiderivation $d:\Omega\to\Omega$ of degree 1 such that
	$d\circ d = 0$
		\item two actions of $\mathfrak{g}$: $\iota:\mathfrak{g}\times\Omega\to\Omega$
			and  $\mathcal{L}:\mathfrak{g}\times\Omega\to\Omega$, where for
			$X\in\mathfrak{g}$,  $\iota_X$ and  $\mathcal{L}_X$ are
			$\mathbb{R}$-linear in $X$,  $\iota_X$ acts on  $\Omega$ as an
			antiderivation of degree -1,  $\iota_X^2 = 0$, and  $\mathcal{L}_X$
			acts as a derivation of degree 0.
	\end{itemize}
	Furthermore, the operators satisfy Cartan's homotopy formula:
	$\mathcal{L}_X= d\iota_X+\iota_Xd$.
\end{defn}
Note that graded-commutative algebra means that if $a\in
\Omega^k,b\in\Omega^l$ then $ba = (-1)^{kl}ab$. 
From Theorem \ref{thm:cartan_homotopy} in the previous section, it is
clear that differential forms $\Omega(N)$ on a $G$-manifold  $N$ is an example
of a $\mathfrak{g}$-dga. 

\begin{defn}
	An element $\alpha\in\Omega$ of a $\mathfrak{g}$-dga is 
	\underline{horizontal} if $\iota_X\alpha=0$ for all $X\in\mathfrak{g}$. 
	It is \underline{invariant} if $\mathcal{L}_X\alpha = 0$ for all $X\in
	\mathfrak{g}$. It is \underline{basic} if it is both horizontal and
	invariant.
\end{defn}

\begin{defn} 
	A \underline{morphism} $\phi : \Omega' \to \Omega$ of
	$\mathfrak{g}$-differential graded algebras is a graded algebra homomorphism that
	commutes with  $d,\iota_X$ and  $\mathcal{L}_X$ for all $X\in\mathfrak{g}$.
\end{defn}
It follows directly that 
morphism $\phi : \Omega'\to \Omega$ of $\mathfrak{g}$-differential graded
algebras maps basic elements to basic elements.

\begin{comment} %%% stuff about principal and associated bundles
\begin{defn}
	A differential form $\alpha\in\Omega(P,V)$ on a principal $G$-bundle  with 
	representation $(V,\rho)$ is \underline{$\rho$-equivariant}   
	if for every $g\in G$, $r_g^*\alpha = \rho(g^{-1})\alpha$. 
	\\
	A differential form $\alpha\in\Omega(P,V)$ on a principal $G$-bundle  with 
	representation $(V,\rho)$ is \underline{basic}  
	if it is horizontal and $\rho$-equivariant. This subspace is
	denoted $\Omega_{bas}(P,V)$.
\end{defn}
Let $P$ be a principle $G$-bundle, and let ($V,\rho$) be a representation of
$G$. Let $E= P\times_\rho V$ be the associated bundle.
\begin{thm} %thm 31.9 tu
	The map $\Omega_{bas}^k(P,V) \to \Omega^k(M,P\times_\rho V)$ given by 
	$\omega \mapsto \alpha_x = f_p \circ \omega_p$
	is a linear isomorphism, where $f_p: V\to E_x$ is the isomorphism  $v\mapsto
	[p,v]$, and $p\in\pi^{-1}(x)$ is any point.
\end{thm}

\begin{cor} % Prop 1.7 BGV
	There is a natural isomorphism between $\Gamma(M,P\times_\rho V)$ and
	$\rho$-equivariant maps in
	$C^{\infty}(P,V)$, given by sending $s\in C^{\infty}(P,V)^G$ to $s_M$
	defined by  $s_M(x) = [p,s(p)]$, where  $p\in\pi^{-1}(x)$ is any element.
\end{cor}


Recall that a \underline{vertical vector} on a fibre bundle $E$ with base  $M$
is a tangent vector  $X\in TE$ such that  $X(\pi^* f) = 0$ for any  $f\in
C^\infty(M)$. The space $V_pP$ of vertical tangent vectors to a point $p$ in a
principal $G$-bundle can be canonically identified with the Lie algebra $\mathfrak{g}$ 
of $G$ in the following way. 
If $X\in\mathfrak{g}$, the \underline{fundamental vector field} associated to
$X$, denoted $X_P \in \Gamma(VP)$, is
 \[
X_P = \odv{}{t}_{t=0} p \exp (t X)
\] 
This is a vertical vector field because $X_P(\pi^*f) = \odv{}{t}_{t=0} f(p \exp
(t X)) = 0$. 
\end{comment}


If $(A,d_A)$ and  $(B,d_B)$ are differential graded algebras, then their tensor product has
multiplication given by 
\begin{equation} \label{eq:tensor_mult}
	 (a\otimes B)(a'\otimes b') = (-1)^{(\deg b)(\deg a')}aa'\otimes bb'
\end{equation}
which respects the grading $(A\otimes B)^k = \bigoplus_{i+j= k} A^i\otimes B^j$.
The differential, interior product and Lie derivative on $A\otimes B$ are defined by 
\begin{align}
	d(a\otimes b) &= (d_A a)\otimes b + (-1)^{\deg a}a\otimes d_Bb \nonumber \\
	\iota_X(a\otimes b) &= (\iota_X a)\otimes b +(-1)^{\deg a}a\otimes \iota_Xb
	\label{eq:tensor_ops} \\
	\mathcal{L}_X(a\otimes b) &= (\mathcal{L}_X a)\otimes b +a\otimes\mathcal{L}_Xb
	\nonumber
\end{align}
 
\begin{prop}[Operations on DGAs]  \label{prop:dga_operations}
	\phantom{}	% prop 18.7, 18.10 tu
	\begin{enumerate}[(i), leftmargin=\parindent]
	    \item 
			If $(A,d_A)$ and  $(B,d_B)$ are $\mathfrak{g}$-differential graded 
			algebras, then $(A\otimes B,d)$ is a $\mathfrak{g}$-differential graded 
			algebra
		\item 
	If $\Omega$ is a  $\mathfrak{g}$-differential graded algebra, the vector
	subspace of basic elements $\Omega_{bas}$ is a $\mathfrak{g}$-differential
	graded algebra.
	\end{enumerate}
\end{prop}
\begin{proof}[\textbf{\textit{Proof}} (sketch)]
	(i) It is required to check that $d$ and $\iota$ are antiderivations,
	$\mathcal{L}_X$ is a derivation, $d\circ d = 0$,  $\iota_X^2=0$, and
	Cartan's homotopy formula. These are all straightforward calculations.\\
	(ii) It is required to show that if $\alpha$ and  $\beta$ are basic, then so
	are $\alpha+\beta$,  $\alpha\beta$,  $d\alpha$,  $\iota_X \alpha$ and
	$\mathcal{L}_X\alpha $. 

	For $\alpha+\beta$, it follows from linearity of  $\iota$ and
	$\mathcal{L}_X$. For $\alpha\beta$, it follows from the (anti)derivation
	property of $\iota_X$ and  $\mathcal{L}_X$. For $d\alpha$, we have 
	 \begin{align*}
		 \iota_X(d\alpha) &= (\mathcal{L}_X-d\iota_X)\alpha = 0 \tag{by Cartan's
		 homotopy formula} \\
			 \mathcal{L}_X(d\alpha) &= d\mathcal{L}_X \alpha = 0 
		\tag{again by Cartan's homotopy formula}
	 \end{align*} 
	The basicness of $\iota_X\alpha$ and  $\mathcal{L}_X\alpha$ also follow 
	from application of Cartan's homotopy formula.
\end{proof}
By contrast, the subalgebra of horizontal elements is not
$d$-invariant.

\subsection{Weil model}
Let $P\xrightarrow{\pi} M$ be a principal  $G$-bundle. With a choice of
connection  $\omega\in\Omega^1(P,\mathfrak{g})$ and associated curvature
$\Omega\in\Omega^2(P,\mathfrak{g})$,
the Weil homomorphism defines a homomorphism $\bigwedge(\mathfrak{g}^*)\otimes
S(\mathfrak{g}^*) \to \Omega(P)$. To construct this, define the linear map 
\[
	f_1 : \mathfrak{g}^* \to \Omega^1(P) \text{ given by }
	f_1(\alpha) = \alpha \circ \omega
\] 
which we can extend to a unique algebra homomorphism 
$\bigwedge(\mathfrak{g}^*)\to\Omega(P)$ given by $f_1(\beta_1\wedge\ldots\wedge \beta_k)
= f_1(\beta_1)\wedge\ldots\wedge f_1(\beta_k)$ for $\beta_i\in\mathfrak{g}^*$.
This is well defined because 1-forms anticommute, thus we can define an
alternating $k$-linear map and use the universal property of  $\bigwedge^k$.
Similarly, define the linear map 
\[
	f_2 : \mathfrak{g}^* \to \Omega^2(P) \text{ given by }
	f_2(\alpha) = \alpha \circ \Omega
\]
Again, we can extend $f_1$ to a unique algebra homomorphism 
$S(\mathfrak{g}^*)\to \Omega(P)$ given by
$f_2(\beta_1\cdots\beta^k)=f_2(\beta_1)\wedge\ldots\wedge f_2(\beta_k)$ for
$\beta_i\in\mathfrak{g}^*$. 

Combining $f_1$ and $f_2$ gives a bilinear map $f_1\times f_2 :
\bigwedge(\mathfrak{g}^*)\times S(\mathfrak{g}^*) \to \Omega(P)$ given by
$(\alpha,\beta)\mapsto f_1(\alpha)\wedge f_2(\beta)$, and hence a linear 
map on the tensor product
\[
w : \bigwedge(\mathfrak{g}^*)\otimes S(\mathfrak{g}^*) \to \Omega(P)
\] 
such that $w(\alpha\otimes \beta) = f_1(\alpha)\wedge f_2(\beta)$.
We call $w$ the \underline{Weil homomorphism} (also called characteristic
homomorphism), and 
$W(\mathfrak{g}) =\bigwedge(\mathfrak{g}^*)\otimes S(\mathfrak{g}^*)$
is called the \underline{Weil algebra}.
Moreover, $w$ is a graded-algebra homomorphism if we assign degree 1 to elements of
$\mathfrak{g}^*$ in $\bigwedge(\mathfrak{g}^*)$ and degree 2 to 
elements of $\mathfrak{g}^*$ in $S(\mathfrak{g}^*)$.

We want to make the Weil algebra into a $\mathfrak{g}$-differential graded
algebra, so we will define a differential operator. Let $X_1,\ldots,X_n$ be a basis
for $\mathfrak{g}$, with dual basis  $\alpha^1,\ldots,\alpha^n$. So the Weil
algebra is generated by 
\begin{equation} \label{eq:weil_basis}
	 \theta_i = \alpha^i \otimes 1, \quad 
	 u_i = 1\otimes \alpha^i \quad\in \bigwedge(\mathfrak{g}^*)\otimes S(\mathfrak{g}^*)
\end{equation}
We can write the connection $\omega\in\Omega^1(P,\mathfrak{g})$ as a linear
combination $\omega = \sum \omega^kX_k$ and the curvature as $\Omega=\sum \Omega^kX_k$,
where  $\omega^k$ and  $\Omega^k$ are  $\mathbb{R}$-valued forms. 
Then the Weil homomorphism acts by 
\begin{equation} \label{eq:weil_homo}
w(\theta_k) = \theta_k\circ \omega = \omega^k
\qquad 
w(u_k) = u_k\circ \Omega = \Omega^k
\end{equation}
We want to define a differential operator $\delta$ such that  $w\circ \delta =
d\circ w$, so that $w$ is a  $\mathfrak{g}$-dga homomorphism. By the 
structural equation $\Omega=d\omega + \frac{1}{2}[\omega,\omega]$ and 
Bianchi identity $d\Omega=[\Omega,\omega]$, we have
\[
d\omega^k = \Omega^k - \frac{1}{2}\sum c_{ij}^k \omega^i\wedge \omega^j,
\qquad 
d\Omega^k = \sum c_{ij}^k \Omega^i\wedge \omega^j
\] 
where $[X_i,X_j]=\sum c_{ij}^kX_k$ are the structure constants of the Lie
algebra. Hence, we must define $\delta$ by
\begin{equation} \label{eq:weil_differential}
\delta\theta_k = u_k - \frac{1}{2}\sum c_{ij}^k \theta_i\theta_j, \qquad
\delta u_k = \sum c_{ij}^k u_i\theta_j
\end{equation}
and extend $\delta$ to  $W(\mathfrak{g})$ as an antiderivation. 
Similarly, we will define interior multiplication. On $\Omega(P)$, we have
for $A\in\mathfrak{g}$
\[
	 \iota_A \omega^k = \omega^k(\underline{A}) = \alpha^k(A), \qquad 
	 \iota_A \Omega^k = 0
\] 
the first property follows from $\omega(\underline{A})=A$ for a connection form,
and the second property follows from $\Omega$ being a horizontal form (see Theorem
\ref{thm:pb_curvature_properties}). Hence in order for $w$ to preserve the
interior multiplication,  $\iota_A$ should be defined by
\begin{equation} \label{eq:weil_interior}
\iota_A \theta_k = \alpha^k(A), \qquad \iota_A u_k = 0
\end{equation}
and extend $\iota_A$ to  $W(\mathfrak{g})$ as an antiderivation. Finally, the
Lie derivative is defined by Cartan's homotopy formula. 
Applying equations (\ref{eq:weil_differential}) and (\ref{eq:weil_interior}), we
obtain 
\begin{equation} \label{eq:weil_lie}
\mathcal{L}_{X_j} \theta_k = \sum c_{ij}^k \theta_i , \qquad
\mathcal{L}_{X_j}u_k = \sum c_{ij}^k u_i
\end{equation}
Since  $w$ commutes with
 $\delta$ and  $\iota_A$, it will commute with  $\mathcal{L}_A$.
The final property to check for $W(\mathfrak{g})$ to be a  $\mathfrak{g}$-dga
is that  $\delta^2 = 0$. 
\begin{thm} % equivariant tu Thm 19.1
	On the Weil algebra $W(\mathfrak{g})$, $\delta^2=0$
\end{thm}

% TODO : p156 why operators don't depend on basis

\begin{thm}[Equivariant de Rham theorem] \label{thm:equivariant_de_Rham} % 19.4 Tu
	For a compact connected Lie group $G$ with Lie algebra $\mathfrak{g}$, and
	$G$-manifold  $M$, there is a graded-algebra isomorphism 
	 \[
		 H_G^*(M) \simeq H^*((W(\mathfrak{g})\otimes \Omega(M))_{bas}, \delta)
	\] 
\end{thm}
\noindent
The complex $(W(\mathfrak{g})\otimes \Omega(M))_{bas}$ with the Weil
differential is called the \underline{Weil model}. It is the basic subcomplex of
the tensor product of two $\mathfrak{g}$-dgas which is well defined by
Proposition \ref{prop:dga_operations}.


\section{Cartan model}
\subsection{Weil-Cartan isomorphism}
The Cartan model arises from algebraically solving for the horizontal forms of the
Weil model $W(\mathfrak{g})\otimes \Omega(M)$. 
We consider the same generating elements for $W(\mathfrak{g})$ as in equation 
(\ref{eq:weil_basis}). Then for a $\mathfrak{g}$-dga  $\mathcal{A}$, we can write 
\[
	W(\mathfrak{g})\otimes \mathcal{A} 
	= \bigwedge(\mathfrak{g}^*) \otimes S(\mathfrak{g}^*) \otimes \mathcal{A}
	= \bigwedge(\theta_1,\ldots,\theta_n) \otimes \mathcal{A}[u_1,\ldots,u_n]
\] 
Thus, an element $\alpha\in W(\mathfrak{g})\otimes \mathcal{A}$ can be written as 
\[
	\alpha = a + \sum \theta_{I}a_I, \qquad a_I \in \mathcal{A}[u_1,\ldots,u_n]
\] 
\begin{thm}[Weil-Cartan isomorphism] \label{thm:weil_cartan_iso} % Thm 21.1 Tu
	Let $G$ be a connected Lie group, and $\mathcal{A}$ be a $\mathfrak{g}$-dga. 
	There is a graded-algebra isomorphism 
	\[
		F : (W(\mathfrak{g})\otimes \mathcal{A})_{hor} 
		\to S(\mathfrak{g}^*)\otimes \mathcal{A} 
		\qquad a+ \sum \theta_I a_I \mapsto a
	\] 
	which induces a graded-algebra isomorphism on the basic subalgebras
	$F : (W(\mathfrak{g})\otimes \mathcal{A})_{bas} \to (S(\mathfrak{g}^*)\otimes
	\mathcal{A})^G$. 
\end{thm}
Typically, we are interested in $\mathcal{A}=\Omega(M)$ for a $G$-manifold  $M$.
The complex $\Omega_G(M):=(S(g^*)\otimes \Omega(M))^G$, called the 
\underline{Cartan model}, is the subalgebra of invariants elements, where the
Lie derivative acts the same way as in the Weil model. 
Elements of the Cartan model are called 
\underline{equivariant forms}.  
Note that the Cartan model is not a $\mathfrak{g}$-dga, but a description 
of the basic Weil subcomplex.
\begin{proof}[\textbf{\textit{Proof}} (sketch).]
	Firstly, we verify that $F$ commutes with $\mathcal{L}_X$ for all
	$X\in\mathfrak{g}$, so that the restriction to basic subalgebras is well defined. 
	It can be shown that $\mathcal{L}_i \theta_k = -\sum_j c_{ij}^k\theta_j$, so
	$(F\circ \mathcal{L}_i)(a+\sum \theta_Ia_I) = \mathcal{L}_i a =
	\mathcal{L}_i \circ F(a+\sum \theta_Ia_I)$.

	We will prove the theorem by constructing a projection $H$ onto the horizontal
	subalgebra of $W(\mathfrak{g})\otimes \mathcal{A}$.
	Denote  $E=S(\mathfrak{g}^*)\otimes \mathcal{A}$, more generally this will
	work for a graded algebra that admits a interior multiplication operator
	$\iota_X$. Let $\theta_1,\ldots,\theta_n$ be a basis for $\mathfrak{g}^*$ in
	$\bigwedge (\mathfrak{g}^*)$. Define 
	 \[
		 H_i := 1 - \theta_i \iota_i : \bigwedge(\mathfrak{g}^*)\otimes E \to 
		\bigwedge(\mathfrak{g}^*)\otimes E, \qquad
		H := \prod H_i
	\] 
	where $\iota_i$ acts by the diagonal
	action on  $\bigwedge(\mathfrak{g}^*)\otimes E$ (see equation
	(\ref{eq:tensor_ops})). Let 
	\[
	J := \bigcap_i \ker \iota_i = (\bigwedge(\mathfrak{g}^*)\otimes E)_{hor}
	\] 
	We can show $H|_E : E \to J$ is a graded algebra isomorphism, and in fact
	the horizontal projection of the Weil model, by the following steps: 
	\begin{enumerate}[(1)]
	    \item $H_i$ is a ring map, and hence  $H$ is a ring map
		\item $H_iH_j = H_jH_i$ 
		\item $\iota_i H_i = 0$, hence  $\Im H_i \subset \ker \iota_i$ 
			and $\Im H \subset J$
		\item $H|_J = 1_J$, thus $\Im H = J$ 
		\item $H_i(\theta_i) = 0$, thus $H$ is supported on $E$, and $H|_E$ is surjective
		\item $H|_E$ is injective
	\end{enumerate}
	The calculations above are all straightforward. The proof is concluded by
	noting that $F$ is a left inverse for  $H$, and hence the unique
	inverse for $H$.
\end{proof}

\begin{remark} \label{remark:cartan_not_horizontal}
	A potential point of confusion is that $H:W(\mathfrak{g})\to W(\mathfrak{g})$ 
	is the projection on to the horizontal component, so in this case 
	$S(\mathfrak{g}^*) = W(\mathfrak{g})_{hor}$. However as an operator on
	$W(\mathfrak{g})\otimes \mathcal{A}$, this is no longer true because
	elements in $S(\mathfrak{g}^*)\otimes \mathcal{A}$ are no longer horizontal.
	This means that $E \neq J$ in the proof, and neither is contained within the other. 
	Consequently, the Cartan model  $(S(\mathfrak{g}^*)\otimes \mathcal{A})^G$
	consists of invariant but not necessarily horizontal elements, but is
	isomorphic to the algebra of basic forms.
\end{remark}

\subsection{Cartan differential}
The Weil-Cartan isomorphism carries the Weil differential $\delta_W$ to a
differential $\delta_C$ on the Cartan model, i.e. defined by
the commutative diagram 
% https://q.uiver.app/#q=WzAsNCxbMCwwLCIoVyhcXG1hdGhmcmFre2d9KVxcb3RpbWVzXFxPbWVnYShNKSlfe1xcdGV4dHtiYXN9fSJdLFswLDEsIihXKFxcbWF0aGZyYWt7Z30pXFxvdGltZXNcXE9tZWdhKE0pKV97XFx0ZXh0e2Jhc319Il0sWzEsMCwiKFMoXFxtYXRoZnJha3tnfV4qKVxcb3RpbWVzXFxPbWVnYShNKSleRyJdLFsxLDEsIihTKFxcbWF0aGZyYWt7Z31eKilcXG90aW1lc1xcT21lZ2EoTSkpXkciXSxbMCwxLCJcXGRlbHRhX1ciLDJdLFsyLDMsIlxcZGVsdGFfQyIsMix7InN0eWxlIjp7ImJvZHkiOnsibmFtZSI6ImRhc2hlZCJ9fX1dLFsyLDAsIkgiLDJdLFsxLDMsIkYiXV0=
\[\begin{tikzcd}
		{(W(\mathfrak{g})\otimes\Omega(M))_{\text{bas}}} &
		{(S(\mathfrak{g}^*)\otimes\Omega(M))^G} \\
			{(W(\mathfrak{g})\otimes\Omega(M))_{\text{bas}}} &
			{(S(\mathfrak{g}^*)\otimes\Omega(M))^G}
				\arrow["{\delta_W}"', from=1-1, to=2-1]
					\arrow["{\delta_C}"', dashed, from=1-2, to=2-2]
						\arrow["H"', from=1-2, to=1-1]
							\arrow["F", from=2-1, to=2-2]
\end{tikzcd}\]
This ensures that the Cartan model is isomorphic to the basic subcomplex as a
differential graded algebra, so becomes another model of equivariant
cohomology. As expected, we have $\delta_C F = F\delta_W H F = F\delta_W$.
To find an explicit description of $\delta_C$, let 
$a\in (S(\mathfrak{g}^*)\otimes\Omega(M))^G$. We have
\[
H(a) = \paren{\prod(1-\theta_i \iota_i)} a
= a - \sum_i \theta_i \iota_i a + \sum_{i,j} (\theta_i\iota_i)(\theta_j\iota_j)a
- (\cdots)
\] 
where $(\cdots)$ are terms which contain some $\theta_i$ as a factor. 
Then 
\[
\delta_W H(a) = \delta_W a - \sum_{i} \paren{u_i - \frac{1}{2}\sum
c_{kl}^i\theta_j\theta_l}\iota_i a + (\cdots)
\] 
where we have used the antiderivation property of $\delta_W$ and the definition
of  $\delta_W\theta_i$.  If we write $a= \sum u^I \eta_I$ where  $\eta_I \in
\Omega(M)$, then 
 \[
\delta_W a = \sum (\delta_W u^I)\eta_I + \sum u^I d\eta_I
\] 
Note that $\deg u^I$ is always even, and all terms in $\delta_W u^I$ contain a
factor of some $\theta_i$. 
Recall that  $F$ simply drops all terms containing $\theta_i$, so it leaves
\begin{equation*}
	\delta_C a := F\delta H(a) = \sum u^I d\eta_I - \sum_i u_i \iota_i a
\end{equation*} 
If we define $d$ on the Cartan model by  $d \paren{\sum u^I \eta_I} = \sum u^I
d\eta_I$, then
\begin{equation}
	\delta_C a = \paren{d - \sum u^i \iota_i} a
\end{equation}
Note that $\delta_W u_i = \sum c_{kl}^iu_k\theta_l$, while 
$\delta_C u_i = 0$ in the Cartan model. 

It may be unsatisfying to give a definition which is based on the choice of a
basis for $\mathfrak{g}$, so we now give an intrinsic, basis-free description.
An element $\alpha = \sum u_I \alpha_I \in S(\mathfrak{g}^*) \otimes
\Omega(M)$, where  $\alpha_I \in \Omega(N)$, can be interpreted as a function
$\mathfrak{g} \to \Omega(M)$ given by 
\[
	\alpha(X) = \sum u_{i_1}(X)\cdots u_{i_n}(X) \alpha_I 
\] 
where $I$ is a multi-index of arbitrary length allowing for repeated indices.
Thus, $(S(\mathfrak{g}^*)\otimes \Omega(M))^G$ are $\Omega(M)$-valued
polymomials on  $\mathfrak{g}$. 
\begin{thm} \label{thm:cartan_diff} % thm 21.4
	The Cartan differential 
	$\delta_C : (S(\g^*)\otimes \Omega(M))^G \to (S(\g^*)\otimes \Omega(M))^G$
	is given by
	\[
		(\delta_C \alpha)(X) = (d - \iota_X)(\alpha(X)), \quad 
		\text{for } X\in\mathfrak{g}.
	\] 
\end{thm}
\begin{proof}
	Since both sides are linear, it suffices to assume $\alpha = u^I\beta$ where
	$\beta\in\Omega(M)$. Then 
	 \[
	(\delta_C \alpha)(X) = u_I(X) d\beta - \sum_i u_i(X)\iota_i u^I(X)\beta
	\] 
	In the second term of this expression, the operator can be written 
	\[
	\sum_i u_i(X)\iota_{X_i} =  \iota_{\sum u_i(X)X_i } = \iota_X
	\] 
	using linearity of $\iota$, and that  $u_i$ is the dual
	basis, completing the proof.
\end{proof}

\subsection{Cartan map}
The notation for the Cartan model alludes to a natural action of  $G$
on the complex $W(\mathfrak{g})$. The action is induced by the
coadjoint representation $\Ad^* :G \to \operatorname{Aut}(\mathfrak{g}^*)$ of  
$G$ on  $\mathfrak{g}^*$. It is the unique action
such that $\gen[*]{\Ad_g^* \alpha, \Ad_g X} = \gen{\alpha, X}$ for
$\alpha\in\mathfrak{g}^*,X\in\mathfrak{g},g\in G$ where $\gen{\alpha,X}=\alpha(X)$ 
denotes the value of the linear functional, given by
\begin{equation} \label{eq:group_coadjoint_action}
	\gen[*]{\Ad^*_g \alpha, Y} := \gen{\alpha,\Ad_{g^{-1}} Y}
\end{equation}
The induced representation $\ad^* : \mathfrak{g} \to
\operatorname{End}(\mathfrak{g}^*)$ on the Lie algebra is found by
differentiating $\Ad^*$ as usual, which gives
\begin{equation} \label{eq:coadjoint_action}
	 \gen{\ad_X^* \alpha, Y} = \gen{\alpha, -\ad_X Y} = -\gen{\alpha,[X,Y]}
\end{equation}
This is precisely the same as the Lie derivative $\mathcal{L}_X$ action on 
the Weil model. To see this, the action in the basis
$X_1,\ldots,X_n$ of $\mathfrak{g}$ with dual basis
$\alpha_1,\ldots,\alpha_k$ of $\mathfrak{g}^*$ is
\begin{align*}
	\gen[*]{\ad_{X_i}^*\alpha_k,Y} 
	&= \gen[1]{\ad_{X_i}^*\alpha_k,\sum_j b_j X_j}  \\
	&= -\gen[1]{\alpha_k, \sum_jb_j\bracket{X_i,X_j}} \\
	&= -\gen[1]{\alpha_k, \sum_{j,l}b_jc_{ij}^lX_l} \\
	&= -\sum_{j}b_jc_{ij}^k 
	= -\sum_{j} c_{ij}^k \alpha^j(Y) 
\end{align*}
Comparing with equation ($\ref{eq:weil_lie}$), this proves the claim, since the
action of $\mathcal{L}_{X_i}$ is the same on both the even and odd generators. 
In this view, the Lie derivative on $W(\mathfrak{g})$ is the coadjoint representation of 
$\mathfrak{g}$ on $\mathfrak{g}^*$. This means that $\alpha \in
W(\mathfrak{g})$ is invariant, i.e. satisfies $\mathcal{L}_X\alpha = 0$ for all
$X\in\mathfrak{g}$ if and only if  $\Ad_g^* \alpha = \alpha$ for all  $g\in G$, 
assuming $G$ is connected. This can be proved by writing any $g\in G$ 
in the form  $A=e^{Y_1}e^{Y_2}\cdots e^{Y_m}$ for some
$Y_i\in\mathfrak{g}$.\cite[Corollary 3.47]{hall}

On the other hand, we know from Theorem \ref{thm:lie_invariant} that 
on a $G$-manifold  $M$,  $\mathcal{L}_X \omega = 0$ if and only if $R_g^*\omega
= \omega$. Therefore, elements in $W(\mathfrak{g})\otimes \Omega(M)$ are
invariant if and only if they are invariant under the product group action
$(\alpha\otimes \omega) \cdot g = \Ad_g^*\alpha \otimes R_g^*\omega$.

Viewed as polymomials, $S(\mathfrak{g}^*)^G$ is the subalgebra of 
$\Ad(G)$-invariant polynomials.
From Theorem $\ref{thm:weil_cartan_iso}$, we have
$W(\mathfrak{g})_{bas} = S(\mathfrak{g}^*)^G$. Thus, a Weil homomorphism 
$w: W(\mathfrak{g}) \to \Omega(P)$ induced by a connection on a principal bundle 
$P\to M$ descends to 
the Chern-Weil homomorphism on the basic subcomplexes
$w : S(\mathfrak{g}^*)^G \to \Omega(P)_{bas}$, given by  
\[
\alpha=\sum a_I u_I \mapsto \sum a_I \Omega^{i_1}\wedge \cdots\wedge\Omega^{i_k}
\] 
Denote the image of the Weil map by $\alpha(\Omega)$, which indicates ``evaluating 
the polynomial at the curvature". 
We wish to generalise the Chern-Weil map to 
$S(\mathfrak{g}^*)\otimes \Omega(P)\to \Omega(P)$, by defining $\alpha\otimes \eta \mapsto
\alpha(\Omega)\wedge \eta$.  However, the problem is that elements in the Cartan model
are invariant but not horizontal as mentioned in Remark
\ref{remark:cartan_not_horizontal}, and hence $\alpha(\Omega) \wedge \eta$ will be
invariant but not necessarily horizontal. We can remedy this by 
projecting onto the horizontal subspace of
$\operatorname{Hor}_{\omega}:\Omega(P)\to \Omega(P)_{hor}$. Thus, we have
motivated the Cartan map  
\begin{equation}
	\operatorname{Car}_{\omega} : (S(\mathfrak{g}^*)\otimes\Omega(P))^G\to \Omega(P)_{bas},
	\qquad \alpha \mapsto \operatorname{Hor}_{\omega}(\alpha(\Omega))
\end{equation}
As Proposition \ref{prop:free_action_cohomology} shows, the equivariant 
cohomology of a $G$-manifold  $P$ with a
free action is isomorphic to the singular cohomology of the quotient
$H_G(P)\simeq H^*(P /G)$. The Cartan map can be used to establish an algebraic
counterpart of this same result. 
% naber p113-114 % Thm 5.17 meinrenken
\begin{thm}[Cartan] % Theorem 5.2 meinrenken
	Suppose $P\xrightarrow{\pi}M$ is a principal $G$-bundle with a connection, 
	and $\operatorname{Car}_{\omega} : (S(\mathfrak{g}^*) \otimes \Omega(P))^G 
	\to \Omega(P)_{bas}$ is the associated  Cartan map. Then the Cartan map 
	is a quasi-isomorphism, i.e. a chain map that
	induces isomorphisms on all cohomology groups
	$H_G^*(P)\simeq H^*(\Omega(P)_{bas}) \simeq H^*(M)$.
	\begin{comment}
	% https://q.uiver.app/#q=WzAsNSxbMCwwLCJXKFxcbWF0aGZyYWt7Z30pXFxvdGltZXNcXE9tZWdhKFApIl0sWzEsMCwiXFxPbWVnYShQKSJdLFswLDEsIihXKFxcbWF0aGZyYWt7Z30pXFxvdGltZXNcXE9tZWdhKFApKV97XFx0ZXh0e2Jhc319Il0sWzEsMSwiXFxPbWVnYShQKV97XFx0ZXh0e2Jhc319Il0sWzIsMSwiXFxzaW1lcVxcT21lZ2EoTSkiXSxbMCwyXSxbMSwzXSxbMCwxLCJ3Il0sWzIsMywidyJdXQ==
	\[\begin{tikzcd}[column sep = 3.5em]
			{W(\mathfrak{g})\otimes\Omega(P)} & {\Omega(P)} \\
				{(S(\mathfrak{g}^*)\otimes\Omega(P))^G} &
				{\Omega(P)_{\text{bas}}} &[-4.1em] {\simeq\Omega(M)}
					\arrow[from=1-1, to=2-1]
						\arrow[from=1-2, to=2-2]
							\arrow["w", from=1-1, to=1-2]
							\arrow["\operatorname{Hor}\circ w", from=2-1, to=2-2]
			\end{tikzcd}\]
	\end{comment}
\end{thm}
A proof is given in chapter 5 of \citet{guillemin}, which constructs an explicit
homotopy operator to show that the inclusion $i(\eta) \mapsto 1\otimes \eta$ is
the homotopy inverse.
The Cartan map will turn out be very useful in the next chapter for the
Mathai-Quillen construction.

% Meinrenken shows that it is the same map as H then weil homo


\begin{comment}
	
\section{BRST Model}
In the context of topological field theories, another model of equivariant
cohomlogy arises naturally, called the BRST model. As a vector space, it is
identical to the Weil model $W(\mathfrak{g})\otimes \Omega(M)$, but with
differential $d_B = d_W + \theta^i\otimes \mathcal{L}_i - u^i \otimes \iota_i$.

% TODO how does it arise? proof of conjugation theorem, why conjugate?
It was shown by Kalkman that the BRST and Weil models are related by the algebra
automorphism of conjugation by $\exp (\theta^i \iota_i)$. 

\end{comment}

\begin{comment} % don't need equivariant version
\section{Equivariant characteristic classes}
\begin{defn} % from ch1 BGV
	Let $\pi: E\to M$ be a fibre bundle and let $G$ be a Lie group. We say $E$
	is a \underline{$G$-equivariant bundle} if $E$ and  $M$ are 
	$G$-manifolds  with $g \cdot \pi = \pi \cdot g$ for all  $g\in G$.\\
	If  $E$ is a vector bundle, we further require that the action  $g: E_x \to
	E_{gx}$ is linear.
\end{defn}

\begin{prop} % prop 9.4 Tu
	Let $f:M\to N$ be a $G$-equivariant map of  $G$-spaces, and let
	$f_G:M_G\to N_G$ be defined by $[e,x]\mapsto [e,f(x)]$. 
	\begin{enumerate}[(i)]
	    \item If $f$ is injective, then $f_G$ is injective
		\item If $f$ is surjective, then $f_G$ is surjective
		\item If $M\xrightarrow{f} N$ is a fiber bundle with fiber $F$, then
			$M_G\xrightarrow{f_G}N_G$ is a fiber bundle with fiber  $F$. 	
	\end{enumerate}
\end{prop}

Let $G$ be a topological group, and  $\pi : E\to M$ be a  $G$-equivariant vector
bundle. By the proposition above, this induces a vector bundle  $\pi_G : E_G\to
M_G$ on homotopy quotients of the same rank. If $E\to M$ is oriented, then so is
$E_G \to M_G$. 

The \underline{equivariant Euler class} of an oriented equivariant vector bundle
$\pi : E \to M$ is defined to be the Euler class of $\pi_G : E_G \to M_G$, i.e.
it is an element of $H_G^*(M)$. 
\end{comment}

\vspace{5mm}
\hrule 
\vspace{5mm}

\textbf{Bibliographical notes}
{\small
\begin{itemize}
	\item An excellent book on equivariant cohomology which this chapter is
	largely based on is \citetitle{equivariant_tu} by
	\citet{equivariant_tu}. The writing is more pedagogical. 
	\item A more comprehensive treatment of equivariant cohomology is given in
	\citet{guillemin}, which covers topics not included in Tu's book such as 
	the Cartan map and the Mathai-Quillen construction. 
\end{itemize}
}

