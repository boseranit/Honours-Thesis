
\chapter{Donaldson Invariants}
\label{chapter1}
Our goal in this chapter is to introduce the Donaldson polynomial invariants of
smooth 4-manifolds. We will not be concerned with proofs of the results needed
to construct them, as that would be beyond the scope of the thesis. 
(something about the later chapters are proof based, this chapter is mainly to
revisit in chapter 4)

\section{Space of connections}
\begin{prop}
	Let $E_1,\ldots,E_k$ and $F$ be vector bundles over a manifold  $M$. 
	There is a bijection
	\[
	\set{
		\begin{array}{c}
			C^\infty\text{-multilinear maps} \\
			T:\Gamma(E_1)\times \cdots \times \Gamma(E_k) \to \Gamma(F)
		\end{array}
	} \longleftrightarrow
	\set{
		\begin{array}{c}
			\text{sections}\\
			  \Gamma(M,E_1^*\otimes \cdots \otimes E_k^*\otimes F)
		
		\end{array}
	}
	\] 
\end{prop}
\begin{prop}
	If $\nabla^A,\nabla^B : \mathcal{X}(M) \times \Gamma(E) \to \Gamma(E)$ are
	two connections on a vector bundle $E\to M$, then 
	 \[
	\nabla^A - \nabla^B \in \Omega^1(X,\End E)
	\] 
	Conversely, given $a\in \Omega^1(X,\End E) \simeq \Hom(T^*M\otimes E,E)$, 
	then $\nabla^A+a$ interpreted as 
	\[
		(\nabla^A + a)_X(s) = \nabla^A_X s + a(X\otimes s)
	\] 
	is again a connection on  $E$. 
\end{prop}
In one direction, it is easy to show $\nabla^B-\nabla^B$ is tensorial (meaning
$C^\infty(M)$ linear in both components), and apply the previous proposition. In the other
direction, we only need to verify the Leibniz rule. Thus, a choice of connection
on $E$, defines a bijection between the space of all connections and
$\Omega^1(M,\End E)$. There is a similar result for principal bundles.

\begin{prop} \label{prop:connection_space}% Lemma 2.9.2 MF
	The space of connections $\mathcal{A}$ on a principal $G$-bundle $P\to M$ is in bijection 
	with the vector space $\Omega^1(M,\ad P) = \Omega^1(M,P\times_{\Ad} \mathfrak{g})$.
\end{prop}
This follows from the fact that the difference of two connections
$\theta_1-\theta_2\in \Omega^1(P,\g)$ is a horizontal and $\Ad$-equivariant form.
Finally we identify this space as $\Omega^1(M,P\times_{\Ad}\g)$, by 
the isomorphism between these spaces (see \cite[Theorem 31.9]{loringtu}). 
Conversely, the sum of $\theta_1$ with 
a horizontal and $\Ad$-equivariant form is again a connection on  $P$.

% p33 DK
Given a real or complex vector bundle $E$ of rank $n$, we can always construct 
its frame bundle with structure group $\GL(n)$ (or $\GL(n,\mathbb{C})$).
Additional algebraic structure on $E$ yields a principal bundle with smaller
structure group. For example,
if $E$ is a complex vector bundle with a Hermitian metric, we get a principal
$\U(n)$-bundle of orthonormal frames in  $E$. In these cases, we can identify
the space of connections on $E$ as $\Omega^1(M,\mathfrak{g}_E)$, where
$\mathfrak{g}_E \subset \End E$ is the bundle, 
where the restriction to $\g_E$ ensures that the new connection is
compatible with the structure group of  $E$.
If the structure group is $\SU(2)$, for example, then
$\g_E$ consists of skew-adjoint, trace-free endomorphisms of the rank two vector
bundle $E$. 

\begin{defn}
	The \underline{gauge group} $\mathcal{G}$ of a vector bundle $E\to M$ is the group of all
	vector bundle automorphisms $\Aut(E)$. Similarly, the gauge group of a
	principal bundle $P\to M$ is the group of principal bundle automorphisms
	$\Aut(P)$.
\end{defn}

\begin{prop} \label{prop:gauge_trans_space} % Lemma 4.1.2 Morgan
	The group of gauge transformations $\mathcal{G}=\Aut(P)$ is isomorphic to 
	sections $\Omega_{\Ad}^0(P,G)\simeq \Omega^0(M,\Ad P)$ under fiber-wise 
	multiplication 
\end{prop}
\begin{proof}
	A gauge transformation $f : P \to P$ preserves the fibers of  $P$ and
	satisfies  $f(p\cdot g) = f(p) \cdot g$. For a fixed $p\in P$, we have 
	$f(p) = p\cdot \psi(p)$ for some $\psi(p)\in G$ because the action is
	transitive. Then $f$ acts by multiplication 
	by $\psi(p)$ on the whole fiber. Hence, $f(p) =
	p\cdot \psi(p)$ for a smooth function  $\psi : P \to G$. 

	Substituting this into $f(p\cdot g) = f(p)\cdot g$, we find  $\psi(p\cdot g)
	= g^{-1}\psi(p) g$ since the action is free. So $R_g^*\psi =
	\Ad_{g^{-1}}\psi$. Hence, $\psi \in \Omega^0_{\Ad}(P,G) \simeq
	\Omega^0(M,P\times_{\Ad}G)$. The isomorphism can be proved in the same way
	as the associated vector bundle case.
	Finally, note that composition of automorphisms corresponds to
	multiplication of $\psi$ in the fibers of groups. 
\end{proof}

The purpose of Propositions \ref{prop:gauge_trans_space} and
\ref{prop:connection_space} is two-fold: this description allows us to identify
$\mathcal{G}$ as an infinite dimensional Lie group with Lie algebra identified
with the space of connections; the other purpose is to define Sobolev
completions of the spaces as in the next section.



The group $\Aut(P)$ is an infinte dimensional Lie group. 
There is a fiberwise exponential map $\exp : \Omega^0(M,\ad P) \to
\Omega^0(M,\Ad P)$ which assigns to any section  $\sigma \in \Gamma(\ad P)$ the
section  $s(x)=\exp(\sigma(x))$. To see that this gives a well defined map, we
use the fact that  $\exp : \g \to G$ satisfies
$\exp(g^{-1}Ag)=g^{-1}\exp(A)g$. The Lie bracket on $\Omega^0(M,\ad P)$ is
defined by the fiberwise bracket, which is well defined because ????
% TODO why well defined?

% TODO explain this 
Then $\mathcal{G}$ acts on $\mathcal{A}$ on the right by pullback 
$\omega\mapsto f^*\omega$.
The action of $f \in \Aut(P)$ on a connection  $\omega\in \Omega^1(P,\g)$ is
$f^*\omega \in \Omega^1(P,\g)$. 
% lemma 4.3.1 Morgan u^*`w is a connection one form
% The stabiliser of a connection A are the sections which are horizontal in A

\section{Quotient by Gauge group}
% 4.4 MF
We wish to form the quotient space $\mathcal{A} /\mathcal{G}$ 
of the space of connections by the action of
the group of gauge transformations $\Aut(P)$. By working with Sobolev completion
spaces, the quotient space can be shown to have a manifold structure by 
application of the slice theorem.

Let $E\to M$ be vector bundle with a metric connection over an oriented 
Riemannian manifold. 
Denote $\nabla : \Omega^k(M,E) \to \Omega^{k+1}(M,E)$ to be the exterior
covariant derivative. Using the metric on $TM$ and  $E$, this induces a
metric on  $(T^*M)^{\otimes k}\otimes E$. We denote the induced norm as 
$\norm{\cdot}_g$ below, but later just as $\abs{\cdot}$.
The metric $g$ on $M$ also gives rise to the Riemannian volume form $\odif{V}_g$
(see \cite[Prop 2.41]{riemannian_manifolds}).
\begin{defn}
    For $1 \leq p < \infty$, and $s \in \Omega^l(M,E)$, 
	define the \underline{$L^p$ norm}
	\[
		 \norm{s}_{L^p} = \paren{\int_M \norm{s(x)}^p_g \odif{V}_g}^{1 /p}
	\] 
	and the \underline{Sobolev norm}
	\begin{align*}
		\norm{s}_{L^p_k} 
		&= \paren{\sum_{j=0}^{k} \norm[*]{\nabla^j s}_{L^p}^p }^{1 /p} 
		= \paren{ \int_M \norm{s(x)}^p_g + \norm{\nabla s(x)}^p_g + \cdots
		+ \norm[*]{\nabla^k s(x)}^p_g \odif{V}_g}^{1 /p} 
	\end{align*}
\end{defn}
In this definition the Sobolev metric depends on the metric on $T^*M^{\otimes
l}\otimes E$ as well as the connection on $E$. It can be shown that the Sobolev
norm is up to equivalence, independent of the choices of metrics and connections
\cite[Lemma 11.22]{math_for_physics}.
\begin{defn}
	Let $1\leq p < \infty$ and  $k\in \mathbb{Z}_{\geq 0}$. The
	\underline{Sobolev
	space} $\Omega^l_{L^p_k}(M,E)$ of $L^p_k$-sections is the completion of
	$\Omega^l(M,E)$ in the Sobolev norm  $\norm{\cdot}_{L^p_k}$.
\end{defn}

We are often interested in Lie algebra valued differential forms 
$\Omega(P,\mathfrak{g})$ on a principal bundle , so we need to define a metric 
on both $P$ and $\g$. The metric on $T_pP$ can be obtained by identifying the 
horizontal subspace induced by the connection with  $T_{\pi(p)}M$, and the vertical
subspace with $\g$. Assuming $G$ is compact, we can always construct a metric on
 $\g$ as follows.
 \begin{thm} \label{thm:lie_inner_product}
	Let $G$ be a compact Lie group, with a representation  $\rho : G \to\GL(V)$.
	Then there exits an invariant inner product on  $V$, i.e. 
	$\gen{\rho(g)v,\rho(g)w} = \gen{v,w}$ for all $g\in G$. Consequently,
	$\rho$ is an orthogonal rep, and the induced rep $\rho_*$ on
	$\mathfrak{g}$ is skew-symmetric valued. 
\end{thm}
The construction is based on choosing an arbitrary inner product on $V$, and
defining $\gen{v,w} := \int_G \gen{\rho(A)w,\rho(A)w}_V \odif{vol}(A)$, where
$\odif{vol}(A)$ is a right-invariant differential form on $G$ (we can also
construct the integral based on the right-invariant Haar measure on $G$).
In particular, this gives an $\Ad$-invariant metric on $\g$. 
This also gives an induced metric on $\ad P = P\times_{Ad}\g$, which is well 
defined because the metric on both $P$ and  $\g$ are invariant under the action 
of  $G$. 

Denote that group of $L^2_3$-gauge transformations by  $\mathcal{G}_3(P) \subset
\Gamma(\Ad P)$ and the space of $L^2_2$-connections by $\mathcal{A}_2(P) \subset
\Omega^1(M,\ad P)$. The Sobolev spaces of sections are Banach manifolds, and 
the $L^2_k$ spaces are in fact Hilbert manifolds. However, we could work with
any Sobolev spaces for which the group of gauge transformations has at least two
derivatives, and the space of connections has one few derivative. 

% TODO 
define irreducible connections

\begin{thm} % Morgan 4.3.5 stabilisers, 4.4.5 local slices, and section 4.5
	The quotient space of gauge equivalent irreducible connections
    on a principal $\SU(2)$-bundle.
	$\mathcal{B}^*_2(P) := \mathcal{A}^*_2(P) / \mathcal{G}_3(P)$ is a 
	Hilbert manifold. 
\end{thm}
This is proved in \cite[Section 4.5]{morgan}. 

% see corollary 4.3.5 and section 4.5 MF
Provided that  $c_2(P)\neq 0$, the only possible stabilisers for elements of 
$\mathcal{A}_2(P)$ are $\{\pm 1\}$ or $S^1$. The former connections are called
irreducible and the latter ones reducible. 

\section{Yang-Mills theory}
\subsection{General definition}
Let $P\to M$ be a principal $G$-bundle over a Riemannian manifold $M$, with 
connection $A$ and associated curvature  $F$. Assume $G$ is a compact Lie group,
though we will typically choose $G=\SU(2)$ or $G=\SO(3)$. 
Since the curvature form of a principal bundle is horizontal and  
$\Ad$-equivariant, we can interpret it as an $\ad(P)$-valued form on the base 
$F\in\Omega(M,\ad P)$.
The Yang-Mills functional is defined by the $L^2$ norm squared of the curvature 
\[
	S_{YM}(A) := \norm{F}^2_{L^2} = \int_M \abs{F}^2 dV_g
\]
where $\abs{F}^2$ comes from the metric on $T^*M^{\otimes 2}\otimes \ad P$. 
The significance in physics is that this represents the energy of a free field, 
generalising electromagnetism for non-abelian gauge groups.  
(cite Baez)

The principal of least action in physics dictates that the classical solutions
are the connections that satisfy the Euler-Lagrange equations of this action 
functional, that is, locally minimise $S_{YM}$. 

Derive $d_A^* F = 0$ using wikipedia and introduce formal adjoint operator

\subsection{Orthogonal or unitary structure group}
In this subsection we assume $G=\SO(N)$ or  $G=\SU(N)$, which allows us to
make a number of simplifications.
\begin{defn}
	Let $(M,g)$ be an oriented Riemannian  $n$-manifold. The \underline{Hodge}
	\underline{star operator} $* : \Omega^k(M) \to \Omega^{n-k}(M)$ is the 
	unique smooth bundle homomorphism satisfying 
	\begin{equation} \label{eq:hodge_property}
	\omega \wedge *\eta = \gen{\omega,\eta}_g dV_g
	\end{equation}
	where $\gen{\cdot,\cdot}_g$ is the induced metric on $(T^*M)^{\otimes k}$.
	In orthonormal coordinates, this is given by
	\[
	*(dx_{i_1}\wedge \cdots\wedge dx_{i_k}) = \sgn(I)dx_{i_{k+1}}\wedge \cdots \wedge
	dx_{i_n} 
	\] 
	where $dx_1\wedge\cdots\wedge dx_n = dV_g$ on some open subset.
\end{defn}
The Hodge star extends to vector valued forms by only acting on the
$(T^*M)^{\otimes k}$ part. Recall that 
for matrix Lie algebra valued forms $\omega,\eta\in\Omega(M,\g)$, the wedge 
product is defined as 
\begin{equation}
    \omega\wedge\eta = \sum_{j,k} \omega_j\wedge \eta_k X_j X_k 
\end{equation}
We are interested in the Hodge star acting on 2-forms on a 4-manifolds in
particular, since $\star : \Omega^2(M,\g)\to\Omega^2(M,\g)$ is a linear operator
with $\star^2= 1$. The two eigenvalues 1 and -1 allow us to decompose into its
eigenspaces $\Omega^2(M,\g) = \Omega^{2,+}(M,\g) \oplus \Omega^{2,-}(M,\g)$,
called self-dual (SD) and anti-self-dual (ASD) two-forms respectively.

For an arbitrary Lie algebra valued form, equation (\ref{eq:hodge_property})
will not hold. But for the case where $G=\SO(N)$ or  $\SU(N)$, we can choose the
Hilbert-Schmidt metric  $\gen{X,Y}=\Tr(X^*Y)$ where $X^*$ is the adjoint, which
is an $\Ad$-invariant metric on $\g$.  In this case, if 
$\alpha \in \Omega^{2,+}(M,\g), \beta \in \Omega^{2,-}(M,\g)$, we have
\begin{equation} \label{eq:trace_hodge}
\Tr(\alpha\wedge \alpha) = -\abs{\alpha}^2 \odif{V}_g, \qquad
\Tr(\beta\wedge \beta) = \abs{\beta}^2 \odif{V}_g, \qquad
\alpha\wedge \beta = 0
\end{equation}
using equation (\ref{eq:hodge_property}) and $\Tr(XY) = -\Tr(X^*Y)$ for
$X,Y\in\g$. The last property follows from eigenspaces being orthogonal for any
metric on $\Omega^2(M,\g)$.


The Yang-Mills functional can be rewritten using equation (\ref{eq:trace_hodge}) as
\[
	S_{YM}(A) = -\int_M \Tr (F\wedge \star F) 
\] 

We now explain the significance of the Chern class.
Recall from Chern-Weil theory that there are two important classes of
$\Ad \GL(r,\mathbb{C})$ invariant polynomials on $\gl(r,\mathbb{C})$. 
The first is the coefficients $f_k(X)$ of $\lambda^{r-k}$ in the characteristic
polynomial $\det(\lambda I + X)$. Another class consists of the trace
polynomials $\Tr(X^k)$. 
The Chern classes of $P$ are defined by $c_k(P) =
[f_k\paren{\frac{i}{2\pi}F}] \in H^{2k}(M)$. 
\begin{thm}
	The second Chern class $c_2(E)\in H^4(M,\mathbb{Z})$ classifies up to
	isomorphism $\SU(2)$-bundles over any compact connected oriented 4-manifold
	 $M$. (Theorem E.5 in Freed and Uhlenbeck)
\end{thm}

From Newton's identity for symmetric polynomials with $k=2$ 
(see \cite[Theorem B.2]{loringtu}), we have $\Tr(X^2)-f_1(X)\Tr(X)+2f_2(X)=0$, 
giving the following relation for any complex vector bundle
\[ % 2.1.28 DK
	\bracket{\frac{1}{8\pi^2} \Tr(F^2)} = c_2(E) - \frac{1}{2}c_1(E)^2 \in H^4(M)
\] 
where we have used the fact that $f_1(X)=\Tr(X)$. 
In the case where the structure group is $\SU(n)$ or $\SO(n)$, the trace of the
curvature is zero, so $c_1(E)=0$ is trivial and 
\[
c_2(E) = \frac{1}{8\pi^2}\int_M \Tr(F^2)
\] 
Now, we can
decompose the field strength $F = F^+ + F^-$ into SD and ASD 
parts. Using the fact that $\Omega^{2,+}$ and $\Omega^{2,-}$ are orthogonal in
the Riemannian inner product, and
that $\Tr(X^2) = -\abs{X}^2$ for skew adjoint matrices, this 
gives 
\begin{align*}
\Tr(F^2) 
&= \Tr(F^{+}\wedge  F^{+}) - \Tr(F^{-}\wedge F^{-}) \\
&= - (\abs[*]{F^+}^2 - \abs[*]{F^-}^2) dV_g 
\end{align*}
Hence, $c_2(E) = \frac{1}{8\pi^2}\int_M
(\abs[*]{F^-}^2-\abs[*]{F^+}^2)$. 

Comparing with the Yang-Mills action, 
\begin{align*}
	S_{YM}(A) = \int_M (\abs[*]{F^+}^2+\abs[*]{F^-}^2) 
	= \begin{cases}
		8\pi^2 c_2(E) + 2\int_M \abs[*]{F^-}^2 \\
		-8\pi^2 c_2(E) + 2\int_M \abs[*]{F^+}^2
	\end{cases}
\end{align*}
We see that for $c_2(E)>0$, the action is bounded below by 
$S(A) >= 8\pi^2\abs{c_2(A)}$ and the self-dual connections $F^-=0$ are
minimisers. For $c_2(E)<0$ the action is bounded
below by $S(A)\geq -8\pi^2c_2(E)$, and minimised by anti-self-dual connections
$F^+=0$. 
Thus the classical equations of motion are equivalent to $\star F = \pm F$,
whose solutions are called (anti-)instantons. 

\section{Moduli space}
% chapter 4 Donaldson Kronheimer, Morgan
Note that $\star F = - F$ is a non-linear differential equation for
non-abelian gauge groups, and defines a subspace of the infinite dimensional
space of connections $\mathcal{A}$. This subspace can be regarded as the zero
locus of the section $s : \mathcal{A} \to \Omega^{2,+}$ given by $s(A) =
F_A+\star F_A$. The main goal is to define a finite-dimensional moduli space
starting from  $s^{-1}(0)$. 
The key property used to do this is that the section is equivariant with respect
to the action of the gauge group: $s(u^*(A))=u^*(s(A))$.

Hence if a connection is ASD, i.e. $s(A)=0$, then the transformed connection is
also ASD. The moduli space is defined by quotienting the space of ASD
connections by the action of the gauge group. 


\begin{remark}
If we reverse the orientation of $M$, then this swaps the SD and ASD forms in
$\Omega^2(M,\g)$. Since the two theories are completely
equivalent, we could work with SD connections. However, there is an
important class of 4-manifolds which have a natural orientation - complex
manifolds. Over this orientation, it turns out that ASD connections are 
associated to holomorphic objects, while SD connections to anti-holomorphic
objects.\cite[p.95]{morgan} For this reason, we choose to work in terms of 
ASD connections by default.
\end{remark}

The space of gauge connections $\mathcal{A}$ on a principal bundle is the
universal bundle for the group of gauge transformations $\mathcal{G}$. (ch 15
Cordes)


% TODO 
formal dimension of the moduli space

\section{Donaldson invariants}
% Lecture 7 morgan
Finally, our goal is to sketch the definition of the Donaldson polynomials
invariants. 
In this section, assume $M$ is a closed,
oriented, simply connected smooth 4-manifold with a fixed orientation of
$H^2_+(M;\mathbb{R})$. Further assume $b_2^+(M) > 1$. 

Let $P\to M$ be a principal $\SU(2)$-bundle with  $c_2(P) > 0$, let 
\[
d = 4c_2(P) - \frac{3}{2}(1+b_2^+(M))
\] 
which is half of the formal dimension of the moduli space
$\mathcal{M}^*(P)$. The Donaldson invariant associated to $P$ is a
symmetric multilinear function of degree  $d$ on $H_2(M;\mathbb{Z})$ 
\[
\gamma_d(M) : H_2(M;\mathbb{Z})^{\otimes d} \to \mathbb{Q}
\] 
Very roughly, the idea behind the definition is as follows: The Donaldson $\mu$ 
map $\mu: H_2(M,\mathbb{Z}) \to H^2(\mathcal{M}^*(P),\mathbb{Z})$ associates
homology classes in $M$ to cohomology classes in the moduli space. 
Since the moduli space is not compact, we need to define the Uhlenbeck 
compactification $\overline{\mathcal{M}}(P)$ of the
moduli space and extend the $\mu$ map to take values in
$H^2(\overline{\mathcal{M}}(P),\mathbb{Z})$.
For $x_1,\ldots,x_d \in H_2(M,\mathbb{Z})$, we have $\mu(x_1),\ldots,\mu(x_d)\in
H^2(\overline{\mathcal{M}}(P),\mathbb{Z})$ and we define the invariant to be the integral
over the moduli space
 \[
\gamma_d(x_1,\ldots,x_d) 
= \int_{\overline{\mathcal{M}}(P)} \mu(x_1)\wedge \cdots\wedge \mu(x_d)
\] 
Hence, the Donaldson invariant can be interpreted as 
an element in the polynomial algebra of $H^2(M,\mathbb{Z})$, i.e. dual to
$H_2(M,\mathbb{Z})$.  

% naber donaldson theory p59
The definition of the $\mu$ map relys on a certain $\SO(3)$-bundle, which we now
sketch. We drop the Sobolev subscripts (rephrase). 
Consider the principal $\mathcal{G}$-bundle
\[
\mathcal{A}^* \times P  \to \mathcal{A}^* \times_\mathcal{G} P 
\]
where the action is given by $(\omega,p)\cdot f = (f^*\omega, f^{-1}(p))$.
This action is free because the elements of $\mathcal{A}^*$ are irreducible:
$(f^*\omega,f^{-1}(p)) = (\omega,p)$ implies $f$ is a stabiliser of  $\omega$,
and  $f^{-1}(p)=p$ leaves only the possibility that $f=\id$.

There is a natural map $\mathcal{A}^* \times_\mathcal{G} P \to
\mathcal{B}^*(P)\times M$ given by $[\omega,p] \mapsto ([\omega],\pi(p))$.
The $\SU(2)$ action on  $\mathcal{A}^*\times P$ given by multiplication on $P$
descends to the quotient $\mathcal{A}^* \times_\mathcal{G} P$ because it 
commutes with the  $\mathcal{G}$ action.
But this action $[\omega,p]\cdot g = [\omega,p\cdot g]$ is not free because 
$g$ is a stabiliser if and only if $g=\pm \id\in \SU(2)$. Thus, there is a free
$\SU(2) /\id \simeq \SO(3)$ action on  $\mathcal{A}^*\times_{\mathcal{G}} P$ and
this becomes a principal $\SO(3)$-bundle.

\vspace{5mm}
\hrule 
\vspace{5mm}



\textbf{Bibliographical notes}
{\small
\begin{itemize}
	\item A great overview of the main ideas in the two approaches to
		constructing a Witten type TFT is given in \citet{TQFTbook}.
	\item The local model for $\mathcal{M}_{ASD}$ was first obtained by 
		\citet{local_moduli}.
	\item The geometry of the space of connections and the moduli space, 
		as well as the Donaldson polynomial invariants are
		discussed in more detail in \citet{morgan}. 
	\item DK
\end{itemize}
}
