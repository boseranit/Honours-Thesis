
\chapter{Characteristic Classes}
\label{chapter1}
\section{History}
Weyl (1918) introduced the concepts of gauge transformation and gauge
invariance, while seeking to unify electromagnetism with general relativity. After the
development of quantum mechanics, an idea was to make the global phase symmetry
into a local gauge symmetry by replacing the momentum operator with a covariant 
derivative in the Schrodinger equation, given by 
$\widehat{p}=\frac{\hbar}{i}(\nabla+iA)$. The result of this $\U(1)$ gauge symmetry
turns out to be profound, since it introduces EM interactions with a charged
quantum mechanical particle. This was the first widely accepted gauge theory,
and popularised by Pauli (\citeyear{pauli_em}).\cite{pauli_em}
Gauge theory was mostly limited to EM and GR until the paper of Yang and Mills 
(1954). They introduced $\SU(2)$ gauge theories for
isotopic spin axes, to understand the strong interaction. This idea later found
application to the model of the electroweak interaction by Weinberg and Salam
(1967), and also the Higgs mechanism (1966) for spontaneous symmetry breaking
allowing gauge fields to acquire mass. 

This also motivated the search for a strong force gauge theory, which is now
known as quantum chromodynamics. The Standard Model unifies the description of
electromagnetism, weak interactions and strong interactions in the language of
quantized gauge theory. 

The origin of TQFT can be traced to the work of Schwarz and Witten.
Schwarz showed that the Ray-Singer torsion, a particular topological invarint,
could be represented as the partition function of a certian quantum field theory
(1978). Unrelated to this observation, Witten gave a framework for understanding
Morse theory in terms of supersymmetric quantum mechanics (1982), which was 
generalised to an infinite dimensional version by Floer. 

In \citeyear{don83}, Donaldson built on his doctoral advisor Atiyah's work on 
Yang-Mills instantons to introduce his famous polynomial invariants, and prove 
Donaldson's theorem.\cite{don83}
Freedman used this work to exhibit existence of exotic differentiable
structures on $\mathbb{R}^{4}$. 
Witten (\citeyear{wittenTQFT}) provided a way
of obtaining the polynomials by formulating a supersymmetric TQFT, which became
known as Donaldson-Witten theory.\cite{wittenTQFT}
Another approach to obtain topological invariants, related to Donaldson's
construction, comes from Seiberg-Witten theory (1994). They formulated a set of
equations that contains the same information as the Yang-Mills equations but
are technically much easier to work with. 

\section{Outline}
Riemannian 4-manifold and a principal $G$-bundle $E$ over it. Given a connection
on this bundle, the curvature  $F(A)\in \Omega^2(\mathfrak{g})$ is a Lie algebra
valued 2-form. Then instantons are solutions to the ASD equation $\star F(A) =
-F(A)$. 

We can defined the gauge group to be the group of bundle automorphisms of  $E$.
This acts on the space of connections, and preserves the subspaces of
instantons, so we can mod out by this to define a moduli space of instantons
$\mathcal{M}(E)$. If we're lucky, this is a smooth manifold and the donaldson
invariant does not depend on the Riemannian metric and is an invariant of smooth
4-manifolds. 

The curvature is the field strength tensor of a physical gauge theory, while the
connection is called the gauge potential. (electromagnetism is an example)
The action of a Yang-Mills gauge theory is given by the integral
\[
S = \int_M \Tr (F\wedge \star F)
\] 
whose Euler-Lagrange equations are the classical equations of motion, i.e. the
classical solutions are stationary points of this functional. Now, we can
decompose the field strength into a self-dual and anti-self-dual part.
This gives 
\[
S = \int \Tr(F^{+}\wedge \star F^{+}) +\int \Tr(F^{-}\wedge \star F^{-})
\] 
and comparing this to the second Chern class $C_2(A) := \int \Tr(F\wedge F)$ we
can see that $\abs{S(A)} >= \abs{C_2(A)}$, i.e. it is locally minimized when
this equality holds. But this holds exactly when either $F^{+}=0$ or $F^{-}=0$.
Thus the classical equations of motion are equivalent to $\star F = \pm F$. So
instantons are just classical solutions to the equation of motion.

Instantons may be related by large gauge transformations, which may be
quotiented out.




\section{Introduction}
A gauge is a ``coordinate system" that varies depending on one's ``location"
with respect to some base space. A gauge transform is a change of coordinates
applied to each such location, and a gauge theory is a model for some system to
which gauge transforms can be applied, and is typically gauge invariant, in that
all meaningful quantities transform naturally under gauge transformations. By
fixing a gauge (or breaking the gauge symmetry), the model becomes easier to
analyse mathematically. 




A quantum field theory is called a TQFT when the correlation functions are
independent of the metric. 
TQFT has been an active area of research ever since the seminal work by Witten
\cite{wittenTQFT}. TQFT contains no excitations that may propagate in the
spacetime, so it does not describe any waves we know in the real world. The
characteristic quantity describing a configuration: the action, remains
invariant under any continuous changes of the topology. 

There are two ways to formally guarantee that the correlation functions remain
invariant under variations of the metric. % labatista, lozano
\begin{itemize}
	\item Schwarz type TFT: the action and the operators are defined without 
		using the metric of the manifold. The most notable example is
		Chern-Simmons gauge theory. Another import set of examples are the BF
		theories. 
	\item Witten type TFT: there is explicit metric
	dependence, but the theory has a scalar symmetry $\delta$ acting on
	the fields such that the correlation
	functions do not depend on the background metric. 
\end{itemize}

This thesis will focus on a particular Witten type TQFT: Donaldson-Witten
theory. Witten-type TQFTs can be formulated in a variety of frameworks. 
\begin{itemize}
	\item 
 The most geometric one corresponds to the Mathai-Quillen formalism. In this
formalism a TQFT is constructed out of a moduli problem. Topological invariants
are then defined as integrals of a certain Euler class over the resulting moduli
space.\cite{cernTQFT}
	\item 
 A different framework is the one based on the twisting of N = 2 supersymmetry.
In this case, information on the physical theory can be used in the
TQFT. (Seiberg-Witten invariants have shown up in this framework)
\end{itemize}

\section{Vector Bundles}
A connection on a vector bundle $E \to M$ is a map  $\nabla : \Gamma(TM) \times
\Gamma(E) \to \Gamma(E)$. (additional properties)

Suppose $\sigma_1,\ldots,\sigma_r \in \Gamma(U,E)$ is a frame for $E$, then the
connection matrix of $\nabla$ relative to the frame is the matrix of 1-forms
$A^{ij}\in \Omega^{1}(U)$ defined by
\[
	\nabla_X \sigma_i = \sum_{j=1}^{r} A^{ji}(X) \otimes \sigma_j
\] 

Similarly, the curvature of a vector bundle is a map $F_\nabla :
\mathcal{X}(M)\times \mathcal{X}(M) \times \Gamma(E) \to \Gamma(E)$. Then the
curvature matrix of the connection $\nabla$ relative to the frame is the matrix
of 2-forms  $\Omega^{ij}\in \Omega^2(U)$ defined by
\[
F_\nabla(X,Y)e_i = \sum_{j=1} \Omega^{ji}(X,Y)e_j
\] 

\section{Principal Bundles}
A connection on a principal $G$-bundle $P\to M$ is a 1-form $\theta \in
\Omega^1(P,\mathfrak{g})$, satisfying certain properties.

The curvature of the connection is a 2-form $F_\theta\in\Omega^2(M,\ad P)$.
But $\pi^* F_\theta = d\theta + \theta\wedge\theta \in\Omega^2(P,\mathfrak{g})$.


\section{Characteristic classes}
Recall that for a compact, oriented, Riemannian 2-manifold the Gauss-Bonnet
theorem states that 
\[
	\int_M \frac{1}{2\pi} K \operatorname{vol} = \chi(M)
\] 
which shows that the integral of the Gaussian curvature, which is locally the 
component of the curvature form relative to an orthonormal frame, is a
diffeomorphism invariant, independent of the Riemannian structure. 

Let $E\to M$ be a vector bundle of rank $r$. Under a change of frame
$\overline{\sigma}= \sigma a$, where $a : M \to GL(r,\mathbb{R})$, recall that
the curvature matrix transforms as $\overline{\Omega}=a^{-1}\Omega a$.  
The motivation for characteristic classes is to find a polynomial $P(X)$ in the
entries of $X\in \mathbb{R}^{r\times r}$, such that it is invariant under
conjugation by all $A\in \GL(r,\mathbb{R})$. That is, $P(A^{-1}XA)=P(X)$; call
this property $\Ad \GL(r,\mathbb{R})$-invariant. Then $P(\Omega)$ would be
independent of the frame and define a global form, with further nice
properties as the following theorem will show.

Denote $\operatorname{Inv}(\mathfrak{gl}(r,\mathbb{R}))$ as the ring of $\Ad
\GL(r,\mathbb{R})$-invariant polynomials on $\mathbb{R}^{r\times r}$.
The following theorem is described in \cite[Thm 23.3]{loringtu}.

\begin{thm}[Chern-Weil theorem ] \label{thm:chern_weil}
	Let $P\in\operatorname{Inv}(\mathbb{R}^{r\times r})$ of degree $k$. 
	Then $P(\Omega) \in \Omega^{2k}(M,\mathbb{C})$ satisfies
	\begin{enumerate}[(i)]
	    \item $dP(\Omega)= 0$, i.e. $P(\Omega)$ is a closed form
		\item Given two connections $A_0$ and $A_1$, and corresponding
			 curvature matrices $\Omega_0$ and $\Omega_1$, the difference
			 $P(\Omega_1)-P(\Omega_0)$ is exact. So the de Rham cohomology class
			 $[P(\Omega)]$ is independent of the choice of connection.
	\end{enumerate}
\end{thm}

Note that $P(\Omega)$ is a global form due to the discussion above.
This theorem can also be applied to a principal $G$-bundle $P \to M$ in a 
straightforward way. Let $P\in\operatorname{Inv}(\mathbb{R}^{r\times r})$. 
Since the curvature is locally in $\Omega \in $ 
We extend the domain of invariant polynomials from
$\mathfrak{gl}(k,\mathbb{R})$, to $\mathfrak{g}$-valued  $p$-forms on  $M$. For
$X\eta$ with  $X\in \mathfrak{g},\eta\in\Omega^{p}(M)$ we can define
\[
	P(X\eta) := \underbrace{k}{\eta\wedge \ldots\wedge\eta}P(X)
\] 

There are two ways to prove part (i). One is to establish 
a correspondence between homogeneous polynomials 
in $\operatorname{Inv}(\mathbb{R}^{r\times r})$ of degree $k$, and symmetric
$k$-linear polynomials  $\widetilde{P}: \mathfrak{gl}(r,\mathbb{R})^{k}\to \mathbb{R}$
which are $\Ad \GL(r,\mathbb{R})$-invariant, i.e. 
\[
\widetilde{P}(\Ad_A X_1,..,\Ad_A X_k) = \widetilde{P}(X_1,\ldots,X_k)
\]
Then it can be proved using appropriate choices of $A$ and $X_i$, and the
Bianchi identity.

In order to prove this we need to establish an algebraic theorem.

\begin{ex} (Coefficients of the characteristic polynomial)
	Let $X \in \mathfrak{gl}(r,\mathbb{R}) = \mathbb{R}^{r\times r}$, and let $\lambda
	\in\mathbb{R}$. The coefficients $f_k(X)$  of $\lambda^{r-k}$ in 
	\[
	\det(\lambda I + X) = \lambda^{r} + f_1(X)\lambda^{r-1} + \ldots + f_r(X)
	\] 
	are polynomials on $\mathfrak{gl}(r,\mathbb{R})$. Moreover, they are $\Ad
	\GL(r,\mathbb{R})$ invariant since the determinant is not basis dependent.
\end{ex}
\begin{ex} (Trace polynomials)
	Define the polynomial $\Sigma_k(X) := \Tr(X^{k})$ which is another example
	of an invaiant polynomial on $\mathfrak{gl}(r,\mathbb{R})$.
\end{ex}
The two sets of polynomials above play a crucial role in characteristic classes,
because of the following theorem.
\begin{thm}[{\cite[Thm 23.4]{loringtu}}] \label{thm:invariant_poly}
	The ring $\operatorname{Inv}(\mathfrak{gl}(r,\mathbb{R}))$ of invariant
	polynomials is generated by the set $f_k(X)$ or the trace polynomials
	$\Sigma_k(X)$:
	\[
	\operatorname{Inv}(\mathfrak{gl}(r,\mathbb{R}))
	= \mathbb{R}[f_1,\ldots,f_r] = \mathbb{R}[\Sigma_1,\ldots,\Sigma_r]
	\] 
\end{thm}

Thus, it is sufficient to prove Thm \ref{thm:chern_weil} for trace polynomials,
because the exterior derivative is linear, and if the cohomology classes of
$\Sigma_k(\Omega)$ are independent of the connection, then it also holds for a
polynomial in $\Sigma_k(\Omega)$.

\begin{defn}
	if $A = [\alpha^{ij}]$ and $B = [\beta^{ij}]$ are matrices of $k$-forms on
	 $M$, with the number of columns of  $A$ equal to number of rows of  $B$,
	 then 
	  \begin{enumerate}[(i)]
	     \item $(A\wedge B)^{ij} = \sum_k \alpha ^{ik} \wedge\beta^{ki} $
	     \item $(dA)^{ij} = d\alpha ^{ij}$
	 \end{enumerate}
\end{defn}
\begin{prop}
	Let $A = [\alpha^{ij}]$ and $B = [\beta^{ij}]$ are matrices of differential
	forms on $M$ with degrees $a$ and  $b$ respectively.
	\begin{enumerate}[(i)]
	    \item $\Tr(A\wedge B) = (-1)^{ab} \Tr(B\wedge A)$
		\item $d \Tr(A) = \Tr(dA)$
	\end{enumerate}
\end{prop}

\begin{proof}[Proof of Thm \ref{thm:chern_weil}]
	\begin{enumerate}[(i)]
	    \item As mentioned, we only need to consider trace polynomials
			$f_k(X)=\Tr(X^k)$. Then 
		\begin{align*}
			d \Tr(\Omega^{k})
			&= \Tr (d \Omega^{k}) \\
			&= \Tr (\Omega^{k} \wedge A - A \wedge \Omega^{k}) \tag{generalised
			Bianchi identity} \\
			&= \Tr (\Omega^{k} \wedge A) - \Tr(A \wedge \Omega^{k}) \\
			&= 0 
		\end{align*}
	\item \cite[Prop 5.28]{morita}	
		\begin{comment}
		Again, we only need to consider $P = \Tr(X^k) 
		\in \operatorname{Inv}(\mathfrak{gl}(r,\mathbb{R}))$. 
		To show that the cohomology class does not depend on the choice of
		connection, we need to show that for two connections $\nabla^0$ and
		$\nabla^1$,$P(\Omega_0) - P(\Omega_1) \in \Im(d)$. We will construct a
		$(2k-1)$-form whose exterior derivative is the above.
		\end{comment}
	Consider the vector bundle $E\times I
	\xrightarrow{\pi} M\times I$, where $I = [0,1]$. Any section of  $E\times
	\mathbb{R}$ is of the form $(x,t) \mapsto (s(x),t)$, and so can be written
	in a frame with coefficients independent of $t$. So sections of $E$ are in
	correspondence with sections of  $E\times I$. Define a connection $\nabla$ 
	on $E\times I$ by $\nabla_{\partial t} s = 0$ and $\nabla_X =
	(1-t)\nabla^0_X+t\nabla^1_X$ where $X$ is any vector field with zero
	component in  $\pdv{}{t}$.

	Using the associated curvature form $\Omega$, we get a cohomology class
	$[P(\Omega)]$. There is a natural inclusion  $i_j : M \to M \times
	\mathbb{R}$ given by $i_j(x)=(x,j)$ where $j=0,1$. By definition of
	$\nabla$, the pullback is $i^*_j \Omega = \Omega^j$, and thus $i^*_j
	P(\Omega) = P(\Omega^j)$.
	
	Also note that $i_0$ and $i_1$ are homotopic, so the induced homomorphisms on
	cohomology are identical. This gives 
	\[
		[P(\Omega^0)] = i_0^*([P(\Omega)]) = i_1^*([P(\Omega)])=[P(\Omega^1)]
	\] 
	\end{enumerate}
\end{proof}

Hence there is a well defined algebra homomorphism $c_E :
\operatorname{Inv}(\mathfrak{gl}(r,\mathbb{R}))\to H^{*}(M)$, called the
Chern-Weil homomorphism.
The cohomology class $[P(\Omega)]$ is called the \underline{characteristic
class} of $E$ associated to  $P$. 

\subsection{Pontrjagin classes}
From Theorem \ref{thm:invariant_poly}, the algebra
$\operatorname{Inv}(\mathfrak{gl}(r,\mathbb{R}))$ is generated by the
coefficients $f_k(X)$ of  $\det(\lambda I + X)$ and by the trace polynomials
$\Tr(X^{k})$. To determine all the characteristic classes of $E$, it suffices to
calculate the characteristic classes associated to either set of polynomials.
The classes arising from $f_k(X)$ are called Pontrjagin classes.



\subsection{Euler class}
Show how $\det(\lambda I + X)$ is an example.

\subsection{Chern class}


