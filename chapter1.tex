
\chapter{Chapter title goes here}
\label{chapter1}
\section{Reference Guide}
\cite{cernTQFT} is a 63 page introduction to TQFT, including the Mathai-Quillen
formalism and Donaldson Witten theory using twisted $N=2$ supersymmetry.
Lacking details

\cite{wittenTQFT} is the seminal article on twisted supersymmetric gauge theory.
Including how donaldson invariants arise

\cite{birminghamTFT} is a 366pg textbook on supersymmetry, BRST, euler
character, sigma models, donaldson theory, Schwarz type TFTs, topological
gravity and renormalization

\cite{cordes95} is 247pg of lecture notes, and focuses on equivariant cohomology
and TQFT from pg 75. Moves through topics very quickly.

\cite{axiomTQFTintro} Motivates the functorial axiomatisation of TQFT using the
path integral approach

\cite{marino} 40 pages, goes through Donaldson invariants, supersymmetry and donaldson
witten theory, by stating a number of result, but without detail and lacking
proofs. 

\cite{moore} 80 pg, similar to marino. Focuses on path integral approach to
cohomological TFT, and Mathai quillen

\cite{TQFTbook} More detailed version of marino's lectures on TFT. 

\cite{MQformula} is the original paper by Mathai and Quillen which introduced a
formula for the euler number.

\cite{atiyahlagrangians} a short article showing how Witten's Lagrangian should
be understood in terms on the Gauss-Bonnet formula. Starts with Mathai-Quillen
formula for Thom class

\cite{MQintro} is a 35 pg introductory accound of the previous paper, beginning
with Mathai-Quillen formalism for finite dimen vector bundles

\section{History}
Weyl (1918) introduced the concepts of gauge transformation and gauge
invariance, while seeking to unify electromagnetism with general relativity. After the
development of quantum mechanics, an idea was to make the global phase symmetry
into a local gauge symmetry by replacing the momentum operator with a covariant 
derivative in the Schrodinger equation, given by 
$\widehat{p}=\frac{\hbar}{i}(\nabla+iA)$. The result of this $\U(1)$ gauge symmetry
turns out to be profound, since it introduces EM interactions with a charged
quantum mechanical particle. This was the first widely accepted gauge theory,
and popularised by Pauli (\citeyear{pauli_em}).\cite{pauli_em}
Gauge theory was mostly limited to EM and GR until the paper of Yang and Mills 
(1954). They introduced $\SU(2)$ gauge theories for
isotopic spin axes, to understand the strong interaction. This idea later found
application to the model of the electroweak interaction by Weinberg and Salam
(1967), and also the Higgs mechanism (1966) for spontaneous symmetry breaking
allowing gauge fields to acquire mass. 

This also motivated the search for a strong force gauge theory, which is now
known as quantum chromodynamics. The Standard Model unifies the description of
electromagnetism, weak interactions and strong interactions in the language of
quantized gauge theory. 

The origin of TQFT can be traced to the work of Schwarz and Witten.
Schwarz showed that the Ray-Singer torsion, a particular topological invarint,
could be represented as the partition function of a certian quantum field theory
(1978). Unrelated to this observation, Witten gave a framework for understanding
Morse theory in terms of supersymmetric quantum mechanics (1982), which was 
generalised to an infinite dimensional version by Floer. 

In \citeyear{don83}, Donaldson built on his doctoral advisor Atiyah's work on 
Yang-Mills instantons to introduce his famous polynomial invariants, and prove 
Donaldson's theorem.\cite{don83}
Freedman used this work to exhibit existence of exotic differentiable
structures on $\mathbb{R}^{4}$. 
Witten (\citeyear{wittenTQFT}) provided a way
of obtaining the polynomials by formulating a supersymmetric TQFT, which became
known as Donaldson-Witten theory.\cite{wittenTQFT}
Another approach to obtain topological invariants, related to Donaldson's
construction, comes from Seiberg-Witten theory (1994). They formulated a set of
equations that contains the same information as the Yang-Mills equations but
are technically much easier to work with. 

\section{Outline}
Riemannian 4-manifold and a principal $G$-bundle $E$ over it. Given a connection
on this bundle, the curvature  $F(A)\in \Omega^2(\mathfrak{g})$ is a Lie algebra
valued 2-form. Then instantons are solutions to the ASD equation $\star F(A) =
-F(A)$. 

We can defined the gauge group to be the group of bundle automorphisms of  $E$.
This acts on the space of connections, and preserves the subspaces of
instantons, so we can mod out by this to define a moduli space of instantons
$\mathcal{M}(E)$. If we're lucky, this is a smooth manifold and the donaldson
invariant does not depend on the Riemannian metric and is an invariant of smooth
4-manifolds. 

The curvature is the field strength tensor of a physical gauge theory, while the
connection is called the gauge potential. (electromagnetism is an example)
The action of a Yang-Mills gauge theory is given by the integral
\[
S = \int_M \Tr (F\wedge \star F)
\] 
whose Euler-Lagrange equations are the classical equations of motion, i.e. the
classical solutions are stationary points of this functional. Now, we can
decompose the field strength into a self-dual and anti-self-dual part.
This gives 
\[
S = \int \Tr(F^{+}\wedge \star F^{+}) +\int \Tr(F^{-}\wedge \star F^{-})
\] 
and comparing this to the second Chern class $C_2(A) := \int \Tr(F\wedge F)$ we
can see that $\abs{S(A)} >= \abs{C_2(A)}$, i.e. it is locally minimized when
this equality holds. But this holds exactly when either $F^{+}=0$ or $F^{-}=0$.
Thus the classical equations of motion are equivalent to $\star F = \pm F$. So
instantons are just classical solutions to the equation of motion.

Instantons may be related by large gauge transformations, which may be
quotiented out.




\section{Introduction}
A gauge is a ``coordinate system" that varies depending on one's ``location"
with respect to some base space. A gauge transform is a change of coordinates
applied to each such location, and a gauge theory is a model for some system to
which gauge transforms can be applied, and is typically gauge invariant, in that
all meaningful quantities transform naturally under gauge transformations. By
fixing a gauge (or breaking the gauge symmetry), the model becomes easier to
analyse mathematically. 




A quantum field theory is called a TQFT when the correlation functions are
independent of the metric. 
TQFT has been an active area of research ever since the seminal work by Witten
\cite{wittenTQFT}. TQFT contains no excitations that may propagate in the
spacetime, so it does not describe any waves we know in the real world. The
characteristic quantity describing a configuration: the action, remains
invariant under any continuous changes of the topology. 

There are two ways to formally guarantee that the correlation functions remain
invariant under variations of the metric. % labatista, lozano
\begin{itemize}
	\item Schwarz type TFT: the action and the operators are defined without 
		using the metric of the manifold. The most notable example is
		Chern-Simmons gauge theory. Another import set of examples are the BF
		theories. 
	\item Witten type TFT: there is explicit metric
	dependence, but the theory has a scalar symmetry $\delta$ acting on
	the fields such that the correlation
	functions do not depend on the background metric. 
\end{itemize}

This thesis will focus on a particular Witten type TQFT: Donaldson-Witten
theory. Witten-type TQFTs can be formulated in a variety of frameworks. 
\begin{itemize}
	\item 
 The most geometric one corresponds to the Mathai-Quillen formalism. In this
formalism a TQFT is constructed out of a moduli problem. Topological invariants
are then defined as integrals of a certain Euler class over the resulting moduli
space.\cite{cernTQFT}
	\item 
 A different framework is the one based on the twisting of N = 2 supersymmetry.
In this case, information on the physical theory can be used in the
TQFT. (Seiberg-Witten invariants have shown up in this framework)
\end{itemize}

\section{Characteristic classes}
Recall that for a compact, oriented, Riemannian 2-manifold the Gauss-Bonnet
theorem states that 
\[
	\int \frac{1}{2\pi} K \operatorname{vol} = \chi(M)
\] 
which shows that the integral of the Gaussian curvature, which is locally the 
component of the curvature form relative to an orthonormal frame, is a
diffeomorphism invariant, independent of the Riemannian structure. 

Let $E\to M$ be a vector bundle of rank $r$. Under a change of frame
$\overline{\sigma}= \sigma a$, where $a : M \to GL(r,\mathbb{R})$, recall that
the curvature matrix transforms as $\overline{\Omega}=a^{-1}\Omega a$.  
The motivation for characteristic classes is to find a polynomial $P(X)$ in the
entries of $X\in \mathbb{R}^{r\times r}$, such that it is invariant under
conjugation by all $A\in \GL(r,\mathbb{R})$. That is, $P(A^{-1}XA)=P(X)$; call
this property $\Ad$-invariant. Then $P(\Omega)$ would be
independent of the frame and define global form, and has further nice
properties as the following theorem will show.

\begin{thm}[Chern-Weil theorem \cite{nakahara}] Let $P$ be an $\Ad$-invariant 
	polynomial in the entries of $\mathbb{R}^{r\times r}$ of degree $k$. 
	Then $P(\Omega) \in \Omega^{2k}(M,\mathbb{C})$ satisfies
	\begin{enumerate}[(i)]
	    \item $dP(\Omega)= 0$, i.e. $P(\Omega)$ is a closed form
		\item Given two connections $A_0$ and $A_1$, and corresponding
			 curvature matrices $\Omega_0$ and $\Omega_1$, the difference
			 $P(\Omega_1)-P(\Omega_0)$ is exact. So the cohomology class
			 $[P(\Omega)]$ is independent of the choice of connection.
	\end{enumerate}
\end{thm}
\begin{proof}
	show all the working
\end{proof}

The cohomology class $[P(F_A)]$ is called the \underline{characteristic
class} of  $E$ associated to  $P$. 

\subsection{Pontrjagin classes}


\subsection{Chern classes}
Show how $\det(\lambda I + X)$ is an example.

\section{Donaldson-Witten theory}
This is a retelling of Witten's 1988 TQFT paper. 
% https://physics.stackexchange.com/questions/224297/what-to-a-physicist-are-instantons-and-the-donaldson-invariants

The physical setting in which the Donaldson invariants will appear is a
Yang-Mills theory living on the four-dimensional manifold $M$ coupled to certain
fields such that the total action has a supersymmetry. The action of this theory
is 
\begin{align*}
	S &= \int \Tr \Big(\frac{1}{4}F_{\mu\nu}F^{\mu\nu} +\frac{1}{4}F_{\mu\nu}(\star
F)^{\mu\nu} +\frac{1}{2}\phi D_\mu D^\mu\lambda - i\eta D_\mu\psi^\mu +
iD_\mu\psi_\nu\chi^{\mu\nu} \\ 
	  &\qquad - \frac{i}{2}\lambda[\psi_\mu,\psi^\mu]-\frac{i}{2}\phi[\eta,\eta]-\frac{1}{8}[\phi,\lambda]^2\Big)\sqrt{g} \odif[order=4]{x}
\end{align*}
where $D_\mu = \nabla_\mu + [A_\mu,+]$ is the gauge covariant derivative and
$\nabla$ is the Riemannian covariant derivative, and all fields are
$\mathfrak{g}$-valued. There is a  $\mathbb{Z}_2$-grading on the space of
fields. The bosonic fields are $\phi,\lambda$ and the fermionic fields are
$\eta,\psi,\chi$ and  $\chi$ is additional constrained to be self-dual. The
action is invariant under the symmetry 
\begin{align*}
	\delta_\epsilon A = \mathrm{i}\epsilon\psi \quad \delta_\epsilon\phi = 0
	\quad \delta_\epsilon\lambda = 2\mathrm{i}\epsilon\eta \\
	\delta_\epsilon\psi = -\epsilon D\phi \quad \delta_\epsilon\eta =
	\frac{1}{2}\epsilon[\phi,\lambda] \quad \delta_\epsilon \chi = \epsilon(F +
	\ast F) 
\end{align*}
with $\epsilon$ a fermionic infinitesimal parameter. As with all
transformations, we think of this one as having a generator: its supercharge
$Q$, which gives all transformations as  $\delta \alpha=-i\epsilon Q(\alpha)$
where $\alpha$ is any field. in a Hamiltonian formulation,  $Q(\alpha)$ would be
the Poisson bracket  $\{Q,\alpha\}$, but on a general manifold we don't have
that option. By explicit computation, we find that 
\[
	\delta_\epsilon\delta_\zeta X - \delta_\zeta\delta_\epsilon X =
	-2\mathrm{i}\epsilon\zeta\phi
\] 
for every field $X$ except  $A$, where it is  $-D\phi$. This holds only on-shell
for  $\chi$, but off-shell for all others. Therefore, the commutator of two such
transformations is a gauge transformation, and hence has no physical impact. The
conserved current associated to this symmetry is 
 \[
	 J = \mathrm{Tr}_\mathfrak{g}\left((F_{\mu\nu} + (\ast F)_{\mu\nu})\psi^\nu
		 - \eta D_\mu \phi - D^\nu\phi \chi_{\mu\nu} -
	 \frac{1}{2}\psi_\mu[\lambda,\phi]\right)\mathrm{d}x^\mu
\] 
where conservation means that $\star J$ is closed, so for any homology 3-cycle
 $\gamma$, the integral  $Q(\gamma) = \int_\gamma \star J$ depends only on the
 homology class of  $\gamma$. Furthermore, one may show that the energy momentum
 tensor  $T_{\mu\nu}= 2 \frac{\delta S}{\delta g^{\mu\nu}}$ of this theory is an
 infinitesimal transform $T_{\mu\nu}=\{Q,\lambda_{\mu\nu}\}$ for $\lambda$ given
 in Witten's eq.(2.34). 

\section{Donaldson invariants as path integrals}
In the following, the path integral measure $\mathcal{D}X$ includes all fields,
and also intends to have gauge equivalence calsses quotiented out. The generic
object we consider is the (unnormalized) expectation value of any observable
$\mathcal{O}$ which is any nice functional in the fields:
\[
Z(\mathcal{O}) = \int \mathcal{O} \exp(-S /e^2) \mathcal{D}X
\] 
If the supersymmetry transformation is non-anomalous, we have
$Z(\{Q,\mathcal{O}\})= 0$ for every observable. We now claim that $Z=Z(1)$ is a
smooth invariant, and in particular will turn out to be a Donaldson invariant.
For  $Z$ to be a smooth invariant, it must be invariant under changes in the
metric. The change of metric is by definition  $\delta
S=\frac{1}{2}\int_M\sqrt{g} T_{\mu\nu}\delta g^{\mu\nu}$ and this leads to
\[
	\delta Z(1) = -\frac{1}{e^2}Z(\{Q,\int_M \sqrt{g}\delta
	g^{\mu\nu}\lambda_{\mu\nu}\}) = 0
\] 
so $Z(1)$ is invariant under changes of the metric. Similarly, it is invariant
under changes of the gauge coupling constant  $e$,   as long as it stays
non-zero. But in the limit of small coupling, the path integral is strongly
dominated by the classical minima of the free theories, and the calssical minima
of the free gauge theory are the anti-self-dual instantons. The self-dual ones
are not minima beause we have added $F\wedge F$ to the Lagrangian. So we may
evaluate  $Z$ by looking at the instanton contributions. 


\section{Euler number}
\subsection{Finite dimensional vector bundle}
Let $E \xrightarrow{\pi} X$ be a real vector bundle.  
% Assume $E$ and  $X$ are orientable,  $X$ compact without boundary, rank  $E$
% is even with  $\rank(E) = 2m \leq \dim(X) = n$.

The Euler class of  $E$

\section{Mathai-Quillen formula for Euler number}
A recurrent concept in this thesis will be the Euler number of a vector bundle.
There are two quite different approaches for calculating the Euler number
$\chi(X) = \chi(TX)$ of a manifold  $X$. This first is topological, and counts
the signed isolated zeros of a vector field on $X$, via the Hopf theorem. The
second is differential geometric and represents  $\chi(X)$ as the integral over
$X$ of a density constructed from the curvature of some connection on  $X$, via
the Gauss-Bonnet theorem. 

By extension, the Euler number of an arbitrary vector bundle $E$ over $X$ can be
determined in terms of a section, or the curvature of a connection on  $E$.

The Mathai-Quillen formula \cite{MQformula} is a more general formula which interpolates between
the two approaches. It relies on the construction of a form $e_{s,\Delta}(E)$ which depends
on both a section  $s$ and connection  $\Delta$, with the property
$\chi(E)=\int_X e(E)$ for all $s$ and  $\Delta$. Moreover, this equation reduces
to the Hopf or Gauss-Bonnet theorem for an appropriate choice of $s$. 

\section{TQFT as a generalisation of Mathai-Quillen}
% section 4.2 MQintro
 This approach is
due to \cite{atiyahlagrangians}, who showed that TQFT can be regarded as an
infinite dimensional generalisation of the Mathai-Quillen construction. 
Although $e(E)$ and  $\int_X e(E)$ do not make sense for
infinite dimensional  $E$ and  $X$, the Mathai-Quillen form  $e_{s,\Delta}$ can
be used to formally define regularized Euler numbers $\chi_s(E) := \int_X
e_{s,\Delta}(E)$. Although not independent of $s$, these numbers  $\chi_s(E)$
are of topological interest for certain choices of  $s$.  


