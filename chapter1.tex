
\chapter{Donaldson Invariants}
\label{chapter1}
\section{Yang-Mills theory}
Riemannian 4-manifold and a principal $G$-bundle $E$ over it. Given a connection
on this bundle, the curvature  $F(A)\in \Omega^2(\mathfrak{g})$ is a Lie algebra
valued 2-form. Then instantons are solutions to the ASD equation $\star F(A) =
-F(A)$. 

We can defined the gauge group to be the group of bundle automorphisms of  $E$.
This acts on the space of connections, and preserves the subspaces of
instantons, so we can mod out by this to define a moduli space of instantons
$\mathcal{M}(E)$. If we're lucky, this is a smooth manifold and the donaldson
invariant does not depend on the Riemannian metric and is an invariant of smooth
4-manifolds. 

The curvature is the field strength tensor of a physical gauge theory, while the
connection is called the gauge potential. (electromagnetism is an example)
The action of a Yang-Mills gauge theory is given by the integral
\[
S = \int_M \Tr (F\wedge \star F)
\] 
whose Euler-Lagrange equations are the classical equations of motion, i.e. the
classical solutions are stationary points of this functional. Now, we can
decompose the field strength $F = F^+ + F^-$ into self-dual and anti-self-dual
parts. Using the fact that $\Omega^{2,+}$ and $\Omega(2,-)$ are orthogonal, this 
gives 
\[
S = \int \Tr(F^{+}\wedge \star F^{+}) +\int \Tr(F^{-}\wedge \star F^{-})
\] 
\begin{thm}
	The second Chern class $c_2(E)\in H^4(M,\mathbb{Z})$ classifies up to
	isomorphism $\SU(2)$-bundles over any compact connected oriented 4-manifold
	 $M$. (appendix A in Freed and Uhlenbeck)
\end{thm}
The Lie algebras of the groups $\SU(n)$ or $\SO(n)$ correspond to traceless
matrices, so in these cases  the second Chern class simplfies 
\[
c_2(E) = \frac{1}{8\pi^2}\int_M (\Tr(F^2)-(\Tr(F)^2)) = 
\frac{1}{8\pi^2}\int_M \Tr(F\wedge F)
\] 
(why is $F\wedge F$ the same as  $F^2$)
Again decomposing $F=F^++F^-$, we can write  $c_2(E) = \frac{1}{8\pi^2}\int_M
(\abs[*]{F^+}^2-\abs[*]{F^-}^2)$. 

Comparing with the Yang-Mills action, 
\begin{align*}
	S_{YM}(A) = \int_M (\abs[*]{F^+}^2+\abs[*]{F^-}^2) 
	= \begin{cases}
		8\pi^2 c_2(E) + 2\int_M \abs[*]{F^-}^2 \\
		-8\pi^2 c_2(E) + 2\int_M \abs[*]{F^+}^2
	\end{cases}
\end{align*}
We see that for $c_2(E)>0$, the action is bounded below by 
$S(A) >= 8\pi^2\abs{c_2(A)}$ and the self-dual connections $F^-=0$ are
minimisers. For $c_2(E)<0$ the action is bounded
below by $S(A)\geq -8\pi^2c_2(E)$, and minimised by anti-self-dual connections
$F^+=0$. 
Thus the classical equations of motion are equivalent to $\star F = \pm F$,
whose solutions are called (anti-)instantons.

Note that $\star F = - F$ is a non-linear differential equation for
non-abelian gauge groups, and defines a subspace of the infinite dimensional
space of connections $\mathcal{A}$. This subspace can be regarded as the zero
locus of the section $s : \mathcal{A} \to \Omega^{2,+}$ given by $s(A) =
F_A+\star F_A$. The main goal is to define a finite-dimensional moduli space
starting from  $s^{-1}(0)$. 
The key property used to do this is that the section is equivariant with respect
to the action of the gauge group: $s(u^*(A))=u^*(s(A))$.

Hence if a connection is ASD, i.e. $s(A)=0$, then the transformed connection is
also ASD. The moduli space is defined by quotienting the space of ASD
connections by the action of the gauge group. 

\section{Yang-Mills theory}

moduli space of connections quotiented by gauge group

The space of gauge connections $\mathcal{A}$ on a principal bundle is the
universe bundle for the group of gauge transformations $\mathcal{G}$. (ch 15
Cordes)

\section{Donaldson invariants}
Intersection form

\vspace{5mm}
\hrule 
\vspace{5mm}



\textbf{Bibliographical notes}
{\small
\begin{itemize}
	\item A great overview of the main ideas in the two approaches to
		constructing a Witten type TFT is given in \citet{TQFTbook}.
	\item The local model for $\mathcal{M}_{ASD}$ was first obtained by 
		\citet{local_moduli}.
\end{itemize}
}
