
\chapter{Donaldson Invariants}
\label{chapter1}
\section{Yang-Mills theory}
Riemannian 4-manifold and a principal $G$-bundle $E$ over it. Given a connection
on this bundle, the curvature  $F(A)\in \Omega^2(\mathfrak{g})$ is a Lie algebra
valued 2-form. Then instantons are solutions to the ASD equation $\star F(A) =
-F(A)$. 

We can defined the gauge group to be the group of bundle automorphisms of  $E$.
This acts on the space of connections, and preserves the subspaces of
instantons, so we can mod out by this to define a moduli space of instantons
$\mathcal{M}(E)$. If we're lucky, this is a smooth manifold and the donaldson
invariant does not depend on the Riemannian metric and is an invariant of smooth
4-manifolds. 

The curvature is the field strength tensor of a physical gauge theory, while the
connection is called the gauge potential. (electromagnetism is an example)
The action of a Yang-Mills gauge theory is given by the integral
\[
S = \int_M \Tr (F\wedge \star F)
\] 
whose Euler-Lagrange equations are the classical equations of motion, i.e. the
classical solutions are stationary points of this functional. Now, we can
decompose the field strength $F = F^+ + F^-$ into self-dual and anti-self-dual
parts. Using the fact that $\Omega^{2,+}$ and $\Omega(2,-)$ are orthogonal, this 
gives 
\[
S = \int \Tr(F^{+}\wedge \star F^{+}) +\int \Tr(F^{-}\wedge \star F^{-})
\] 
\begin{thm}
	The second Chern class $c_2(E)\in H^4(M,\mathbb{Z})$ classifies up to
	isomorphism $\SU(2)$-bundles over any compact connected oriented 4-manifold
	 $M$. (appendix A in Freed and Uhlenbeck)
\end{thm}
The Lie algebras of the groups $\SU(n)$ or $\SO(n)$ correspond to traceless
matrices, so in these cases  the second Chern class simplfies 
\[
c_2(E) = \frac{1}{8\pi^2}\int_M (\Tr(F^2)-(\Tr(F)^2)) = 
\frac{1}{8\pi^2}\int_M \Tr(F\wedge F)
\] 
(why is $F\wedge F$ the same as  $F^2$)
Again decomposing $F=F^++F^-$, we can write  $c_2(E) = \frac{1}{8\pi^2}\int_M
(\abs[*]{F^+}^2-\abs[*]{F^-}^2)$. 

Comparing with the Yang-Mills action, 
\begin{align*}
	S_{YM}(A) = \int_M (\abs[*]{F^+}^2+\abs[*]{F^-}^2) 
	= \begin{cases}
		8\pi^2 c_2(E) + 2\int_M \abs[*]{F^-}^2 \\
		-8\pi^2 c_2(E) + 2\int_M \abs[*]{F^+}^2
	\end{cases}
\end{align*}
We see that for $c_2(E)>0$, the action is bounded below by 
$S(A) >= 8\pi^2\abs{c_2(A)}$ and the self-dual connections $F^-=0$ are
minimisers. For $c_2(E)<0$ the action is bounded
below by $S(A)\geq -8\pi^2c_2(E)$, and minimised by anti-self-dual connections
$F^+=0$. 
Thus the classical equations of motion are equivalent to $\star F = \pm F$,
whose solutions are called (anti-)instantons.

Note that $\star F = - F$ is a non-linear differential equation for
non-abelian gauge groups, and defines a subspace of the infinite dimensional
space of connections $\mathcal{A}$. This subspace can be regarded as the zero
locus of the section $s : \mathcal{A} \to \Omega^{2,+}$ given by $s(A) =
F_A+\star F_A$. The main goal is to define a finite-dimensional moduli space
starting from  $s^{-1}(0)$. 
The key property used to do this is that the section is equivariant with respect
to the action of the gauge group: $s(u^*(A))=u^*(s(A))$.

Hence if a connection is ASD, i.e. $s(A)=0$, then the transformed connection is
also ASD. The moduli space is defined by quotienting the space of ASD
connections by the action of the gauge group. 

\section{Gauge Theory Motivation}
Let $M$ be a space-time manifold. In field theory, a particle or field is 
described in terms of a function $\psi:M\to V$, where  $V$ is some vector
space typically over the complex numbers. There is an implicit choice of
``reference frame". Suppose  $P_x$ is the space of all reference frames at $x$, and
that any two frames are uniquely related by a group  $G$ of transformations,
such as rotations, e.g. $pg\in P_x$. 
By requiring a smooth concatentation of the frames, this leads to the idea of $P$ as a
 principal $G$-bundle over  $M$.

Furthermore, suppose the group has a representation
on $V$, such that if  $\psi:P\to V$ describes the value of  $\psi$ relative to
$p\in P_x$, then $\psi(pg)=g^{-1}\cdot\psi(p)$ is the value relative to $pg$. 

If  $U_{\alpha}\subset M$, then a continuous choice of a reference frame, described by a
section $\sigma:U_{\alpha}\to P$ is called a gauge. Using this gauge, we can pull $\psi$
down to obtain a local wave function on $U_{\alpha}$ given by
$\psi(\sigma(y))$. If $\sigma:U_{\beta}\to P$ is another gauge, they are related
by $\sigma_\beta(y) = \sigma_\alpha(y)g_{\alpha\beta}(y)$, so
$\psi(\sigma_\alpha(y))=g_{\alpha\beta}(y)^{-1}\cdot\psi(\sigma_\beta(y))$.
See \S Terminology for a comparison of language used in physics vs mathematics. 

By combining the representation, we can describe the field, as a section of
the vector bundle associated to $P$ by the representation  $\rho:G\to \GL(V)$,
which is of the form $\psi:M\to P\times V / (p,v) \sim (pg, g^{-1}\cdot v)$. 

In order to describe the dynamics of the field, we usually need to calculate the
space-time derivatives. But physically meaningful quantities should transform
correctly under a change of gauge (or local trivialisation). It turns out that 
fixing a gauge and taking ordinary derivatives does not work. Informally, the parallel
transport map $\Gamma(x\to x+dx) : E_x \to E_{x+dx}$ is a way to identify two
fibres of the vector bundle together, allowing us to differentiate. 
We wish this map to be approximately transitive in that 
\[
\Gamma(x+dx \to x+dx+dx') \circ \Gamma(x\to x+dx) \approx \Gamma(x\to x+dx+dx')
\] 
The precise nature of this error is described by the curvature of the
connection. The new derivative, called the connection (or covariant derivative), can be 
described locally using a connection 1-form (or gauge potential),
by $\partial_\mu+A$, where
$A$ is a matrix or vector (if $\dim V=1$) that depends on the tangent vector.

General relativity is based on
transformations of space-time itself, which we can call external symmetries.
However, in QFT we only consider transformations that change reference frames, 
but leave the points in spacetime fixed by acting on the fields of the theory, 
called internal symmetries.   
An example of this is gauge symmetries (or changes of local trivialisation),
which are an artefact of our choice of coordinates/fields. There is no
physical meaning, since all states related by such a transformation are
physically the same. 
Although it is possible to describe particles without gauge theory, by defining
local operators which create and destroy particles, with certain transformation
properties, it is easier to work with fields with extra structure.
% Weinberg 5.9 shows massless fields with helicity pm 1 cannot be described 
% by an ordinary 4-vector field but must be associated to a gauge symmetry eq31

For instance, the photon has two polarisation states, but we instead have a
connection (or gauge field) $A^{\mu}$ with four components, that transforms 
nicely under the Lorentz group $\O(3,1)$, and only consider gauge independent
quantities to be physical. In contrast, a global symmetry is a ``true" symmetry
of the system, because it corresponds to a conserved quantity via Noether's
theorem, and are physically not the same.  

Note that this thesis does not discuss the quantization of gauge potentials, and
without this, very little of physical significance, such as scattering cross
sections or particle lifetimes can be computed. % see Faddeev Slavnov 
% all of above based on bleecker
% loringtu 29.10 the connection matrix on a vector bundle under a certain frame 
%is the pullback of a certain 1-form on Fr(E) by the frame e:U to Fr(E)
% this motivates a connection on a principal bundle

\section{Introduction}

A quantum field theory is called a TQFT when the correlation functions are
independent of the metric. 
TQFT has been an active area of research ever since the seminal work by Witten
\cite{wittenTQFT}. TQFT contains no excitations that may propagate in the
spacetime, so it does not describe any waves we know in the real world. The
characteristic quantity describing a configuration: the action, remains
invariant under any continuous changes of the topology. 

There are two ways to formally guarantee that the correlation functions remain
invariant under variations of the metric. % labatista, lozano
\begin{itemize}
	\item Schwarz type TFT: the action and the operators are defined without 
		using the metric of the manifold. The most notable example is
		Chern-Simmons gauge theory. Another import set of examples are the BF
		theories. 
	\item Witten type TFT: there is explicit metric
	dependence, but the theory has a scalar symmetry $\delta$ acting on
	the fields such that the correlation
	functions do not depend on the background metric. 
\end{itemize}

This thesis will focus on a particular Witten type TQFT: Donaldson-Witten
theory. Witten-type TQFTs can be formulated in a variety of frameworks. 
\begin{itemize}
	\item 
 The most geometric one corresponds to the Mathai-Quillen formalism. In this
formalism a TQFT is constructed out of a moduli problem. Topological invariants
are then defined as integrals of a certain Euler class over the resulting moduli
space.\cite{cernTQFT}
	\item 
 A different framework is the one based on the twisting of N = 2 supersymmetry.
In this case, information on the physical theory can be used in the
TQFT. (Seiberg-Witten invariants have shown up in this framework)
\end{itemize}

\section{Classical electromagnetism}
How electromagnetism motivates Yang-Mills theory

\section{Topological Aspects of Four-manifolds}
Intersection form

moduli space of connections quotiented by gauge group

transversality

\vspace{5mm}
\hrule 
\vspace{5mm}

\textbf{Bibliographical notes}
{\small
\begin{itemize}
	\item A great overview of the main ideas in the two approaches to
		constructing a Witten type TFT is given in \citet{TQFTbook}.
\end{itemize}
}
