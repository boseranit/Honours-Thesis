
\chapter{Donaldson Invariants}
\label{chapter_donaldson}
In order to discuss the Atiyah and Jeffrey's interpretation of Witten's TQFT
using the Mathai-Quillen form, we first need to take a brief detour to introduce
the Donaldson polynomial invariants of smooth 4-manifolds. 
The details of the construction of the invariants are highly technical, which
this chapter will not be concerned with, as that would be beyond the scope of 
the thesis. 

\section{Differential forms as normed spaces}
We must first describe how to consider the space of differential forms
as a normed space, because this will be central to the concepts of this chapter.
Let $E\to M$ be vector bundle with a metric connection over an oriented 
Riemannian manifold. 
Denote $\nabla : \Omega^k(M,E) \to \Omega^{k+1}(M,E)$ to be the exterior
covariant derivative. Using the metric on $TM$ and  $E$, this induces a
metric on  $\bigwedge^k(T^*M)\otimes E$. We denote the induced norm as 
$\norm{\cdot}_g$ below, but later just as $\abs{\cdot}$.
The metric $g$ on $M$ also gives rise to the unique Riemannian volume form 
$\odif{V}_g$ (see \cite[Prop 2.41]{riemannian_manifolds}).
\begin{defn}
    For $1 \leq p < \infty$, and $s \in \Omega^l(M,E)$, 
	define the \underline{$L^p$ norm}
	\[
		 \norm{s}_{L^p} = \paren{\int_M \norm{s(x)}^p_g \odif{V}_g}^{1 /p}
	\] 
	and the \underline{Sobolev norm}
	\begin{align*}
		\norm{s}_{L^p_k} 
		&= \paren{\sum_{j=0}^{k} \norm[*]{\nabla^j s}_{L^p}^p }^{1 /p} 
		= \paren{ \int_M \norm{s(x)}^p_g + \norm{\nabla s(x)}^p_g + \cdots
		+ \norm[*]{\nabla^k s(x)}^p_g \odif{V}_g}^{1 /p} 
	\end{align*}
\end{defn}
In this definition the Sobolev metric depends on the metric on $T^*M^{\otimes
l}\otimes E$ as well as the connection on $E$. It can be shown that the Sobolev
norm is up to equivalence, independent of the choices of metrics and connections
\cite[Lemma 11.22]{math_for_physics}.
\begin{defn}
	Let $1\leq p < \infty$ and  $k\in \mathbb{Z}_{\geq 0}$. The
	\underline{Sobolev
	space} $\Omega^l_{L^p_k}(M,E)$ of $L^p_k$-sections is the completion of
	$\Omega^l(M,E)$ in the Sobolev norm  $\norm{\cdot}_{L^p_k}$.
\end{defn}

We are often interested in Lie algebra valued differential forms 
$\Omega(P,\mathfrak{g})$ on a principal bundle , so we need to define a metric 
on both $P$ and $\g$. The metric on $T_pP$ can be obtained by identifying the 
horizontal subspace induced by the connection with  $T_{\pi(p)}M$, and the vertical
subspace with $\g$. Assuming $G$ is compact, we can always construct a metric on
 $\g$ as follows.
 \begin{thm} \label{thm:lie_inner_product}
	Let $G$ be a compact Lie group, with a representation  $\rho : G \to\GL(V)$.
	Then there exits an invariant inner product on  $V$, i.e. 
	$\gen{\rho(g)v,\rho(g)w} = \gen{v,w}$ for all $g\in G$. Consequently,
	$\rho$ is an orthogonal rep, and the induced rep $\rho_*$ on
	$\mathfrak{g}$ is skew-symmetric valued. 
\end{thm}
The construction is based on choosing an arbitrary inner product on $V$, and
defining $\gen{v,w} := \int_G \gen{\rho(A)w,\rho(A)w}_V \odif{vol}(A)$, where
$\odif{vol}(A)$ is a right-invariant differential form on $G$ (we can also
construct the integral based on the right-invariant Haar measure on $G$).
In particular, this gives an $\Ad$-invariant metric on $\g$. 
This also gives an induced metric on $\ad P = P\times_{\Ad}\g$, which is well 
defined because the metric on both $P$ and  $\g$ are invariant under the action 
of  $G$. 

\section{Yang-Mills theory}
Donaldson's polynomial invariants is based on the study of the moduli space of
gauge equivalent solutions to the Yang-Mills equations on a 4-dimensional 
manifold. The solutions can be easily described using the Hodge star operator,
which we define below.
\begin{defn}
	Let $(M,g)$ be an oriented Riemannian  $n$-manifold. The \underline{Hodge}
	\underline{star operator} $\star : \Omega^k(M) \to \Omega^{n-k}(M)$ is the 
	unique smooth bundle homomorphism satisfying 
	\begin{equation} \label{eq:hodge_property}
	\omega \wedge \star\eta = \gen{\omega,\eta}_g dV_g
	\end{equation}
	where $\gen{\cdot,\cdot}_g$ is the induced metric on $\bigwedge^k(T^*M)$.
	In orthonormal coordinates, this is given by
	\[
	\star(dx_{i_1}\wedge \cdots\wedge dx_{i_k}) = \sgn(I)dx_{i_{k+1}}\wedge \cdots \wedge
	dx_{i_n} 
	\] 
	where $dx_1\wedge\cdots\wedge dx_n = dV_g$ on some open subset.
\end{defn}
The Hodge star extends to vector valued forms by only acting on the
$\bigwedge^k(T^*M)$ part. 
We are interested in the Hodge star acting on 2-forms on a 4-manifolds in
particular, since $\star : \Omega^2(M,\g)\to\Omega^2(M,\g)$ is a linear operator
with $\star^2= 1$. The two eigenvalues 1 and -1 allow us to decompose into its
eigenspaces $\Omega^2(M,\g) = \Omega^{2,+}(M,\g) \oplus \Omega^{2,-}(M,\g)$,
called self-dual (SD) and anti-self-dual (ASD) two-forms respectively.


\subsection{General definition}
Let $P\to M$ be a principal $G$-bundle over an $n$-dimensional Riemannian 
manifold $M$, with connection $A$ and associated curvature $F$. 
Assume $G$ is a compact Lie group,
though we will typically choose $G=\SU(2)$ or $G=\SO(3)$. 
Since the curvature form of a principal bundle is horizontal and  
$\Ad$-equivariant, we can interpret it as an $\ad(P)$-valued form on the base 
$F\in\Omega(M,\ad P)$.
The Yang-Mills functional is defined by the $L^2$ norm squared of the curvature 
\[
	S_{YM}(A) := \norm{F}^2_{L^2} = \int_M \abs{F}^2 dV_g
\]
where $\abs{F}^2$ comes from the metric on $\bigwedge^2T^*M\otimes \ad P$. 
The significance in physics is that this represents the action of a free field, 
generalising electromagnetism for non-abelian gauge groups.\cite[p. 277]{baez}  

The principal of least action in physics dictates that the classical solutions
are the connections that satisfy the Euler-Lagrange equations of this action 
functional, that is, locally minimise $S_{YM}$. Given a perturbation
$A+ta$ of a connection  $A$ in the affine space $\Omega^1(M,\ad P)$, the 
curvature is related by
\[
F_{A+ta} = d_A(A+ta) + (A+ta)\wedge (A+ta) = F_A + td_A a + t^2a\wedge a
\] 
Therefore,
\begin{align*}
	&\odv{}{t}_{t=0}S_{YM}(A+ta)
	= \odv{}{t}_{t=0} \int_M \abs{F_A + td_A a + t^2a\wedge a}^2 \\
	&= \odv{}{t}_{t=0} \int_M \paren{\abs{F_A}^2 + 2t\gen{F_A,d_A a} 
	+ 2t^2\gen{F_A,a\wedge a} + t^4 \abs{a\wedge a}^2} dV_g \\
	&=  2\int_M  \gen{F_A,d_A a}  dV_g 
	=  2\int_M  \gen{d_A^*F_A, a}  dV_g 
\end{align*}
where $d_A^* :\Omega^{k}(M,\ad P)\to \Omega^{k-1}(M,\ad P)$ is the formal adjoint of 
$d_A$ using the $L^2$ inner product on
$\Omega^k(M,\ad P)$, defined by $\gen{d_A s, t}_{L^2} = \gen{s,d_A^*t}_{L^2}$.
It can be shown that its explicit form is $d_A^* = (-1)^{n(k+1)-1}\star d_A \star$.
The equation above shows that the connection $A$ is a critical point if and only
if  $d_A^* F_A = 0$, or $d_A \star F_A = 0$. A connection satisfying this
condition is called a Yang-Mills connection.
If $\star F_A = \pm F_A$, then by the Bianchi identity it is a YM connection.

\subsection{Orthogonal or unitary structure group}
In this subsection we assume $G=\SO(N)$ or  $G=\SU(N)$, which allows us to
make a number of simplifications.
We define the trace wedge for matrix Lie algebra valued forms 
$\omega,\eta\in\Omega(M,\g)$ by  
\begin{equation} \label{eq:matrix_form_wedge}
    \Tr(\omega\wedge\eta) = \sum_{j,k} \omega_j\wedge \eta_k \Tr(X_j X_k)
\end{equation}

For an arbitrary Lie algebra valued form, equation (\ref{eq:hodge_property})
will not hold. But for the case where $G=\SO(N)$ or  $\SU(N)$, we can choose the
$\Ad$-invariant Hilbert-Schmidt metric on $\g$ defined by
$\gen{X,Y}=\Tr(X^*Y)$ where $X^*$ is the adjoint. 
In this case, if $\alpha,\beta \in \Omega^k(M,\g)$, we have 
\begin{align} 
	-\Tr(\alpha \wedge \star \beta)
	%&= -\Tr(\alpha^+ \wedge \beta^+) +\Tr(\alpha^- \wedge \beta^-) \nonumber \\
	%&= \gen[*]{\alpha^+,\beta^+}dV_g + \gen[*]{\alpha^-,\beta^-}dV_g \nonumber \\
	&= \gen{\alpha,\beta}dV_g \label{eq:trace_hodge}
\end{align} 
which follows from expanding using equation (\ref{eq:matrix_form_wedge}),
applying the property property $\Tr(X^*Y)=-\Tr(XY)$ for $X,Y\in\g$, and
finally using equation (\ref{eq:hodge_property}). 
Hence the Yang-Mills functional can be rewritten using equation (\ref{eq:trace_hodge}) as
\[
	S_{YM}(A) 
	=-\int_M \Tr (F\wedge \star F) 
\] 
\noindent
Recall from Chern-Weil theory that there are two important classes of
$\Ad \GL(r,\mathbb{C})$ invariant polynomials on $\gl(r,\mathbb{C})$. 
The first is the coefficients $f_k(X)$ of $\lambda^{r-k}$ in the characteristic
polynomial $\det(\lambda I + X)$. Another class consists of the trace
polynomials $\Tr(X^k)$. 
The Chern classes of $P$ are defined by $c_k(P) =
[f_k\paren{\frac{i}{2\pi}F}] \in H^{2k}(M)$ and is independent of the choice of
connection. Their importance is due to the
following theorem:
\begin{thm}[{\cite[Theorem E.5]{freed_uhlenbeck}}]
	The second Chern class $c_2(E)\in H^4(M,\mathbb{Z})$ classifies up to
	isomorphism $\SU(2)$-bundles over any compact connected oriented 
	four-manifold $M$. 
\end{thm}

From Newton's identity for symmetric polynomials with $k=2$ 
(see \cite[Theorem B.2]{loringtu}), we have $\Tr(X^2)-f_1(X)\Tr(X)+2f_2(X)=0$, 
giving the following relation for any complex vector bundle
\[ % 2.1.28 DK
	\bracket{\frac{1}{8\pi^2} \Tr(F^2)} = c_2(E) - \frac{1}{2}c_1(E)^2 \in H^4(M)
\] 
where we have used the fact that $f_1(X)=\Tr(X)$. 
In the case where the structure group is $\SU(n)$ or $\SO(n)$, the trace of the
curvature is zero, so $c_1(E)=0$ is trivial and we can define 
\begin{equation} \label{eq:k}
	k := c_2(E)[M] = \frac{1}{8\pi^2}\int_M \Tr(F^2)
\end{equation}
Now, we can
decompose the field strength $F = F^+ + F^-$ into SD and ASD 
parts. Using the fact that $\Omega^{2,+}$ and $\Omega^{2,-}$ are orthogonal, 
$\Tr(X^*Y) = -\Tr(XY)$ and the properties of the hodge star,
this gives 
\begin{align}
k
= \frac{1}{8\pi^2}\int_M\Tr(F^2) 
&= \frac{1}{8\pi^2}\int_M(\Tr(F^{+}\wedge  F^{+}) + \Tr(F^{-}\wedge F^{-}))
\nonumber \\
&= \frac{1}{8\pi^2}\int_M(-\abs[*]{F^+}^2 + \abs[*]{F^-}^2) dV_g 
\label{eq:second_chern}
\end{align}
Comparing with the Yang-Mills action, 
\begin{align*}
	S_{YM}(A) = \int_M (\abs[*]{F^+}^2+\abs[*]{F^-}^2) dV_g 
	= \begin{cases}
		8\pi^2 k + 2\int_M \abs[*]{F^+}^2 dV_g\\
		-8\pi^2 k + 2\int_M \abs[*]{F^-}^2 dV_g
	\end{cases}
\end{align*}
We see that for $k>0$, the action is bounded below by 
$S(A) \geq 8\pi^2k$ and the ASD connections $F^+=0$ are
minimisers. For $k<0$ the action is bounded
below by $S(A)\geq -8\pi^2k$, and minimised by SD connections
$F^-=0$. 
Thus the classical equations of motion are equivalent to $\star F = \pm F$,
whose solutions are called (anti-)instantons. 
\section{Space of connections}
\begin{prop}
	Let $E_1,\ldots,E_k$ and $F$ be vector bundles over a manifold  $M$. 
	There is a bijection
	\[
	\set{
		\begin{array}{c}
			C^\infty\text{-multilinear maps} \\
			T:\Gamma(E_1)\times \cdots \times \Gamma(E_k) \to \Gamma(F)
		\end{array}
	} \longleftrightarrow
	\set{
		\begin{array}{c}
			\text{sections}\\
			  \Gamma(M,E_1^*\otimes \cdots \otimes E_k^*\otimes F)
		
		\end{array}
	}
	\] 
\end{prop}
\begin{prop}
	If $\nabla^A,\nabla^B : \mathcal{X}(M) \times \Gamma(E) \to \Gamma(E)$ are
	two connections on a vector bundle $E\to M$, then 
	 \[
	\nabla^A - \nabla^B \in \Omega^1(X,\End E)
	\] 
	Conversely, given $a\in \Omega^1(X,\End E) \simeq \Hom(T^*M\otimes E,E)$, 
	then $\nabla^A+a$ interpreted as 
	\[
		(\nabla^A + a)_X(s) = \nabla^A_X s + a(X\otimes s)
	\] 
	is again a connection on  $E$. This proposition is summarised by the
	statement: the space of connections is an affine space modelled on
	$\Omega^1(X,\End E)$. 
\end{prop}
In one direction, it is easy to show $\nabla^A-\nabla^B$ is tensorial (meaning
$C^\infty(M)$ linear in both components), and apply the previous proposition. In the other
direction, we only need to verify the Leibniz rule. Thus, a choice of reference 
connection on $E$ defines a bijection between the space of all connections and
$\Omega^1(M,\End E)$. There is a similar result for principal bundles.

\begin{prop} \label{prop:connection_space}% Lemma 2.9.2 MF
	The space of connections $\mathcal{A}$ on a principal $G$-bundle $P\to M$ 
	is an affine space modelled on 
	the vector space $\Omega^1(M,\ad P) \simeq \Omega^1(M,P\times_{\Ad} \mathfrak{g})$.
\end{prop}
This follows from the fact that the difference of two connections
$\theta_1-\theta_2\in \Omega^1(P,\g)$ is a horizontal and $\Ad$-equivariant form.
Finally we identify this space as $\Omega^1(M,P\times_{\Ad}\g)$, by 
the isomorphism between these spaces (see \cite[Theorem 31.9]{loringtu}). 
Conversely, the sum of $\theta_1$ with 
a horizontal and $\Ad$-equivariant form is again a connection on  $P$.

% p33 DK
Given a real or complex vector bundle $E$ of rank $n$, we can always construct 
its frame bundle with structure group $\GL(n)$ (or $\GL(n,\mathbb{C})$).
Additional algebraic structure on $E$ yields a principal bundle with smaller
structure group. For example,
if $E$ is a complex vector bundle with a Hermitian metric, we get a principal
$\U(n)$-bundle of orthonormal frames in  $E$. In these cases, we can identify
the space of connections on $E$ as $\Omega^1(M,\mathfrak{g}_E)$, where
the restriction to $\g_E \subset \End E$ ensures that the new connection is
compatible with the structure group of  $E$.
If the structure group is $\SU(2)$, for example, then
$\g_E$ consists of skew-adjoint, trace-free endomorphisms of the rank two vector
bundle $E$. The point is, we may view connections on principal bundles as a
generalisation of the vector bundle case, so we work with the former.

\begin{prop} \label{prop:adP_connection}
	To every connection $\omega \in \Omega^1(M,\ad P)$ on a principal
	$G$-bundle $P\to M$, there is an associated vector bundle connection
	$d_\omega$ on  $\ad P \to M$, where  $\ad P$ acts on itself by the fiberwise 
	Lie bracket. Assume $G$ is a matrix Lie group.
\end{prop}
\begin{proof}
	Let $v\in \mathcal{X}(M)$ and $\lambda\otimes X \in \Gamma(\ad P)$, where
	$\lambda\in C^\infty(M),X\in\ad P$. We define the 
	connection by $d_\omega(v,\lambda\otimes X) = 
	D\lambda(v) \otimes X + \lambda [\omega(v),X]_{\ad P}$. Here
	the Lie bracket on elements of $\ad P$ is defined by 
	\[
		[[p,A],[p,B]]_{\ad P} := [p,[A,B]_{\g}]
	\] 
	Then $d_\omega$ is $C^\infty$-linear in  $v$, and linear in  $\lambda$. 
	The fiberwise Lie bracket is well defined because 
	\[
		[p\cdot g, [g^{-1}Ag,g^{-1}Bg]] 
		= [p\cdot g, g^{-1}[A,B]g]
		= [p, gg^{-1}[A,B]gg^{-1}] = [p,[A,B]]
	\] 
	Finally, $d_\omega$ satisfies the Leibniz rule by construction. 
\end{proof}

\begin{defn}
	The \underline{gauge group} $\mathcal{G}$ of a vector bundle $E\to M$ is the group of all
	vector bundle automorphisms $\Aut(E)$. Similarly, the gauge group of a
	principal bundle $P\to M$ is the group of principal bundle automorphisms
	$\Aut(P)$.
\end{defn}

\begin{prop} \label{prop:gauge_trans_space} % Lemma 4.1.2 Morgan
	The group of gauge transformations $\mathcal{G}=\Aut(P)$ is isomorphic to 
	sections $\Omega_{\Ad}^0(P,G)\simeq \Omega^0(M,\Ad P)$ under fiber-wise 
	multiplication 
\end{prop}
\begin{proof}
	A gauge transformation $f : P \to P$ preserves the fibers of  $P$ and
	satisfies  $f(p\cdot g) = f(p) \cdot g$. For a fixed $p\in P$, we have 
	$f(p) = p\cdot \psi(p)$ for some $\psi(p)\in G$ because the action is
	transitive. Then $f$ acts by multiplication 
	by $\psi(p)$ on the whole fiber. Hence, $f(p) =
	p\cdot \psi(p)$ for a smooth function  $\psi : P \to G$. 

	Substituting this into $f(p\cdot g) = f(p)\cdot g$, we find  $\psi(p\cdot g)
	= g^{-1}\psi(p) g$ since the action is free. So $R_g^*\psi =
	\Ad_{g^{-1}}\psi$. Hence, $\psi \in \Omega^0_{\Ad}(P,G) \simeq
	\Omega^0(M,P\times_{\Ad}G)$. This isomorphism can be proved in the same way
	as the associated vector bundle case.
	Finally, note that composition of automorphisms corresponds to
	multiplication of $\psi$ in the fibers of groups. 
\end{proof}
The action of $f \in \Aut(P)$ on a connection  $\omega\in \Omega^1(P,\g)$ is
$f^*\omega \in \Omega^1(P,\g)$. Consequently, the action of $\Omega^0(M,\Ad
P)$ on  $\Omega^1(M,\ad P)$ is still denoted the same way.
% lemma 4.3.1 Morgan u^*`w is a connection one form
% The stabiliser of a connection A are the sections which are horizontal in A

The purpose of Propositions \ref{prop:gauge_trans_space} and
\ref{prop:connection_space} is two-fold: this description allows us to 
define Sobolev completions of these spaces; the other 
purpose is to identify $\mathcal{G}$ as an infinite dimensional Lie group with 
Lie algebra $\Omega^0(M,\ad P)$.
The gauge group $\mathcal{G}$ has  
the fiberwise exponential map $\exp : \Omega^0(M,\ad P) \to
\Omega^0(M,\Ad P)$ which assigns to any section  $\sigma \in \Gamma(\ad P)$ the
section  $s(x)=\exp(\sigma(x))$. To see that this gives a well defined map, we
use the fact that  $\exp : \g \to G$ satisfies
$\exp(g^{-1}Ag)=g^{-1}\exp(A)g$. The Lie bracket on $\Omega^0(M,\ad P)$ is
defined by the fiberwise bracket, which we have shown is well defined. 
 

\section{Moduli space}
Before defining the Donaldson invariants, we study the structure of the moduli 
space, which will allow us to understand the invariants better. However, the details
rely on elliptic theory and is quite involved, so we only sketch the arguments.

% 4.4 MF
The first step is to form the quotient space $\mathcal{A} /\mathcal{G}$ of gauge 
equivalent connections.  
Singularities in this quotient space are characterised by reducible connections,
as defined below.
\begin{defn} % freed uhlenbeck p54
	A connection $\omega\in \Omega(P,\g)$ on a principal $G$-bundle  $P\to M$ 
	is \underline{reducible} if the gauge group $\mathcal{G}$ modulo its center 
	does not act freely on the connection $\omega$.
\end{defn}
% freed p36
\noindent
For the group $G=\SU(2)$, the center of 
$\mathcal{G}\simeq \Gamma(P\times_{\Ad} \SU(2))$ is $Z:=\Gamma(P\times_{\Ad}
\mathbb{Z}_2)$, since $\mathbb{Z}_2$ is the center of $\SU(2)$. 

% also corollary 4.3.5 and section 4.5 morgan
\begin{thm}[{\cite[Theorem 3.1]{freed_uhlenbeck}}] % Freed, Uhlenbeck 3.1, 10.8
	Suppose $\omega$ is a connection on the principal $\SU(2)$-bundle  $P\to M$ 
	which is not flat. Then the following are equivalent:
	\begin{enumerate}[(a)]
		\item  $\omega$ is reducible, i.e.  $\mathcal{G}_\omega / Z \neq 1$, 
			where $\mathcal{G}_\omega$ is the stabiliser of $\omega$
	    \item $\mathcal{G}_{\omega} / Z \simeq U(1)$ 	
		\item $\nabla : \Omega^0(\ad P) \to \Omega^1(\ad P)$ has a nonzero
			kernel, where  $\nabla$ is the induced covariant derivative
		\item For any associated bundle $\eta = P\times_G V$ and induced 
			connection $D$,  the bundle $\eta = \eta_1 \oplus \eta_2$ and the
			connection $D= d_1\oplus d_2$ both split.
	\end{enumerate}
\end{thm}

Let us denote by $\mathcal{A}^*$ the space of irreducible connections on a
principal bundle, and $\mathcal{B}^* := \mathcal{A^*} / \mathcal{G}$ the
topological quotient by the gauge group.  
The classical equation of motion $\star F = - F$ is a non-linear differential 
equation for non-abelian gauge groups, and defines a subspace of the infinite
dimensional space of connections $\mathcal{A}$.
The key property is that ASD connections are preserved by the
action of the gauge group. 
Therefore, ASD connections are a subspace of 
$\mathcal{B}^*$, which we call the \underline{moduli space} $\mathcal{M}^*$.
\begin{remark}
If we reverse the orientation of $M$, then this swaps the SD and ASD forms in
$\Omega^2(M,\ad P)$. Since the two theories are completely
equivalent, we could work with SD connections. However, there is an
important class of 4-manifolds which have a natural orientation - complex
manifolds. Over this orientation, it turns out that ASD connections are 
associated to holomorphic objects, while SD connections to anti-holomorphic
objects.\cite[p.95]{morgan} For this reason, we choose to work in terms of 
ASD connections by default.
\end{remark}


By working with Sobolev completion
spaces, $\mathcal{M}^*$ can be shown to have a manifold structure by 
application of the slice theorem. Denote
\begin{itemize}
	\item the vector space of $L^2_l$ $\ad P$-valued forms by 
$\Omega^*_{l}(M,\ad P)$. 
	\item the space of $L^2_l$-connections by $\mathcal{A}_l(P)$ which is
		modelled on $\Omega^1_l(M,\ad P)$. 
	\item the group of $L^2_l$-gauge transformations by $\mathcal{G}_l(P) =
		\Omega^0_l(M,\ad P)$  
\end{itemize}
The Sobolev spaces of sections are Banach manifolds, and 
the $L^2_k$ spaces are in fact Hilbert manifolds with the inner product defined
similarly to the $L^2$ norm. 
\begin{comment}
However, we could work with
any Sobolev spaces for which the group of gauge transformations has at least two
derivatives, and the space of connections has one fewer derivative. 
\end{comment}

\begin{thm}[{\cite[Section 4.5]{morgan}}] 
	% Morgan 4.3.5 stabilisers, 4.4.5 local slices, and section 4.5
	The space of gauge equivalent irreducible connections
    on a principal $\SU(2)$-bundle
	$\mathcal{B}^*_2(P) := \mathcal{A}^*_2(P) / \mathcal{G}_3(P)$ is a 
	Hilbert manifold. 
\end{thm}

To show that the moduli space has a manifold structure, there is more work to
do. Let us clarify one point straight away: the reason we can analyse the ASD 
subspace of $\mathcal{B}_2^*$ in place of smooth connections is that for any $L^2_l$ 
connection $A$, there is a $L^2_{l+1}$ gauge transformation $u$ such that
$u(A)$ is a smooth connection.\cite[Prop 4.2.16]{donaldson_kronheimer} 
Therefore, the moduli spaces  $\mathcal{M}^*$ 
and $\mathcal{M}^*_l$ are actually the same.
Since we will not be dealing with the analysis related to the properties of the
Sobolev spaces, from this point forward we will drop the subscripts.
We now establish a couple of basic properties of the curvature operator and gauge
group actions, that will be important in the next chapter.

\begin{comment}
The space of gauge connections $\mathcal{A}$ on a principal bundle is the
universal bundle for the group of gauge transformations $\mathcal{G}$. (ch 15
Cordes)
\end{comment}


% freed, uhlenbeck p54, A.4
\begin{prop}[{\cite[p.54]{freed_uhlenbeck}}] \label{prop:curvature_derivative}
Fix a reference connection $\omega_0$ to identify
$\mathcal{A}$ with the affine space $\Omega^1(M,\ad P)$. 
The curvature operator $F:\Omega^1(M,\ad P) \to \Omega^2(M,\ad P)$ is smooth, and
its differential at $\omega$ is the covariant derivative
\[
d_{\omega} : \Omega^1(M,\ad P) \to \Omega^2(M,\ad P)
\]
\end{prop}
\noindent
This is not hard to see, because at $\omega + tA$,
\[
F(\omega + tA) = F(\omega) + t d_{\omega}A + t^2 A\wedge A 
\] % since $\omega_0 \wedge A + A\wedge\omega_0 = 0$
since the exterior derivative $d$ on $\Omega(P,\g)$ corresponds to the covariant
derivative $d_{\omega}$ on $\Omega(M,P\times_{\Ad}\g)$. 

\begin{prop} \label{prop:gauge_derivative} % DK p 131 
	Fix $\omega \in \mathcal{A}$ as a reference connection. 
	The derivative of the gauge group action 
	$L_\omega : \mathcal{G} \to \mathcal{A}$, where $L_\omega(f) = f^*\omega$, 
	is given by 
	\[
	DL_\omega|_{\id}(\phi) = \odv{}{t}_{t=0} L_\omega(\exp(t\phi)) = d_\omega \phi
	\] 
	where $d_\omega : \Omega^0(M,\ad P) \to \Omega^1(M,\ad P)$ is the covariant
	derivative.
\end{prop}
\begin{proof}
	Viewing $\omega$ as a connection on  $\ad P \to M$, 
	consider the connection 1-form $A_\alpha \in \Omega^1(U_\alpha,\gl(\g))$ in a 
	local trivialisation for $\ad P$. Any $u \in \Omega^0(M,\Ad P)$ acts as a 
	gauge transformation of $\ad P$, and the connection 1-form transforms as 
	\[
	u_\alpha^*A_\alpha = (du_\alpha) u_\alpha^{-1} + u_\alpha^{-1}A_\alpha u_\alpha
	\] 
	where $u_\alpha : U_\alpha \to\GL(\g)$ is a matrix valued function 
	under the trivialisation.  
	Therefore, the derivative of the gauge action at $\phi\in \Omega^0(M,\ad P)$ is 
	\begin{align*}
		\odv{}{t}_{t=0} (\exp(t\phi)^* A_\alpha)
		&= \odv{}{t}_{t=0} (d(\exp(t\phi))\exp(-t\phi)
		+ \exp(-t\phi)A_\alpha\exp(t\phi)) \\
		&= d\phi -\phi A_\alpha + A_\alpha\phi \\
		&= d\phi + [A_\alpha, \phi] 
	\end{align*}
	which is precisely the covariant derivative $d_\omega \phi$ under the trivialisation.
\end{proof}

The strategy to work out the local structure of $\mathcal{M}^*$ is to apply 
an infinite dimensional version of the implicit function theorem. 
Roughly, it states that the
zero set of a smooth Fredholm map between Hilbert manifolds is locally 
isomorphic to a subset defined
by a smooth map between finite dimensional manifolds (see \cite[Lemma
5.2.1]{morgan} for the precise statement). 


% naber witten conjecture p69-73, naber p117-119, physical p25
If $d^*_\omega$ is the formal adjoint of $d_\omega : \Omega^k(M,\ad P) \to
\Omega^{k+1}(M,\ad P)$, then it can be shown that $d^*_\omega \circ d_\omega$ is
an elliptic operator. 
The structure of the moduli space is unraveled by studying what is called the
fundamental elliptic complex $\mathcal{E}_\omega$ associated with an ASD 
connection $\omega$:
\[
\begin{tikzcd}
	0 \arrow[r] &\Omega^0(M,\ad P) \arrow[r,"d_\omega"] 
				&\Omega^1(M,\ad P) \arrow[r,"d^+_\omega"] &\Omega^{2,+}(M,\ad P)
\end{tikzcd} 
\]
where $d_\omega^+$ denotes the covariant derivative folowed by projection to the 
self-dual subspace.
The generalised Hodge decomposition guarantees that the cohomology groups of this
complex $H^*(\mathcal{E}_\omega)$ are finite dimensional. 
The first linear operator is the differential of the action of $\mathcal{G}$ on
$\mathcal{A}$ by Proposition \ref{prop:gauge_derivative}, thus its image is
the tangent space to the $\mathcal{G}$-orbit through $\omega$. The second
operator is the differential of the self dual curvature map at  $\omega$ by
Proposition \ref{prop:curvature_derivative}, thus
its kernel is the formal tangent space of the ASD subspace of $\mathcal{A}$ 
at $\omega$. 
Hence, the first cohomology $H^1(\mathcal{E}_\omega)$ is the formal tangent
space to $\mathcal{M}$ at $[\omega]$.

\begin{comment}
% (generalised hodge decomposition theorem)	
It also gives us an orthogonal decomposition 
\[ % morgan p 97
\Omega^1(M,\ad P) = \Im(d_\omega) \oplus \ker(d^*_\omega)
\] 
$\Im(d_\omega)$ is the tangent space to the orbit $\omega \cdot \mathcal{G}$ at
$\omega$. 
If $\omega$ is irreducible, there is a sufficiently small neighbourhood of 
$\omega$ in  $\ker(d^*_\omega)$ (thought of as a subset of connections)
which intersects an orbit no more than once, so projects injectively into the
moduli space. So it provides a local model for the moduli space.
\end{comment}

% Morgan Corollary 5.2.6
The Hodge decomposition theorem also allows us to deduce that $d_\omega^+$
restricted to $\ker d_\omega^*$ is a
Fredholm operator, i.e. its kernel and cokernel are finite dimensional. 
Let $\omega\in \mathcal{A}^*$ be an irreducible ASD connection. 
Then by application of the implicit function theorem, one can show
there is a mapping 
\[
\Psi : H^1(\mathcal{E}_\omega) \to H^2(\mathcal{E}_\omega)
\] 
such that $\Psi^{-1}(0)$ is identified with a
neighbourhood of $[\omega] \in \mathcal{M}^*$. In the case where 
$H^2(\mathcal{E}_\omega)$ is zero, the moduli space is a smooth manifold near
$[\omega]$ of dimension equal to  $\dim H^1(\mathcal{E}_\omega)$. This can be
computed via the Atiyah-Singer Index Theorem, which tells us that 
the dimension is $8c_2(P) - 3(b_0(M) - b_1(M) + b_2^+(M))$, where $b_i(M)=\dim
H^i(M,\mathbb{R})$ and  $b_2^+(M)=\dim H^2_+(M,\mathbb{R})$, i.e. the 
dimension of the maximal positive definite
subspace of $H^2(M,\mathbb{R})$ under the cup product.

Unfortunately, the moduli space is not compact, so Donaldson constructed what is
called the Uhlenbeck compactification $\overline{\mathcal{M}}(P)$ of the
moduli space. Donaldson showed that sequences of ASD connections can have
curvatures which blow up near a finite number of points, but elsewhere converge
to an ASD connection. The compactification is constructed by adjoining these
limit connections paired with the blowup points to the moduli space. After
defining the appropriate topology, this becomes compact.

\section{Donaldson invariants}
% Lecture 7 morgan
Finally, our goal is to sketch the definition of the Donaldson polynomials
invariants. 
In this section, assume $M$ is a closed,
oriented, simply connected smooth 4-manifold with a fixed orientation of
$H^2_+(M;\mathbb{R})$ (which induces one of the moduli space), and 
$P\to M$ is a principal $\SU(2)$-bundle with  $c_2(P) > 0$. Define 
\[
d = 4c_2(P) - \frac{3}{2}(1+b_2^+(M))
\] 
which is half of the formal dimension of the moduli space
$\mathcal{M}^*(P)$. The Donaldson invariant associated to $P$ is a
symmetric multilinear function of degree  $d$ on $H_2(M;\mathbb{Z})$ 
\[
D_{M,d} : H_2(M;\mathbb{Z})^{\otimes d} \to \mathbb{Q}
\] 
Very roughly, the idea behind the definition is as follows: The Donaldson $\mu$ 
map $\mu: H_2(M,\mathbb{Z}) \to H^2(\mathcal{M}^*(P),\mathbb{Z})$ associates
homology classes in $M$ to cohomology classes in the moduli space. 
This map can then be extended to take values in
$H^2(\overline{\mathcal{M}}(P),\mathbb{Z})$, which we call $\overline{\mu}$.
For $\gamma_1,\ldots,\gamma_d \in H_2(M,\mathbb{Z})$, we have 
$\overline{\mu}(\gamma_1),\ldots,\overline{\mu}(\gamma_d)\in 
H^2(\overline{\mathcal{M}}(P),\mathbb{Z})$ 
and we define the invariant to be the integral over the compactified moduli space
 \[
D_{M,d}(\gamma_1,\ldots,\gamma_d) 
= \int_{\overline{\mathcal{M}}(P)} \overline{\mu}(\gamma_1)\wedge \cdots\wedge
\overline{\mu}(\gamma_d)
\] 
Hence the Donaldson invariant can be interpreted as 
an element in the polynomial algebra of $H^2(M,\mathbb{Z})$, i.e. dual to
$H_2(M,\mathbb{Z})$.  
% naber Witten II p70 TODO reference
To be more precise, $\overline{\mathcal{M}}(P)$ is generally not a manifold, and the above
refers to the pairing of the cup products of $\mathcal{\mu}(\gamma_i)$ with the 
fundamental homology class $[\overline{\mathcal{M}}(P)]$. One finds that
$\overline{\mathcal{M}}(P)$ admits a fundamental class only if $k$ is in what is 
called the stable range of $M$:  $k > \frac{3}{4}(1+b_2^*(M))$. 
Refer to equation \ref{eq:second_chern} for the definition of $k$.

% naber donaldson theory p59
The definition of the $\mu$ map relys on a certain $\SO(3)$-bundle, which we now
sketch.  Consider the principal $\mathcal{G}$-bundle
\[
\mathcal{A}^* \times P  \to \mathcal{A}^* \times_\mathcal{G} P 
\]
where the action is given by $(\omega,p)\cdot f = (f^*\omega, f^{-1}(p))$.
This action is free because the elements of $\mathcal{A}^*$ are irreducible:
$(f^*\omega,f^{-1}(p)) = (\omega,p)$ implies $f$ is a stabiliser of  $\omega$,
and  $f^{-1}(p)=p$ leaves only the possibility that $f=\id$.

There is a natural projection $\mathcal{A}^* \times_\mathcal{G} P \to
\mathcal{B}^*\times M$ given by $[\omega,p] \mapsto ([\omega],\pi(p))$.
The $\SU(2)$ action on  $\mathcal{A}^*\times P$ given by multiplication on $P$
descends to the quotient $\mathcal{A}^* \times_\mathcal{G} P$ because it 
commutes with the  $\mathcal{G}$ action.
But this action $[\omega,p]\cdot g = [\omega,p\cdot g]$ is not free because 
$g$ is a stabiliser if and only if $g=\pm \id\in \SU(2)$. Thus, there is a free
$\SU(2) /\{\pm\id\} \simeq \SO(3)$ action and we have a principal $\SO(3)$-bundle
\[
\xi := \mathcal{A}^*\times_{\mathcal{G}} P \to \mathcal{B}^*\times M
\] 
As an $\SO(3)$-bundle, it can be shown that the first Pontryagin class
$p_1(\xi) \in H^4(\mathcal{B}^*\times M,\mathbb{Z})$ is divisible by
4.\cite[Lemma 7.2.1]{morgan} 
Recall that the slant product in algebraic topology is a map 
\[
H^{p+q}(X\times Y) \times H_p(Y) \to H^q(X), \qquad 
(x,\alpha) \mapsto x / \alpha
\] 
Then we can define the map $H_2(M,\mathbb{Z})\to H^2(\mathcal{B}^*,\mathbb{Z})$
given by $\gamma \mapsto -\frac{1}{4} p_1(\xi) / \gamma$. 
The reason for the $-1 /4$ is
that when an $\SO(3)$-bundle $\mathcal{P}$ lifts  to an $\SU(2)$-bundle
$\mathcal{P}'$, we have the relation % TODO reference
$c_2(\mathcal{P}')=-\frac{1}{4}p_1(\mathcal{P})$. Finally, the $\mu$ map is
defined as the restriction of $\mathcal{B}^*$ to the moduli space:
 \[
\mu : H_2(M,\mathbb{Z}) \to H^2(\mathcal{M}^*,\mathbb{Z}),\qquad
\mu(\gamma) = -\frac{1}{4}p_1(\xi) /\gamma
\] 
The extension of this map to the Uhlenbeck compactification 
$\overline{\mathcal{M}}(P)$ requires the Taubes
gluing procedure, for details refer to \cite{morgan} or \cite{donaldson_kronheimer}.




\vspace{5mm}
\hrule 
\vspace{5mm}

\begin{comment}
		\item A great overview of the main ideas in the two approaches to
		constructing a Witten type TFT is given in \citet{TQFTbook}.
\end{comment}

\textbf{Bibliographical notes}
{\small
\begin{itemize}
	\item The local model for $\mathcal{M}$ obtained from the elliptic complex 
		was first studied by \citet{local_moduli}.
	\item The geometry of the the moduli space, and the theory leading up to the
		Donaldson polynomial invariants are
		discussed in more detail in \citet{morgan} and
		\citet{donaldson_kronheimer}. 
	\item A succint review of Donaldson theory and the invariants can be found
		in the conference paper by \citet{naber_donaldson}.
\end{itemize}
}
